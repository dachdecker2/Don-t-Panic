
	\subsection{Wer sind wir?}
		\glqq\textit{ Wir befinden uns im Jahre 2015 n.Chr. Die ganze Informatik ist von der Faulheit besetzt... Die ganze Informatik? Nein! Ein von unbeugsamen Freiwilligen bevölkerter Raum hört nicht auf, dem Eindringling  Widerstand zu leisten.} \grqq \\
		Spaß beiseite. Wir sind die aktive Fachschaft der Informatik. Aktiv, weil alle Studenten Teil der Fachschaft sind. \\
		Grundsätzlich kann man aber über uns sagen, dass wir ein versprengter Haufen von verrückten, abgedrehten Nerds, Geeks und anderer Lebensformen in den Augen von Außenstehenden sind, dies aber teilweise auf Vorurteilen  
	beruht. Welche davon stimmen und welche nicht, dürft ihr gerne selber herausfinden. Wir bezeichnen uns aber gerne als normal. 	
	\subsection{Was machen wir}
		Das grundlegende Vorurteil gegenüber eine Informatik-Fachschaft besteht darin, dass wir ständig am Trinken und am Rollenspiele spielen sind. \\
		Damit die Wahrheit ans Licht kommt, müssen wir aber sagen, ja das stimmt teilweise (welcher Teil nun stimmt, bleibt wieder euch überlassen). Aber das ist nur ein kleiner Teil unserer Arbeit.\\
		Der andere große Teil betrifft hauptsächlich den Universitätsalltag.\\ Viel Zeit verbringen wir z.B. mit Gremienarbeit, wo wir derzeit die studentischen Vertreter stellen. Zu diesen Gremien zählen derzeit:
		\begin{itemize}
			\item I-Rat - die offizielle Ger\"uchtek\"uche
			\item FBR - wo man zusieht, dass keiner am Fachbereich Mist macht.
			\item Lust-Ausschuss - weil's Spass macht
			\item Prüfungsausschüsse - sind halt wichtig, und so.
			\item QSL-Mittelausschuss - Geld verteilen muss auch sein.
			\item FSR - das sind wir offiziell.
			\item FSK - und das sind andere Fachschaften.
		\end{itemize}
		Zu unseren weiteren Aufgaben zählt auch, dass wir anderen Studenten helfen. Dies kann auf unterschiedlichste Weise passieren:
		\begin{itemize}
			\item Bei Fragen zu organisatorischen Themen können wir selber Antworten geben oder auf Personen verweisen, die euch da besser weiter helfen können.
			\item Falls ihr berechtigte Kritik an Vorlesungen und/oder Übungen nicht selbst äußern wollt oder sie ignoriert wird, sind wir dafür da, diese Kritik an die verantwortliche Person weiter zu tragen, wozu uns mehr Möglichkeiten zur Verfügung stehen.
			\item Wir organisieren die OE für Erstsemestler, damit ihr in der ersten Woche nicht vor lauter ungewohnten Sachen erschlagen werdet.
			\item Wir erstellen dieses Heft, damit ihr wichtige Themen nachschlagen könnt.
			\item Wir organisieren Sommerfest und Weihnachtsfeier!
		\end{itemize}
		
	\subsection{Du willst zu uns?}
		\glqq Wir grillen gerade und haben ein wenig zu trinken da. Willst du nicht zu uns stoßen?\grqq { }Bei vielen Fachschaftlern hat diese Masche schon funktioniert. Doch gerüchteweise gibt es unter den Studenten auch immer ein paar ''Rationalisten'' die ''Argumente'' brauchen. Und weil ''Wir sind cool!'' nicht immer ausreicht, sind hier ein paar Kommentare von und über uns:
		\begin{itemize}
			\item Wir sind cool!
			\item Social Engeneering lernen, wo besser als an der Uni?
			\item Die Gerüchteküche sind wir.
			\item Form die Uni nach eurem Bild.
			\item Lern die Professoren kennen.
			\item Mach die Uni zu deinem Zuhause and der Uni.
			\item Lern die Studenten kennen, die dir dein Zweistundenproblem in zwei Minuten lösen.
			\item Werde Student, der die Zweistundenprobleme Probleme in zwei Minuten löst.
			\item CP für Gremienarbeit.
			\item Das Direktorat fragt dich!
			\item Entscheide für wen Ausnahmen gemacht werden.
			\item Firmenpolitik begegnet dir überall. Lerne die Kunst an der Uni.
			\item Bekomm Schlüssel.
			\item Fachschaftsfeiern.
		\end{itemize}
