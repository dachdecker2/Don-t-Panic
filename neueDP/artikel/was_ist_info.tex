\spaltenanfang
Viele von euch haben eine Vorstellung davon was Informatik ist. Immerhin wollt ihr das ja studieren. Aber eure Vorstellung von Informatik wird sich wahrscheinlich in den n\"achsten Semstern deutlich \"andern.

Du verbringst viel Zeit vor deinem Computer und bist Experte im Umgang mit den neuesten Produkten aus dem Hause MS? Du kennst dich bestens aus im WWW und streifst Nachts leidenschaftlich von Chatroom zu Chatroom? Als begabter Hardware Tüftler bist du mit deiner selbst übertakteten Grafik Hardware ein Meister aller 3D Spiele und hast in World of Warcraft Rang und Namen? Außerdem ist Mathematik eh viel zu langweilig und überflüssig, braucht doch eh keiner, oder? Als eingefleischter Programmierer reicht es dir eh mal ab und zu den ein oder anderen Schnipsel PHP zu programmieren?

Willkommen in der Informatik, wahrscheinlich bist du hier ... falsch.


Spätestens jetzt fragen sich einige wahrscheinlich: Was soll der Blödsinn? Informatik hat doch mit Computern, EDV und Internet zu tun? Um jemanden zu zitieren (von dem ihr in eurem Studium noch einiges lernen werdet – nur Geduld): „Computer Science is no more about computers than astronomy is about telescopes.“ Folgendes werdet ihr also (im wesentlichen) nicht im Informatik Studium lernen:

\begin{itemize}
     \item EDV Grundlagen
     \item Wie kann ich Webseiten gestalten?
     \item Was ist die beste Hardware?
     \item Wo bekomme ich die her?
     \item Wie baue ich mir damit meinen PC zusammen?
     \item Und wie vernetze ich diesen?
     \item Und wenn ich schon dabei bin: Wie stelle ich die neueste Verschlüsselung für meinen 802.11x WLAN Router ein?
\end{itemize}

Das lernt ihr selbst in eurer Freizeit. Vieles davon wird im Studium sogar vorausgesetzt.

Informatik ist eine Wissenschaft. Sie besteht aus Mathematik, Algorithmen, Datenstrukturen, Technik (aha – also doch?) und ihrer Anwendung. Sie beschäftigt sich hauptsächlich mit Problemen und (sofern es sie gibt) deren Lösungen. Wir von der Fachschaft wollen euch natürlich nicht für dumm verkaufen. Wir haben für die diesjährige Dont Panic Interviews mit einigen Professoren von unserem Institut (die ihr mit Sicherheit auch bald kennen lernen werdet) geführt.


\spaltenende
