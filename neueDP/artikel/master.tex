\spaltenanfang
Nachdem du es also gepackt hast, einen Bachelor mit einer Note von nicht schlechter als 3.0 zu erwerben, hast du dich für den Masterstudiengang an der Goethe-Universität Frankfurt entschieden.


Da die Ordnung für Studienanfänger im Masterstudiengang aber etwas verwirrend sein kann, gibt es dazu hier einige Erläuterungen.


Zunächst sollte bemerkt werden, das du dich entscheiden musst, was für einen Master du letztendlich machen willst, denn der Master in Informatik ist unterteilt in vier Schwerpunkte:
\begin{noindEnumerate}
\item Master mit Anwendungsfach
\item Master mit vertieftem Anwendungsfach
\item Master in allgemeiner Informatik
\item Master mit Spezialisierung
\end{noindEnumerate}

Hierzu sei gesagt, dass wenn du im Bachelor kein Anwendungsfach studiert hattest, du keine andere Möglichkeit hast, als den Master mit Anwendungsfach zu machen.


Hattest du hingegen im Bachelorstudium schon ein Anwendungsfach belegt, so kannst du nun ein neues Anwendungsfach belegen, das alte Anwendungsfach vertiefen oder dich zwischen dem Master in allgemeiner Informatik und dem Master mit Spezialisierung entscheiden, falls du dich gegen ein Anwendungsfach entscheidest.


Kommst du von einer anderen Universität und hast deinen Bachelor nicht an der Universität Frankfurt abgelegt, so kann es sein, das du Auflagen erteilt bekommst oder die Wahl des Schwerpunktes eingeschränkt wird. Die Vorlesungen, die du als Auflagen erteilt bekommst und die nicht mehr als 30 CP umfassen, musst du dann innerhalb von 14 Monaten erfolgreich abschließen.


Insgesamt geht das Masterstudium über vier Semester und besteht aus 120 CP. Egal für welche dieser vier Formen des Masterstudiums du dich entscheidest, am Ende musst du bei allen im Abschlussmodul eine Masterarbeit schreiben, die 30 CP bringt und gewöhnlich über ein Semester geht (6 Monate).


Die Informatikmodule sind in drei Gebiete aufgeteilt: „Informatik der Systeme“, „Grundlagen der Informatik“ und „Angewandte Informatik“. Je nach gewähltem Schwerpunkt musst du aus jedem der Gebiete eine bestimmte Anzahl an CP erbringen. Unter den Informatikmodulen muss jedoch mindestens ein Seminar und ein Praktikum sein.


Zusätzlich sind in jedem Schwerpunkt Ergänzungsmodule zu belegen, die 3 bis 6~CP bringen. Bei diesem Ergänzungsmodul handelt es sich um Veranstaltungen wie „Tutoriumsleitung“, die Vorlesungen „IT-Projektmanagement“ und „Neue Medien und Gesellschaft“. Desweiteren gehört das Modul „Soft Skills“ noch zu den Ergänzungsmodulen. In diesem Modul können im entsprechenden Umfang Veranstaltungen gewählt werden, die wissenschaftliches Arbeiten, Präsentationstechniken, Themen aus dem Bereich „Informatik und Gesellschaft“ und Wissensethik und weitere Soft Skills vermitteln. Diese Veranstaltungen werden nicht unbedingt vom Institut für Informatik angeboten, sondern auch z.B. vom Didaktischen Zentrum der Uni. 
\spaltenende
