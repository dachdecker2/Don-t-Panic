In typischer Informatikermanier definieren wir am Anfang, was denn so wichtig sein k\"onnte. Also pr\"asentieren wir euch: Das deklarative W\"orterbuch der bisher eventuell unbenannten Dinge:
\spaltenanfang
\paragraph{AStA -} \glqq Allgemeiner Studierendenausschuss\grqq . Ist der gewählte Vorsitz des StuPa.
\paragraph{B.Sc. -}\glqq Bachelor of Science\grqq . Ist der erste Abschluss den ihr als Informatiker bekommen könnt.
\paragraph{Bachelorordnung -} Das Regelwerk, wie Ihr euren Studienabschluss bekommen könnt. Sollte von jedem Studenten in seinem Studium mindestens einmal gelesen werden!!! Sollten dabei Fragen aufkommen, könnt ihr euch an das Prüfungsamt wenden oder an die Fachschaft.
\paragraph{Bockenheim, Campus -} Die Großbaustelle, auf der du dich gerade befindest. Außerdem soll hier demnächst ein Kulturcampus entstehen und die Informatik angeblich bis 2017 auf den Riedbergcampus ziehen.
\paragraph{CP -} \glqq Creditpoints\grqq . Berechnungseinheiten des ECTS(European Credit Point Transfer System), die hochschulübergreifend angerechnet werden können sollten. Dabei gilt: 1 CP $\approx$ 30 Stunden Arbeitsaufwand.
\paragraph{Dekan -} Der Dekan ist ein Professor des Fachbereichs. Er wird vom Fachbereichsrat gewählt und „leitet die Geschäfte des Fachbereichs“ für einen Zeitraum von drei Jahren, d.h. er vertritt den Fachbereich nach außen, und er führt auch den Vorsitz im Fachbereichsrat. Er darf viele Entscheidungen auch selbst treffen, für die früher ein Beschluss des \textbf{FBR} notwendig war.
\paragraph{Dekanat -} siehe \textbf{Dekan}, \textbf{Prodekan} und \textbf{Studiendekan}. Daneben wird der Begriff „Dekanat“ für das Sekretariat eines Fachbereichs benutzt, dort wird ein Fachbereich verwaltet.
\paragraph{Direktorium(I-Rat) -} Das Direktorium ist eigentlich der Institutsrat. Anscheinend klang dies zu langweilig, weshalb das Direktorium auch I-Rat genannt wird.
\paragraph{Direktor, geschäftsführender -} Auch GD genannt, leitet das Direktorium. Außerdem repräsentiert er das Institut nach Außen.
\paragraph{Direktorat -}  Hierbei handelt es sich um das Pendant zum Dekanat, nur eben auf Institutsebene. Sämtliche Angelegenheiten der Informatik kann man zunächst im Direktorat der Informatik regeln. Hier könnt ihr auch die \textbf{Bachelorordnung} erhalten.
\paragraph{Evaluierung/Evaluation -} Studierende müssen per Gesetz in die Evaluierung, sprich die Reflexion und Bewertung von Lehrveranstaltungen einbezogen werden. Das ist im Großen und Ganzen auch der einzige Weg, den Studierende haben, um ihre Meinung zu einer Veranstaltung kund zu tun. Die Alternativen wären nicht ernstzunehmende Nörgelei oder die Fachschaft anzusprechen. Dennoch habt ihr dabei eine einigermaßen anonyme Möglichkeit zu Kritik und Lob und die Professoren bekommen tolle Statistiken. Wenn ihr also zur Mitte  oder zum Ende eines Semesters Evaluierungs-Fragebögen bekommt, nehmt sie ernst und beantwortet die Fragen. Selbst wenn ihr eine Veranstaltung abgebrochen habt (gerade dann!), ist es wichtig zu wissen, warum und was besser gemacht werden müsste.
\paragraph{Fachschaftsraum -} Ist der kleine Raum, der an die Studentlounge anschließt. Hier finden Donnerstags ab 16:00 Uhr die Fachschaftstreffen statt.
\paragraph{Fachschaftstreffen -} Das regelmäßige Treffen der Studierendenvertretung in der Informatik. Es findet Donnerstags ab 16:00 Uhr im Fachschaftsraum statt, wobei am ersten Donnerstag im Monat immer die wichtigen Dinge besprochen werden.
\paragraph{FBR -} \glqq Fachbereichsrat\grqq. Der FBR ist ein Gremium welches über die zentralen Belange des Fachbereichs 12 entscheidet. Die studentischen Vertreter werden dabei direkt Gewählt.
\paragraph{Fischerräume -} Die Rechnerräume der Informatik, welche sich hinter dem Magnushörsaal befinden und nur von außen  betretbar sind. Einloggen könnt ihr euch mit eurem RBI-Account.
\paragraph{FS/FS-Inf -} Die Fachschaft bzw. die Fachschaft Informatik. Der Begriff „Fachschaft“ wird aber auch in einem wesentlich engeren Sinn für die wenigen Leutchen benutzt, die sich ganz aktiv um die Belange der Studierenden kümmern. Siehe auch den Artikel „Fachschaftsarbeit“ (Seite \pageref{fachschaftsarbeit}).
\paragraph{HiWis -} Hilfswissenschaftler. Das sind eure \textbf{Tutoren} in den Übungen, die studentischen Mitarbeiter in der Bibliothek oder in der RBI. Diesen günstigen Arbeiterschwärmen kann man sich meist schon nach ein paar Semestern anschließen.
\paragraph{Hörsäle -} Die Hörsäle H I - H IV und H 1 - H 16 findet ihr im Hörsaalgebäude.
\paragraph{HRZ-Account -} Jeder Studierende erhält vom \textbf{\underline{H}ochschul\underline{r}echen\underline{z}entrum} einen Account. Durch diesen lassen sich verschiedene Dienste der Uni Frankfurt nutzen. Leider sind die Accountnamen inzwischen pseudonym und schlecht zu merken. Der wichtigste Nutzungsgrund liegt darin, dass ihr über den Zugang die persönlichen Daten Eures Studiums verwalten und euch zu Prüfungen anmelden könnt, siehe dazu auch \textbf{QIS-LSF}. Ansonsten braucht ihr ihn für WLAN (an fast allen Unis Deutschlands), eventuell E-Mail, FTP und andere Dinge. Die Zugangsdaten erhaltet ihr zusammen mit dem \textbf{Studentenausweis}. Nicht zu verwechseln mit dem \textbf{RBI-Account}.
\paragraph{Institutsrat -} Der Institutsrat ist ein Gremium , das die Belange des Institutes Informatik behandelt und teilweise, wo nicht der Fachbereichsrat gefragt ist, entscheidet.
\paragraph{Kommilitonen -} Sind die Leute links und rechts neben dir.
\paragraph{Lernzentrum -}  Befindet sich im Erdgeschoss des Informatikgebäudes, gleich links wenn ihr zum Haupteingang hineinkommt. Hier könnt ihr versuchen zu lernen, sollte ihr euch wieder erwarten doch inmitten von Diskussionen konzentrieren können. Im Lernzentrum sind die Skripte der momentan laufenden Basismodule vorhanden. Außerdem habt ihr hier die Möglichkeit einen Mitarbeiter der Uni um Hilfe bei euren Aufgaben zu bitten. Es gibt sogar Brettspiele zum Ausleihen. An das Lernzentrum sind die ''Student-Lounge'' genannten Räume der Fachschaft angeschlossen.
\paragraph{Magnus-Hörsaal -} Ist der einzige Hörsaal, den die Informatik im Gebäude hat.
\paragraph{Modul -} Ein Modul ist eine Lehreinheit, das die fachlich sinnvoll aus ein bis mehreren Lehrveranstaltungen zusammengesetzt ist. Ein Modul wird innerhalb eines Semesters oder auch über 2 Semester veranstaltet.
\paragraph{M.Sc. -} \glqq Master of Science\grqq . Großer Bruder vom Bachelor.
\paragraph{Munchkin -} Ein sehr beliebtes Kartenspiel im Fachbereich. 
\paragraph{Prodekan -} Sowas wie ein stellvertretender Dekan. Der Dekan und der Prodekan können auch eine Art „Aufgabenteilung“ unter sich vereinbaren.
\paragraph{Prüfungsamt -} Im Prüfungsamt meldet man sich für Klausren, mündliche Prüfungen und Studenleistungen aller Art an. Zwar geschehen inzwischen die meisten Anmeldungen über das \textbf{QIS-LFS}, aber das ist manchmal unzuverlässig.
\paragraph{Prüfungsausschuss -} Der Prüfungsausschuss ist ein Ausschuss der sich mit Prüfungen befasst (Sach bloß!). Hier siten Professoren und Studenten, die Entscheidungen treffen, die das \textbf{Prüfungsamt} nicht treffe darf.  Vor allem kann man hier Außnhameregelungen formlos beantragen. Dazu schreibt man einfach einen Brief an den Ausschuss, in dem man kurz sagt was man will, und wieso. Den kann man dann im \textbf{Prüfungsamt} einwerfen.
\paragraph{Prüfungsprotokoll-Datenbanken -} Etwas, bei dem dir ruhig mulmig zu Mute sein darf, ist eine mündliche Prüfung. Doch dem mulmigen Gefühl kann abgeholfen werden: Schau dir doch einfach mal an, was den anderen so passiert ist, als sie in ihrer Prüfung waren. Dafür gibt es eine eigene Datenbank, die ihr unter \url{http://myexam.gdv.cs.uni-frankfurt.de/} finden könnt. Mitmachen ist dabei Pflicht, denn ohne Geben gibt es bei dieser Idee auch kein Nehmen!
\paragraph{QIS-LSF -} Das elektronische Informationssystem der Uni. Hier könnt ihr euch für Prüfungen anmelden, eure Kontaktdaten ändern oder euch den Vorlesungskatalog anschauen. Für das QIS-LSF braucht ihr euren HRZ-Account
\paragraph{Rekursion -} Rekursion siehe Rekursion!
\paragraph{RBI -} \glqq Rechner-Betriebsgruppe Informatik\grqq . Die RBI kümmert sich um alles, was mit Elektronik in der Informatik zu tun hat. Über euren RBI-Account, den ihr in der RBI beantragen könnt, habt ihr pro Semester 500 Druckseiten frei und könnt noch weitere Angebote nutzen.
\paragraph{Skript -} Entweder eine frühere Mitschrift eines fleißigen Studenten (eher selten anzutreffen) oder die Kopien der vom Professor benutzten Folien, gelegentlich auch ein extra hierfür entworfener Text. Ist entweder in der entsprechenden Fachbereichs-Bibliothek, auf der Internet-Präsenz des jeweiligen Dozenten oder gar nicht zu finden. Dabei sollte man jedoch zwei Dinge beachten: 1. Das Lesen des Skriptes ersetzt nur sehr selten die Teilnahme an der Veranstaltung. 2. Wer ein komplettes Skript auf dem Drucker im RBI ausdruckt, läuft Gefahr, von vor dem Drucker wartenden Kommilitonen gesteinigt zu werden, oder zumindest als DAU (Dümmster anzunehmender User) des Monats nominiert zu werden; vgl. \textbf{RBI}.
\paragraph{Statusgruppen -} Die Statusgruppen der Universität sind Professoren, Studierende, Wissenschaftliche Mitarbeiter (WiMi) und die administrativ-technischen Mitarbeiter (SoMis).
\paragraph{Studentenausweis(Goethe-Card) -} Diese kleine Plastikkarte erfüllt gleich mehrere Aufgaben: sie dient euch als Studierendenausweis, Bibliotheksausweis und als Semesterticket. Der Zeitraum, für den die Goethe-Card gültig ist, wird mit Spezialtinte auf die Karte aufgedruckt. Nach der Rückmeldung könnt Ihr den Aufdruck erneuern, indem ihr einen der dafür vorgesehenen Automaten verwendet (z.B. im Gebäude \textit{Neue Mensa}). Durch den enthaltenen Chip werden Informationen zu den von Euch ausgeliehenen Büchern gespeichert, außerdem könnt ihr sie als Schlüssel für Schließfächer verwenden. Ihr könnt die Goethe-Card mit Geld ``aufladen'', und mit diesem Guthaben die von der Uni aufgestellten Kopierer nutzen, oder in der Mensa bezahlen.
\paragraph{Studentlounge -} Die \textbf{Fachschaftsräume} hinter dem Lernzentrum, die normalerweise offen sind und von allen Studierenden als eine Erweiterung vom \textbf{Lernzentrum} oder auch als ein Platz, wo man sich auf Sofas entspannen könnte, betrachtet werden kann. Auch hier solltet ihr allgemeine Regeln des sozialen Zusammenlebens beachten. Also haltet den Raum sauber, entsorgt euren Müll und haltet euch an die Hausordnung.
\paragraph{Studiendekan -} Auch ein Mitglied der „Drei Dekane“. Er ist im Besonderen zuständig für Fragen von Lehre und Studium.
\paragraph{Studien-Service-Center -} Befindet sich im PEG-Gebäude auf dem Westend. Hier könnt ihr alle Arten von Anträgen(Teilzeit,etc.) stellen.
\paragraph{StuPa -} Das StuPa ist, wie der Name „Studentenparlament“ ja auch sagt, das Parlament der Studierenden. Und wie es sich für jedes ordentliche Parlament gehört, wird auch im StuPa Politik gemacht: Deshalb hält man die StuPa-Sitzungen auch kaum aus. Studentische Interessen stehen leider eher selten im Vordergrund. Welche politischen Hochschulgruppen in das StuPa reingewählt werden, könnt ihr versuchen in Wahlen zu beeinflussen.
\paragraph{SWS -} Abkürzung für Semesterwochenstunde, sprich die Zeit, die eine Veranstaltung im Verlauf des Semesters wöchentlich dauert. Die private Vor- bzw. Nacharbeit dauert normalerweise noch das doppelte dazu. Also: 1 SWS = 3 Stunden fürs Studium wochentlich.
\paragraph{Tutor -} Als Tutor endet man aus drei verschiedenen Gründen. Ein Grund wäre, dass ihr CP für euer Ergänzungsmodul braucht und euch dadurch entschieden habt Tutor zu werden. Der zweite Grund wäre, dass ihr Geld braucht und für einen Studentlohn von 8,50 Euro arbeiten wollt. Oder aber ihr wollt euer Wissen weiter geben bzw. euer Wissen auffrischen. Eine Mischung aus den drei Gründen tritt auch gelegentlich auf.
\paragraph{Umzug -} Eine sehr beliebte Sage innerhalb des Fachbereichs 12. Von Generation zu Generation wird weiter verbreitet, dass wir zum Riedberg umziehen werden. Erste historische Erwähnung dieser Sage wurde in den Protokollen der 4. Ratssitzung des Direktorium für Informatik im Jahre 1979 gefunden.
\paragraph{Variable, metasyntaktisch -} Platzhalter und genereller Ausdruck der Faulheit von Informatikern. Erlaubt einem sich unpr\"azise auszudr\"ucken. Beispielsweise seien hier foo, bar, foobar, xyzzy, Ding, Dinge und Zeuch genannt. Zeuch nimmt dabei eine Sonderstellung ein, da dadurch gerne komplette Prozeduren, Programmbl\"ocke und Krempel von Dingen die Zeuch machen beschrieben wird.
\paragraph{Wahlen -} Irgendwann werdet ihr ein Brief nach Hause bekommen, womit ihr z.B. den FBR oder das StuPa wählen könnt. Wer oder was gew\"ahlt wird ist dabei vielen Studenten leidenschaftlich unklar.
\paragraph{Wert, metasyntaktisch} Wenn einem nichts einf\"allt nimmt man die. Metasyntaktische Werte haben oft eine besondere Bedeutung, die aber mit der Zeit in Vergessenheit geraten ist. Beispiele sind 7, 42, 37, 0815, 1337, 31337, \texttt{0xdeadbeef}, \texttt{0xcafebabe} und f\"ur Stringkonstanten ''foo''. Werden fast immer \textit{metasyntaktischen Variablen} zugeordnet.
\paragraph{Westend, Campus -} Der Campus der Geisteswissenschaften. Zu erreichen per Bus mit Linie 36 oder 75, per U-Bahn 1,2,3 oder 8 (Holzhausenstraße) oder falls man Zeit hat auch zu Fuß durch den Palmengarten.
\paragraph{WiMis -} Wissenschaftliche Mitarbeiter. Sie sind an einer Professur angestellt, um dort zu promovieren, also um irgendwann einmal einen Doktortitel zu erhalten. Bis dahin forschen sie fleissig und unterstützen den Professor bei Lehrveranstaltungen. Gerade in Übungen werdet ihr ab und zu mit einem WiMi zu tun haben.
\spaltenende
