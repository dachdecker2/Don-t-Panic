\spaltenanfang
Seit dem Wintersemester 1997/98 kann man in Frankfurt Informatik als Lehramtsfach im L3-Studiengang (Gymnasiales Lehramt) studieren – mit sehr guten Berufsaussichten, denn viele entschließen sich nicht dazu und viele hören schon bald wieder auf\ldots Warum eigentlich? So ein Info-Lehramtsstudium ist doch ganz einfach. Zunächst einmal stellen wir die für das Lehramtstudium relevanten Ämter vor:


LSA - \textit{Das Landesschulamt}
Das Landesschulamt (LSA) (ehemals: AfL / Amt für Lehrerbildung)
Das LSA ist dem Hessischen Kultusministerium direkt untergeordnet und ist verantwortlich für die Ausbildung von Lehrkräften aller Fachrichtungen und Schulformen in ganz Hessen.
Du wirst im Regelfall mit dem LSA direkt nur zweimal zu tun haben: Bei deinem „Orientierungspraktikum“ und bei deinem „Betriebspraktikum“, dazu später mehr.
Kontakt mit dem LSA kannst du am besten über deren Homepage (http://lehrerbildung.lsa.hessen.de, dann links auf „Lehramtsstudierende“ ) und per E-Mail an den zuständigen Sachbearbeiter aufnehmen.


SBS-Büro \textit{Büro für Schulpraktische Studien}

Das SPS-Büro der ABL (Akademie für Bildungsforschung und Lehrerbildung) übernimmt innerhalb der Goethe-Universität die Koordination der sogenannten Schulpraktischen Studien (SPS), auch hierzu später mehr. Du findest das SPS-Büro momentan im Juridicum auf dem Campus Bockenheim im 10. OG.
Du meldest dich beim SPS-Büro für die Schulpraktischen Studien an und im Gegensatz zu allen anderen Ämtern an der Universität muss dem SPS-Büro eine Änderung deines Studiums explizit schriftlich mitgeteilt werden (z.B.: bei einem Fachwechsel). Die Anmeldefristen für die SPS findest du auf der Homepage des SPS-Büros (www.abl.uni-frankfurt.de/40729270/Schulpraktische-Studien ). Diese Fristen sind absolut verbindlich und es gibt keinen Spielraum – also hier besonders genau sein!


ZPL – \textit{Zentrales Prüfungsamt für Lehramtsstudiengänge}
Das ZPL befindet sich momentan am Campus Bockenheim im Juridicum. Es ist dafür verantwortlich die von dir abgelegten Prüfungen (abgeschlossene Module) zu registrieren und die Zwischenprüfung, sowie die Meldung zur Ersten Staatsprüfung zu verwalten.
Zu Beginn deines Studiums meldest du dich zur Zwischenprüfung beim ZPL an (Informationen bekommst du entweder auf der Homepage des ZPL, oder bei einer Einführungsveranstaltung). Du legst die Zwischenprüfung automatisch ab, wenn du in vollständig abgeschlossenen Modulen insgesamt 90 CP beim ZPL in Form von kopierten (!) Modulscheinen eingereicht hast.
Die Zwischenprüfung kannst du ablegen, sobald du die benötigten CP gesammelt hast (hierbei gibt es genauere Auflagen, die du der Studienordnung entnehmen kannst), sie sollte allerdings spätestens zwei Semester vor der Meldung zur Ersten Staatsprüfung eingereicht werden, damit eventuelle formale Probleme behoben werden können.\\

Neben diesen Ämtern möchten wir dir einige Informationen zu einigen wichtigen Begriffen und Punkten im Lehramtsstudium geben.

\paragraph{Die Schulpraktischen Studien (SPS)}
Die SPS sind der praktische Teil der universitären Lehramtsausbildung. Du wirst in deinem Studium zwei SPS-Veranstaltungen absolvieren, eine bildungswissenschaftliche und eine fachwissenschaftliche. Die SPS setzen sich in der Regel aus einem Vorbereitungsseminar in einem Semester, einem 5- wöchigen Schulpraktikum in den Semesterferien und einer Nachbereitungsveranstaltung im folgenden Semester zusammen.

\paragraph{Modulscheine}
Im Gegensatz zu den meisten anderen Studiengängen ist das Lehramtsstudium noch nicht vollständig digitalisiert. Du kannst dir auf der Seite des ZPL sog. „Modulscheine“ ausdrucken, in welche die Dozenten die Prüfungsnoten eintragen müssen.
In der Regel ist es so, dass zu einem Modul mehrere Veranstaltungen gehören, in diesem Fall ist es empfehlenswert in der zweiten Veranstaltung nur eine Kopie des bereits zum Teil ausgefüllten Scheins abzugeben, denn ein Verlust ist für dich mit erheblicher Arbeit verbunden.

\paragraph{Die fachspezifischen Anhänge und die Studienordnung}
Die Studienordnung für Lehramtsstudiengänge setzt sich aus der
"Studienordnung" an sich und den "fachspezifischen Anhängen" zusammen. Die Studienordnung regelt generelle Formalia, wie zum Beispiel den Umfang der Studium, die benötigeten CP und die Abläufe der Ersten Staatsprüfung und der Zwischenprüfung.
Die fachspezifischen Anhänge hingegen beschränken sich auf ein Fach (z.B.: L3 Informatik) und beinhalten eine Übersicht über die zu belegenden Module.
Sowohl die Studienordnung, als auch die fachspezifischen Anhänge finden sich auf der Homepage des ZPL und sind leicht über eine Suchmaschine unter dem Stichwort "fachspezifische Anhänge Uni Frankfurt" zu finden.
\textbf{Zu Beginn des Studiums empfiehlt sich auf jeden Fall eine Lektüre der Studienordnung sowie der fachspezifischen Anhänge, zumindest sollten sie quergelesen werden!}
Innerhalb der spezifischen Anhänge findest du einen Studienverlaufsplan, also einen exemplarischen Plan, in welcher Reihenfolge und in welchem Semester du welche Veranstaltungen belegen kannst. Die Vorgaben sind allerdings nicht verpflichtend, du hast dadurch allerdings den Vorteil, dass keine Überschneidungen entstehen (dürften).

\paragraph{Orientierungspraktikum (OP)}
Das OP muss durch das LSA bestätigt worden sein, bevor du dich für die ersten SPS anmelden kannst. Es muss in einer pädagogischen Institution (oft: Kindergarten, Jugendbetreuung) stattfinden und 5 Wochen (120 Stunden) dauern. Auf der Homepage des LSA befindet sich ein Vordruck für einen Praktikumsbericht, der ausgefüllt werden muss – auf der letzten Seite des Berichts ist ein Formular von dir und vom Betrieb auszufüllen, zudem muss vom Betrieb eine (unformale) Bestätigung ausgestellt werden.
Für das OP können Tätigkeiten aus FSJ oder ähnlichem anerkannt werden, in diesem Fall muss nur eine Bescheinigung des Betriebs beiliegen sowie die letzte Seite ausgefüllt werden. Der Bericht entfällt in diesem Fall.

\paragraph{Betriebspraktikum (BP)}
Das Betriebspraktikum umfasst 8 Wochen bei "branchenüblicher Arbeitszeit" (in Ordnung sind auch zwei mal 4 Wochen, auch in unterschiedlichen Betrieben). Es muss in einem Betrieb stattfinden, der knapp gesagt, nichts mit (pädagogisch-) sozialen Tätigkeiten zu tun hat. Auch hierzu findest du auf der Homepage des LSA einen Vordruck für einen Praktikumsbericht. Es wird eine Bestätigung des Betriebs benötigt.
Als Betriebspraktikum können Nebenjobs in nicht-pädagogischen Betrieben anerkannt werden, die über einen längeren Zeitpunkt gemacht wurden. In diesem Fall benötigst du eine Bestätigung des Betriebs und musst den Bericht dennoch schreiben!
Das Betriebspraktikum muss vor der Meldung zur Ersten Staatsprüfung eingereicht werden.
\spaltenende
