\documentclass[12pt,a4paper]{scrartcl}
\usepackage[utf8x]{inputenc}
\usepackage{ucs}
\usepackage[german]{babel}
\usepackage{textcomp}
\usepackage{graphicx}
\usepackage{wrapfig} %Grafiken von Text umfliessen lassen
\usepackage{multicol} %zweispaltiger Text
\usepackage{multirow} %mehrere Tabellenzeilen zu einer zusammenfassen
\usepackage{rotating} %Rotation von Schrift
\usepackage{wallpaper} %Bilder auch ans Hintergrundbilder einbinden
\usepackage{hyperref} %Um in der PDF-Version klickbare Hyperlinks zu haben
\usepackage{color} % für Farben im allgemeinen
\usepackage{colortbl} %farbige Tabellen
\usepackage[final]{pdfpages}
\usepackage[a4paper,left=15mm,right=15mm, top=12mm, bottom=26mm]{geometry} %Seitenränder
\usepackage{paralist}

%%%%%%%%%%%%%%%%%%%%%%%%%%%%%%%%%%%%%%%%%%%%%%%%%%%%
% Variablen
\newcommand{\emailfachschaft}{\raisebox{-0.5mm}{\includegraphics{bilder/email_fachschaft}}~}
%\setcounter{tocdepth}{1} % Tiefe des Inhaltsverzeichnisses auf 1(=section) setzen

%%%%%%%%%%%%%%%%%%%%%%%%%%%%%%%%%%%%%%%%%%%%%%%%%%%%
% eigene Kommandos
\newcommand{\spaltenanfang}{\begin{multicols}{2}}
\newcommand{\spaltenende}{\end{multicols}}

\newcounter{mycounter}
\newenvironment{noindEnumerate}
{\begin{list}{\arabic{mycounter}.~~}{\usecounter{mycounter} \labelsep=0pt \labelwidth=0pt \leftmargin=0pt \itemindent=0pt \itemsep=-5pt}}
	{\end{list}}
\newenvironment{noindItemize}
{\begin{list}{\labelitemi}{\usecounter{mycounter} \labelsep=5pt \labelwidth=0pt \leftmargin=5pt \itemindent=0pt \itemsep=-5pt}}
	{\end{list}}


\begin{document}
%%%%%%%%%%%%%%%%%%%%%%%%%%%%%%%%%%%%%%%%%%%%%%%%%%%%
% Titelseite
\begin{titlepage}

\thispagestyle{empty}
\title{\Huge{Don't Panic! 42}}
\author{Magazin für die OE SS 2015}
\date{Herausgegeben von der Fachschaft Informatik}


%\begin{document}

\ThisCenterWallPaper{1.0}{bilder/bender_titelseite4}
\maketitle
\newpage
\end{titlepage}


%%%%%%%%%%%%%%%%%%%%%%%%%%%%%%%%%%%%%%%%%%%%%%%%%%%%
% Inhaltsverzeichnis
\tableofcontents
\newpage

%%%%%%%%%%%%%%%%%%%%%%%%%%%%%%%%%%%%%%%%%%%%%%%%%%%%
% Willkommen
\section*{Ei Guude!}
Da seid ihr also. Die Neuen. Und ihr seid hier aus einem Grund. Einem spezifischen Grund. Ihr wollt was lernen.
Und das werdet ihr. An der Uni lernt ihr, so wie wir das gelernt haben,
\"uber jegliche Form der Ahnungslosigkeit hinwegzut\"auschen. Und wir helfen euch dabei!\\
Daher pr\"asentiert die Fachschaft Informatik euch voller Stolz und Selbstverherrlichung die \emph{Don't Panic! 42}.

Und damit fangen wir auch mit der ersten Lektion in Sachen Hochschulkultur an: Selbstbeweihr\"aucherung.
Damit ihr ganz genau wisst, auf wen ihr die Schuld schieben k\"onnt, sind hier alle, die an der Don't Panic! mitgearbeitet haben:
\begin{itemize}
	\item Alexandra Herrmann - Kein Typ
	\item Jonathan Cyriax Brast - Ein Typ
	\item Joshua Sole - Noch ein Typ
	\item Johannes Göpel - Datentyp
	\item Kyle Rinfreschi - Auch ein Typ
	\item Linda Homeier - Typin
	\item Manuel Dittrich - UnTypisch  
	\item Shayan Naderi - orientalischer Typ
	\item Sandra Stelzenmüller - Typendes
	\item Tobias Rothenberger - RBI Typ
	\item \LaTeX - stark typisiert
	\item Und aus historischen Gr\"unden all unsere Vorg\"anger:
	Johannes Sch\"opp, Simon Pruy, Pavel Safre, Grzegorz Lato, Markus Palcer, Tim F\"oller, Sabrina Brandt, Markus Strobel, Michael Bals, Christoph Burschka, Sebastian Behr, Max Hahn-Klimroth und Igor Geier.
\end{itemize}
Außerdem möchten wir Randall Munroe von xkcd (\url{http://xkcd.com}) für die tollen Comics und f\"ur deren Lizenzierung unter Creative Commons (CC) danken.\\

\begin{flushright}Die Fachschaft\\\tiny kollektives Schwarmbewusstsein der Informatik\end{flushright}


%%%%%%%%%%%%%%%%%%%%%%%%%%%%%%%%%%%%%%%%%%%%%%%%%%%%
% Impressum
\section{Impressum}
%\vspace{5mm}
\section*{\begin{turn}{10} \textbf{WILLKOMMEN!} \end{turn} }
%\vspace{5mm}
Ei Guude!\\

Da seit ihr also. Die Neuen. Und ihr seit hier aus einem Grund. Einem spezifischen Grund. Ihr wollt was lernen.
Und das werdet ihr. An der Uni lernt ihr, so wie wir das gelernt haben,
\"uber jegliche Form der Ahnungslosigkeit hinwegzut\"auschen. Und wir helfen euch dabei!\\

Daher pr\"asentiert die Fachschaft Informatik euch voller Stolz und Selbstverherrlichung die \emph{Don't Panic! 42}.

Und damit fangen wir auch mit der ersten Lektion in Sachen Hochschulkultur an: Selbstbeweihr\"aucherung.
Damit ihr ganz genau wisst, auf wen ihr die Schuld schieben k\"onnt sind hier alle, die an der Don't Panic! mitgearbeitet haben:
\begin{itemize}
	\item Alexandra Herrmann - Kein Typ
	\item Jonathan Cyriax Brast - So ein Typ
	\item Joshua Sole - Noch ein Typ
	\item Linda Homeier - Typin
	\item Simon Pruy - Auch ein Typ
	\item Wiebke Belitz-Hellwich - Auch kein Typ
	\item Johannes Schöpp - Komischer Typ
	\item \LaTeX - stark typisiert
	\item Und aus historischen Gr\"unden all unsere Vorg\"anger:
	Pavel Safre, Grzegorz Lato, Markus Palcer, Tim F\"oller, Sabrina Brandt, Markus Strobel, Michael Bals, Christoph Burschka, Sebastian Behr und Igor Geier.
\end{itemize}


Außerdem möchten wir xkcd (\url{http://xkcd.com}) für die tollen Comics danken.

Es war viel Arbeit. Wir hoffen ihr wisst das zu schätzen.
Ansonsten wünschen wir euch jetzt aber einen guten Einstieg in euer Studium!

\begin{flushright}Die Fachschaft\end{flushright}

\section{Impressum}
\spaltenanfang
\textbf{Studentische Vertretung der Lehreinheit Informatik der Johann Wolfgang Goethe-Universität Frankfurt am Main}\\
Robert-Mayer-Straße 11 - 15\\
D-60325 Frankfurt am Main\\
\url{www.fsinf-frankfurt.de}\\
\url{www.fsinf-forum.de}\\
\emailfachschaft\\
%~\newline
Don‘t Panic! 42\\
OE SS 2015\\
Oft überarbeitete und erweiterte Auflage\\
Version der Ausgabe: 2.0\\
April 2015\\
Erscheinungsweise: jedes Semester\\
Auflage: 250\\
Druck und Bindung: HRZ Druckzentrum

\spaltenende

\newpage

%%%%%%%%%%%%%%%%%%%%%%%%%%%%%%%%%%%%%%%%%%%%%%%%%%%%
% Hallo
\section{Hallo erstmal...}
\spaltenanfang
Herzlich willkommen bei uns am Fachbereich! Was du gerade in deinen Händen
hältst, ist die Zeitschrift der diessemestrigen Orientierungsveranstaltung, die
wir „Don’t Panic“ getauft haben. \textbf{Don’t Panic!} - das soll auch als
Motto über der ganzen Veranstaltung stehen. Diesmal wieder etwas \textsl{mehr} editiert, 
nachdem wir das letzte Mal nur leicht editiert haben, da wir, nachdem wir das 
vorletzte Mal alles in einer Nacht geschrieben hatten, so viele Kommafehler 
eingebaut hatten, dass die \textbf{Don't Panic!} eine Seite l"anger wurde
\footnote{Ihr denkt, dieser Satz ist lang? Wartet mal auf das nächste Semester...}. 


Naja, jetzt ist erst mal wieder Semesteranfang, und es haben sich lauter Menschen
entschlossen, in Frankfurt einen Studiengang der Informatik aufzunehmen. Einige
von ihnen haben schon vorher etwas anderes studiert oder kommen von einer anderen
Hochschule. Die kennen sich meist schon recht gut im Uni-Dschungel aus, und 
auch die ganzen Begriffe, Abkürzungen und Redewendungen sind für sie meist keine
böhmischen Dörfer mehr. Aber für einen beachtlichen Teil der Erstsemester ist
erfahrungsgemäß so ziemlich alles neu. Und deswegen werden wir uns bemühen,
euch während dieser Orientierungsveranstaltung so ziemlich alles zu erklären.
Ihr werdet hoffentlich schnell merken, dass das alles halb so wild ist und kein
Grund zur Panik besteht, also \textbf{Don’t Panic!} Bei dieser
Orientierungsveranstaltung haben wir uns im Wesentlichen zwei Ziele gesetzt:


Wir wollen euch mit allen notwendigen Informationen versorgen, damit ihr an
eurem ersten Tag in der Uni wenigstens so ungefähr wisst, wo die wichtigsten
Einrichtungen sind, welche Veranstaltungen so laufen und welche davon für euch
Sinn machen. Wir wollen euch ein paar Ratschläge und Tipps mit auf den Weg
geben. Und nicht zuletzt können wir euch von einer großen Sammlung von Fehlern
berichten, die wir gemacht haben und die \textsl{ihr} ja nicht unbedingt noch
mal machen müsst.


Das Unileben und das Informatikstudium bringen viele Begriffe mit sich, die
euch vielleicht unbekannt sind. Vielleicht möchtet ihr einige Stichworte auch
noch einmal in kompakter Form nachlesen. Aus diesem Grund haben wir euch ein
Glossar der wichtigsten Begriffe zusammengestellt.


Aber wir möchten auch, dass ihr euch heute gegenseitig ein
bisschen kennenlernt, damit euch am nächsten Montag wenigstens schon ein paar
Gesichter bekannt vorkommen, wenn der Stress losgeht. Ganz generell empfehlen
wir euch, sich in kleinen Gruppen zusammenzutun. In einer Gruppe weiß eigentlich
immer jemand, was wo aushängt, bis wann man sich irgendwo eingetragen haben
muss und vieles mehr. Auch das eigentliche Studieren, das Lernen und das Lösen
von Aufgaben ist in einer Gruppe wesentlich erfolgversprechender und mit
Sicherheit auch angenehmer.\\
\spaltenende

\newpage

%%%%%%%%%%%%%%%%%%%%%%%%%%%%%%%%%%%%%%%%%%%%%%%%%%%%
% Glossar
\label{glossar}
\section{Das Wörterbuch der wahrscheinlich unbekannten Dinge}
In typischer Informatikermanier definieren wir am Anfang, was denn so wichtig sein k\"onnte. Also pr\"asentieren wir euch: Das deklarative W\"orterbuch der bisher eventuell unbenannten Dinge:
\spaltenanfang
\paragraph{AStA -} \glqq Allgemeiner Studierendenausschuss\grqq . Ist der gewählte Vorsitz des StuPa.
\paragraph{B.Sc. -}\glqq Bachelor of Science\grqq . Ist der erste Abschluss den ihr als Informatiker bekommen könnt.
\paragraph{Bachelorordnung -} Das Regelwerk, wie Ihr euren Studienabschluss bekommen könnt. Sollte von jedem Studenten in seinem Studium mindestens einmal gelesen werden!!! Sollten dabei Fragen aufkommen, könnt ihr euch an das Prüfungsamt wenden oder an die Fachschaft.
\paragraph{Bockenheim, Campus -} Die Großbaustelle, auf der du dich gerade befindest. Außerdem soll hier demnächst ein Kulturcampus entstehen und die Informatik angeblich bis 2017 auf den Riedbergcampus ziehen.
\paragraph{CP -} \glqq Creditpoints\grqq . Berechnungseinheiten des ECTS(European Credit Point Transfer System), die hochschulübergreifend angerechnet werden können sollten. Dabei gilt: 1 CP $\approx$ 30 Stunden Arbeitsaufwand.
\paragraph{Dekan -} Der Dekan ist ein Professor des Fachbereichs. Er wird vom Fachbereichsrat gewählt und „leitet die Geschäfte des Fachbereichs“ für einen Zeitraum von drei Jahren, d.h. er vertritt den Fachbereich nach außen, und er führt auch den Vorsitz im Fachbereichsrat. Er darf viele Entscheidungen auch selbst treffen, für die früher ein Beschluss des \textbf{FBR} notwendig war.
\paragraph{Dekanat -} siehe \textbf{Dekan}, \textbf{Prodekan} und \textbf{Studiendekan}. Daneben wird der Begriff „Dekanat“ für das Sekretariat eines Fachbereichs benutzt, dort wird ein Fachbereich verwaltet.
\paragraph{Direktorium(I-Rat) -} Das Direktorium ist eigentlich der Institutsrat. Anscheinend klang dies zu langweilig, weshalb das Direktorium auch I-Rat genannt wird.
\paragraph{Direktor, geschäftsführender -} Auch GD genannt, leitet das Direktorium. Außerdem repräsentiert er das Institut nach Außen.
\paragraph{Direktorat -}  Hierbei handelt es sich um das Pendant zum Dekanat, nur eben auf Institutsebene. Sämtliche Angelegenheiten der Informatik kann man zunächst im Direktorat der Informatik regeln. Hier könnt ihr auch die \textbf{Bachelorordnung} erhalten.
\paragraph{Evaluierung/Evaluation -} Studierende müssen per Gesetz in die Evaluierung, sprich die Reflexion und Bewertung von Lehrveranstaltungen einbezogen werden. Das ist im Großen und Ganzen auch der einzige Weg, den Studierende haben, um ihre Meinung zu einer Veranstaltung kund zu tun. Die Alternativen wären nicht ernstzunehmende Nörgelei oder die Fachschaft anzusprechen. Dennoch habt ihr dabei eine einigermaßen anonyme Möglichkeit zu Kritik und Lob und die Professoren bekommen tolle Statistiken. Wenn ihr also zur Mitte  oder zum Ende eines Semesters Evaluierungs-Fragebögen bekommt, nehmt sie ernst und beantwortet die Fragen. Selbst wenn ihr eine Veranstaltung abgebrochen habt (gerade dann!), ist es wichtig zu wissen, warum und was besser gemacht werden müsste.
\paragraph{Fachschaftsraum -} Ist der kleine Raum, der an die Studentlounge anschließt. Hier finden Donnerstags ab 16:00 Uhr die Fachschaftstreffen statt.
\paragraph{Fachschaftstreffen -} Das regelmäßige Treffen der Studierendenvertretung in der Informatik. Es findet Donnerstags ab 16:00 Uhr im Fachschaftsraum statt, wobei am ersten Donnerstag im Monat immer die wichtigen Dinge besprochen werden.
\paragraph{FBR -} \glqq Fachbereichsrat\grqq. Der FBR ist ein Gremium welches über die zentralen Belange des Fachbereichs 12 entscheidet. Die studentischen Vertreter werden dabei direkt Gewählt.
\paragraph{Fischerräume -} Die Rechnerräume der Informatik, welche sich hinter dem Magnushörsaal befinden und nur von außen  betretbar sind. Einloggen könnt ihr euch mit eurem RBI-Account.
\paragraph{FS/FS-Inf -} Die Fachschaft bzw. die Fachschaft Informatik. Der Begriff „Fachschaft“ wird aber auch in einem wesentlich engeren Sinn für die wenigen Leutchen benutzt, die sich ganz aktiv um die Belange der Studierenden kümmern. Siehe auch den Artikel „Fachschaftsarbeit“ (Seite \pageref{fachschaftsarbeit}).
\paragraph{HiWis -} Hilfswissenschaftler. Das sind eure \textbf{Tutoren} in den Übungen, die studentischen Mitarbeiter in der Bibliothek oder in der RBI. Diesen günstigen Arbeiterschwärmen kann man sich meist schon nach ein paar Semestern anschließen.
\paragraph{Hörsäle -} Die Hörsäle H I - H IV und H 1 - H 16 findet ihr im Hörsaalgebäude.
\paragraph{HRZ-Account -} Jeder Studierende erhält vom \textbf{\underline{H}ochschul\underline{r}echen\underline{z}entrum} einen Account. Durch diesen lassen sich verschiedene Dienste der Uni Frankfurt nutzen. Leider sind die Accountnamen inzwischen pseudonym und schlecht zu merken. Der wichtigste Nutzungsgrund liegt darin, dass ihr über den Zugang die persönlichen Daten Eures Studiums verwalten und euch zu Prüfungen anmelden könnt, siehe dazu auch \textbf{QIS-LSF}. Ansonsten braucht ihr ihn für WLAN (an fast allen Unis Deutschlands), eventuell E-Mail, FTP und andere Dinge. Die Zugangsdaten erhaltet ihr zusammen mit dem \textbf{Studentenausweis}. Nicht zu verwechseln mit dem \textbf{RBI-Account}.
\paragraph{Institutsrat -} Der Institutsrat ist ein Gremium , das die Belange des Institutes Informatik behandelt und teilweise, wo nicht der Fachbereichsrat gefragt ist, entscheidet.
\paragraph{Kommilitonen -} Sind die Leute links und rechts neben dir.
\paragraph{Lernzentrum -}  Befindet sich im Erdgeschoss des Informatikgebäudes, gleich links wenn ihr zum Haupteingang hineinkommt. Hier könnt ihr versuchen zu lernen, sollte ihr euch wieder erwarten doch inmitten von Diskussionen konzentrieren können. Im Lernzentrum sind die Skripte der momentan laufenden Basismodule vorhanden. Außerdem habt ihr hier die Möglichkeit einen Mitarbeiter der Uni um Hilfe bei euren Aufgaben zu bitten. Es gibt sogar Brettspiele zum Ausleihen. An das Lernzentrum sind die ''Student-Lounge'' genannten Räume der Fachschaft angeschlossen.
\paragraph{Magnus-Hörsaal -} Ist der einzige Hörsaal, den die Informatik im Gebäude hat.
\paragraph{Modul -} Ein Modul ist eine Lehreinheit, das die fachlich sinnvoll aus ein bis mehreren Lehrveranstaltungen zusammengesetzt ist. Ein Modul wird innerhalb eines Semesters oder auch über 2 Semester veranstaltet.
\paragraph{M.Sc. -} \glqq Master of Science\grqq . Großer Bruder vom Bachelor.
\paragraph{Munchkin -} Ein sehr beliebtes Kartenspiel im Fachbereich. 
\paragraph{Prodekan -} Sowas wie ein stellvertretender Dekan. Der Dekan und der Prodekan können auch eine Art „Aufgabenteilung“ unter sich vereinbaren.
\paragraph{Prüfungsamt -} Im Prüfungsamt meldet man sich für Klausren, mündliche Prüfungen und Studenleistungen aller Art an. Zwar geschehen inzwischen die meisten Anmeldungen über das \textbf{QIS-LFS}, aber das ist manchmal unzuverlässig.
\paragraph{Prüfungsausschuss -} Der Prüfungsausschuss ist ein Ausschuss der sich mit Prüfungen befasst (Sach bloß!). Hier siten Professoren und Studenten, die Entscheidungen treffen, die das \textbf{Prüfungsamt} nicht treffe darf.  Vor allem kann man hier Außnhameregelungen formlos beantragen. Dazu schreibt man einfach einen Brief an den Ausschuss, in dem man kurz sagt was man will, und wieso. Den kann man dann im \textbf{Prüfungsamt} einwerfen.
\paragraph{Prüfungsprotokoll-Datenbanken -} Etwas, bei dem dir ruhig mulmig zu Mute sein darf, ist eine mündliche Prüfung. Doch dem mulmigen Gefühl kann abgeholfen werden: Schau dir doch einfach mal an, was den anderen so passiert ist, als sie in ihrer Prüfung waren. Dafür gibt es eine eigene Datenbank, die ihr unter \url{http://myexam.gdv.cs.uni-frankfurt.de/} finden könnt. Mitmachen ist dabei Pflicht, denn ohne Geben gibt es bei dieser Idee auch kein Nehmen!
\paragraph{QIS-LSF -} Das elektronische Informationssystem der Uni. Hier könnt ihr euch für Prüfungen anmelden, eure Kontaktdaten ändern oder euch den Vorlesungskatalog anschauen. Für das QIS-LSF braucht ihr euren HRZ-Account
\paragraph{Rekursion -} Rekursion siehe Rekursion!
\paragraph{RBI -} \glqq Rechner-Betriebsgruppe Informatik\grqq . Die RBI kümmert sich um alles, was mit Elektronik in der Informatik zu tun hat. Über euren RBI-Account, den ihr in der RBI beantragen könnt, habt ihr pro Semester 500 Druckseiten frei und könnt noch weitere Angebote nutzen.
\paragraph{Skript -} Entweder eine frühere Mitschrift eines fleißigen Studenten (eher selten anzutreffen) oder die Kopien der vom Professor benutzten Folien, gelegentlich auch ein extra hierfür entworfener Text. Ist entweder in der entsprechenden Fachbereichs-Bibliothek, auf der Internet-Präsenz des jeweiligen Dozenten oder gar nicht zu finden. Dabei sollte man jedoch zwei Dinge beachten: 1. Das Lesen des Skriptes ersetzt nur sehr selten die Teilnahme an der Veranstaltung. 2. Wer ein komplettes Skript auf dem Drucker im RBI ausdruckt, läuft Gefahr, von vor dem Drucker wartenden Kommilitonen gesteinigt zu werden, oder zumindest als DAU (Dümmster anzunehmender User) des Monats nominiert zu werden; vgl. \textbf{RBI}.
\paragraph{Statusgruppen -} Die Statusgruppen der Universität sind Professoren, Studierende, Wissenschaftliche Mitarbeiter (WiMi) und die administrativ-technischen Mitarbeiter (SoMis).
\paragraph{Studentenausweis(Goethe-Card) -} Diese kleine Plastikkarte erfüllt gleich mehrere Aufgaben: sie dient euch als Studierendenausweis, Bibliotheksausweis und als Semesterticket. Der Zeitraum, für den die Goethe-Card gültig ist, wird mit Spezialtinte auf die Karte aufgedruckt. Nach der Rückmeldung könnt Ihr den Aufdruck erneuern, indem ihr einen der dafür vorgesehenen Automaten verwendet (z.B. im Gebäude \textit{Neue Mensa}). Durch den enthaltenen Chip werden Informationen zu den von Euch ausgeliehenen Büchern gespeichert, außerdem könnt ihr sie als Schlüssel für Schließfächer verwenden. Ihr könnt die Goethe-Card mit Geld ``aufladen'', und mit diesem Guthaben die von der Uni aufgestellten Kopierer nutzen, oder in der Mensa bezahlen.
\paragraph{Studentlounge -} Die \textbf{Fachschaftsräume} hinter dem Lernzentrum, die normalerweise offen sind und von allen Studierenden als eine Erweiterung vom \textbf{Lernzentrum} oder auch als ein Platz, wo man sich auf Sofas entspannen könnte, betrachtet werden kann. Auch hier solltet ihr allgemeine Regeln des sozialen Zusammenlebens beachten. Also haltet den Raum sauber, entsorgt euren Müll und haltet euch an die Hausordnung.
\paragraph{Studiendekan -} Auch ein Mitglied der „Drei Dekane“. Er ist im Besonderen zuständig für Fragen von Lehre und Studium.
\paragraph{Studien-Service-Center -} Befindet sich im PEG-Gebäude auf dem Westend. Hier könnt ihr alle Arten von Anträgen(Teilzeit,etc.) stellen.
\paragraph{StuPa -} Das StuPa ist, wie der Name „Studentenparlament“ ja auch sagt, das Parlament der Studierenden. Und wie es sich für jedes ordentliche Parlament gehört, wird auch im StuPa Politik gemacht: Deshalb hält man die StuPa-Sitzungen auch kaum aus. Studentische Interessen stehen leider eher selten im Vordergrund. Welche politischen Hochschulgruppen in das StuPa reingewählt werden, könnt ihr versuchen in Wahlen zu beeinflussen.
\paragraph{SWS -} Abkürzung für Semesterwochenstunde, sprich die Zeit, die eine Veranstaltung im Verlauf des Semesters wöchentlich dauert. Die private Vor- bzw. Nacharbeit dauert normalerweise noch das doppelte dazu. Also: 1 SWS = 3 Stunden fürs Studium wochentlich.
\paragraph{Tutor -} Als Tutor endet man aus drei verschiedenen Gründen. Ein Grund wäre, dass ihr CP für euer Ergänzungsmodul braucht und euch dadurch entschieden habt Tutor zu werden. Der zweite Grund wäre, dass ihr Geld braucht und für einen Studentlohn von 8,50 Euro arbeiten wollt. Oder aber ihr wollt euer Wissen weiter geben bzw. euer Wissen auffrischen. Eine Mischung aus den drei Gründen tritt auch gelegentlich auf.
\paragraph{Umzug -} Eine sehr beliebte Sage innerhalb des Fachbereichs 12. Von Generation zu Generation wird weiter verbreitet, dass wir zum Riedberg umziehen werden. Erste historische Erwähnung dieser Sage wurde in den Protokollen der 4. Ratssitzung des Direktorium für Informatik im Jahre 1979 gefunden.
\paragraph{Variable, metasyntaktisch -} Platzhalter und genereller Ausdruck der Faulheit von Informatikern. Erlaubt einem sich unpr\"azise auszudr\"ucken. Beispielsweise seien hier foo, bar, foobar, xyzzy, Ding, Dinge und Zeuch genannt. Zeuch nimmt dabei eine Sonderstellung ein, da dadurch gerne komplette Prozeduren, Programmbl\"ocke und Krempel von Dingen die Zeuch machen beschrieben wird.
\paragraph{Wahlen -} Irgendwann werdet ihr ein Brief nach Hause bekommen, womit ihr z.B. den FBR oder das StuPa wählen könnt. Wer oder was gew\"ahlt wird ist dabei vielen Studenten leidenschaftlich unklar.
\paragraph{Wert, metasyntaktisch} Wenn einem nichts einf\"allt nimmt man die. Metasyntaktische Werte haben oft eine besondere Bedeutung, die aber mit der Zeit in Vergessenheit geraten ist. Beispiele sind 7, 42, 37, 0815, 1337, 31337, \texttt{0xdeadbeef}, \texttt{0xcafebabe} und f\"ur Stringkonstanten ''foo''. Werden fast immer \textit{metasyntaktischen Variablen} zugeordnet.
\paragraph{Westend, Campus -} Der Campus der Geisteswissenschaften. Zu erreichen per Bus mit Linie 36 oder 75, per U-Bahn 1,2,3 oder 8 (Holzhausenstraße) oder falls man Zeit hat auch zu Fuß durch den Palmengarten.
\paragraph{WiMis -} Wissenschaftliche Mitarbeiter. Sie sind an einer Professur angestellt, um dort zu promovieren, also um irgendwann einmal einen Doktortitel zu erhalten. Bis dahin forschen sie fleissig und unterstützen den Professor bei Lehrveranstaltungen. Gerade in Übungen werdet ihr ab und zu mit einem WiMi zu tun haben.
\spaltenende

\begin{center}
	\includegraphics[scale=0.5]{comics/dictionary-attack}
\end{center}

%%%%%%%%%%%%%%%%%%%%%%%%%%%%%%%%%%%%%%%%%%%%%%%%%%%%
% Studium
\section{Euer Studium}
%%%%%%%%%%%%%%%%%%%%%%%%%%%%%%%%%%%%%%%%%%%%%%%%%%%%
% Stundenplan
\subsection{Ein Stundenplan zum Ausfüllen}
\newcommand{\x}{\parbox[0pt][2.7cm][c]{3.9cm}{\hspace{3.9cm}} }
\newcommand{\xT}{\parbox[0pt][1.6cm][c]{1.9cm}{\hspace{3.9cm}} }
% --- Farbdefinitionen ----------------------------------------
\definecolor{hellgrau}{rgb}{0.95,0.95,0.95}

%\begin{sidewaystable}
%\begin{addmargin}[-2mm]{0cm}

%\begin{center}
%\begin{tabular}{|c|c|c|c|c|c|}
%\hline Zeit & Montag & Dienstag & Mittwoch & Donnerstag & Freitag \\ 
%\hline 8-10  & \x & \x & \x & \x & \x \\
%\hline 10-12 & \x & \x & \x & \x & \x \\ 
%\hline 12-14 & \x & \x & \x & \x & \x \\ 
%\hline 14-16 & \x & \x & \x & \x & \x \\ 
%\hline 16-18 & \x & \x & \x & \footnotesize 16:00 : Fachschaftstreffen & \x \\ 
%\hline ab 18 & \x & \x & \x & \x & \x \\ 
%\hline 
%\end{tabular}
%\end{center}

%\end{addmargin}
%\end{sidewaystable}

\vspace{24.2cm} % ein Korrekturabstand wg. Drehen
\begin{center}
\begin{rotate}{90}
\begin{tabular}{|c|c|c|c|c|c|}
\hline \cellcolor{black} \xT
	& \cellcolor{black} \textcolor{white}{\textbf{Montag}}
	& \cellcolor{black} \textcolor{white}{\textbf{Dienstag}}
	& \cellcolor{black} \textcolor{white}{\textbf{Mittwoch}}
	& \cellcolor{black} \textcolor{white}{\textbf{Donnerstag}}
	& \cellcolor{black} \textcolor{white}{\textbf{Freitag}} \\ 
\hline 8:00-10:00  & \x & \x & \x & \x & \x \\
\hline  \cellcolor{hellgrau}10:00-12:00 &  \cellcolor{hellgrau} \x &  \cellcolor{hellgrau} \x &  \cellcolor{hellgrau} \x &  \cellcolor{hellgrau} \x &  \cellcolor{hellgrau} \x \\ 
\hline 12:00-14:00 & \x & \x & \x & \x & \x \\ 
\hline  \cellcolor{hellgrau} 14:00-16:00 &  \cellcolor{hellgrau} \x &  \cellcolor{hellgrau} \x &  \cellcolor{hellgrau} \x &  \cellcolor{hellgrau} \x &  \cellcolor{hellgrau} \x \\ 
\hline 16:00-18:00 & \x & \x & \x & \footnotesize & \x \\ 
\hline  \cellcolor{hellgrau} 18:00-20:00 &  \cellcolor{hellgrau} \x &  \cellcolor{hellgrau} \x &  \cellcolor{hellgrau} \x &  \cellcolor{hellgrau} \x &  \cellcolor{hellgrau} \x \\ 
\hline 
\end{tabular}
\end{rotate}
\end{center}
\subsection{Studienverlaufsplan}
\includegraphics[width=18cm]{bilder/basismodulebachelorSS}
\newline
\includegraphics[width=18cm]{bilder/sonstigemodulebachelor}
\newpage
%%%%%%%%%%%%%%%%%%%%%%%%%%%%%%%%%%%%%%%%%%%%%%%%%%%%
% Bachelor Rollenspiel
\subsection{BA Informatik - Ein verkorkstes Rollenspiel}
\spaltenanfang
TODO: mehr Ragecomics einbauen...

\section*{Der Bachelor Informatik - ein vollkommen verkorkstes Rollenspielsystem}
Was ist eingendlich so ein Bachlor? Denn mein Englisch-Deutsch-W\"orterbuch
sagt, dass ich mich in einen Studiengang eingetragen habe, in dem ich zum
''Jungesellen der Wissenschaft Informatik'' gemacht werden soll. Und der
Jungesellenstatus in der Wissenschaft kann nicht wirklich mein Ziel sein, oder?

Wie immer: Don't Panic! Bachelor ist nur ein Name, den die Politik \"ubernommen
hat, um internationaler zu klingen.  So ist das halt bei Hochschulreformen. Und
au{\ss}erdem ist was der Bachelor genau ist erst wichtig wenn ihr fertig seit.
Jetzt ist erstmal wichtig dass ihr sinnvoll studieren k\"onnt. Und dazu solltet
ihr ein paar Grundlagen wissen.

\textbf{Merke: Das Bachelorstudium ist im Prinzip ein verkorkstes Rollenspiel}

Als erstes gibt es f\"ur Rollenspielfanatiker und Munchkins das offizielle
Regelwerk zum Studium: Die \textsc{\textit{Bachelorordnung}}. Die besteht wie jedes
Regelwerk aus einem kleinen Teil mit Regeln
%TODO:*U NO SAY*
und dem Teil mit den riesigen
Tabellen, die Zeuch beschreiben. Aber anstatt coole R\"ustung und so gibt es
nur Skills. Nur hei{\ss}en die hier Module und man bekommt die erst, wenn man sie
verdient hat.

Ausserdem gibt es sowas wie XP. Die heissen hier aber CP, weil das alles
bitterer Ernst ist und sich deshalb nicht an g\"angige Rollenspiel
Designkonventionen gehalten wird. Und von denen bekommt man etwa einen pro 30
Stunden Studium.

Aber auch das funktioniert irgendwie anders als man das gewohnt ist. Statt dass
man XP bekommt, mit denen man sich bessere Skills holt um dann in Proben
bessere Chancen zu haben, muss man hier erst die Pr\"ufungen bestehen, bekommt
dann das Modul und damit die CP.
%TODO: *DAFU?*.
{\large
\begin{verbatim}
[XP] -> [skills] -> [pruefungen]
[CP] <- [module] <- [pruefungen]
\end{verbatim}}

%---------

\textbf{Bacheloraufbau:}
Schauen wir uns also das System mal genauer an.
Das Ziel ist es den Abschluss Bachelor zu bekommen. Dazu muss man 180CP
bekommen haben, die man mit dem Abschluss von Modulen bekommen hat. Und wie
jeder weiss, muss ein guter Munchkin Min-Maxen. Aber auch das kann der Bachelor
nicht gut.

Um Module abzuschliessen muss man eine Klausur schreiben, eine m\"undliche
Pr\"ufung machen, einen Vortrag halten, gen\"ugend Abgaben gemacht haben, oder
sonst irgendwie gezeigt haben, dass man die CP auch wirklich verdient hat.


\textbf{Die Modulkategorien:}
Die Module sind wie f\"ur Skills \"ublich in verschiedene Kategorien
eingeteilt. Das kann euch sowohl einschr\"anken als auch Freiheiten geben.
Als erstes sind f\"ur euch die \emph{Basismodule} interessant, denn die m\"usst ihr als Informatiker alle machen.\\
Dann kommen die \emph{Vertiefungsmodule} ins Spiel. Hier kann man seinen Studenten in mindestens drei von f\"unf Kategorien spezialisieren.\\
Die \emph{Anwendungsfachmodule} sind Multiklassenskills, in denen ihr Fertigkeiten aus anderen F\"achern lernen sollt.\\
\emph{Erg\"anzungsmodule} sind durch die \texttt{Gute Idee\texttrademark} entstanden, dass Informatiker mindestens 150 Stunden Soft Skills oder soziales Zeuch gemacht haben sollten. Wir wollen doch keine verschrobenen antisozialen Studenten bauen.\\
Und zum Abschluss gibt es das \emph{Abschlussmodul}, das die Bachelorarbeit und das Oberseminar \"uber die Arbeiten enth\"alt.\\

\textbf{Basismodule:}
Es gibt ein paar Skills die jeder Informatiker haben sollte. Und es gibt
Basismodule. In den Basismodulen lernt ihr offiziell vier Dinge: Mathe, wie
Computer funktionieren (wird gerne auch ''Hardware'' genannt), wie man Computer
programmiert und Theoretische Informatik. Das sind alles Vorlesungen, bis auf
das Programmier-Praktikum und das Hardware-Praktikum.
Insgesamt gibt es hier 93CP zu holen.

\textbf{Vertiefungsmodule:}
Die gibt es in f\"unf Spezialisierungen: BKSPP, ISWV, TS, ANI und GDI. ''Betriebs-
und Kommunikationssysteme und Progammiersprachen und -paradigmen'' enth\"alt
genau was im Titel steht. ''Informationssysteme und Wissensverarbeitung''
besch\"aftigt sich mit Textverarbeitung , Datenbanken und k\"unstlicher
Intelligenz. In ''Technische Systeme'' lernt man mehr \"uber Microcontroller,
Rechnerarchitektur und Chipdesign. In ''Angewandter Informatik'' ist alles was
sonst nicht untergebracht werden konnte. Da sind dann so Sachen wie
Computergrafik, Zeug und Krempel drin. Wer aber statt Zeug und Krempel lieber
was Abgehobenes macht, kann in ''Grundlagen der Informatik'' fast schon ein
Mathe Studium simulieren. Dort gibt es mehr Logik, mehr Algorithmentheorie und mehr
Beweise.

Und jetzt kommt der Haken: Du musst mindestens 43CP machen. Davon in einem Modul
mindestens 16CP und in zwei anderen mindestens 8CP. Die anderen 11CP kannst
du machen wie du willst. Dabei musst du die drei Vertiefungsgebiete, in denen
du leveln willst, dem Pr\"ufungsamt \underline{vor} der ersten Klausur mitteilen. Alles
Klar? OK!

\textbf{Anwendungsfachmodule:}
Die Muliklassenskills, auch Anwendungsfachmodule genannt, erlauben euch was anderes als Informatik zu lernen.
Die mei{\ss}ten dieser Nebenfachmodule sind geregelt. Das hei{\ss}t, das
Basisregelwerk ''Die Bachelorordnung'' hat vorgefertigte L\"osungen wie ihr das
Nebenfach machen k\"onnt. Ist euer Nebenfach nicht drin, k\"onnt ihr das trotzdem studieren, aber dann halt ungeregelt.
Dazu muss unser Pr\"ufungsamt mit dem ''gegnerischen'' Pr\"ufungsamt ''kl\"aren'' wie das ganze abgewickelt werden soll.
Fakt ist aber, ihr braucht immer mindestens 24CP.

\textbf{Erg\"anzungsmodule:}
Irgendein schlauer Informatiker hat sich mal gedacht: \emph{Wir wollen, das
Frankfurter Informatiker nicht nur Fachidioten sind, sondern auch ''sozial''
sein k\"onnen. Die haben dann ''Soft Skills''. Wie ''Teamf\"ahigkeit'' und so. Und das
ist dann gut.} Aber zu viel sozial nicht in den Studienverlaufsplan passt, macht ihr 5CP.
Da ist dann auch die Studienorientierung STO drin. Die im ersten Semester 2CP gibt, und falls du sie sp\"ater machst - nur noch halb so viel.
Da wird euch nochmal auf dem Silbertablett gereicht, wie man studiert.

\textbf{Die restlichen 15CP:}
Gibt es mit der Bachelorarbeit.

\textbf{Die Bachelorarbeit:}
Das ist sowas wie eine echte wissenschaftliche Arbeit. Wenn ihr die gemacht habt gibt das 15CP im Abschlussmodul.

\textbf{Die Klausuregeln:}
Und nun steigt die Spannung. Die Klausurphase beginnt.
W\"ahrend im Semester Wochen vergehen k\"onnen, ohne das viel passiert, sind die Klausurphasen eher wie Kampfrunden.
Eine Megasekunde, die im echten Leben vergeht, kommt einem wie eine Gigasekunde an der Uni vor.
Aber keine Panik, es gibt einige Tricks und ein paar Regeln, die ihr
anwenden k\"onnt, um sinnvoll, lebendig, und mit ein paar mehr CP durch die
Klausurphase zu kommen.

\textbf{Anmeldung:}
Zu den Klausuren m\"usst ihr euch meist zwei Wochen vorher anmelden. Und vor der ersten Klausur m\"usst ihr euch f\"ur den Bachelor anmelden.
Wenn ihr die Klausur schreibt und nicht angemeldet seit, gibt das keine XP.

\textbf{Timing:}
Zuerst ist wichtig zu planen wann ihr Klausuren schreibt.
Die Termine selbst k\"onnt ihr zwar nicht \"andern, aber ein guter Munchkin hat die Bachelorordnung gelesen
und hat festgestellt, dass viele Klausuren jedes Semester angeboten werden. Aber wie soll mir die Klausur verschieben helfen?
N\"achstes Semester sind doch wieder Klausuren, oder? Aber da ist es wichtig
zu wissen, wie die Professoren diese Regel mit jedem Semester auslegen.
Professoren waren alle mal Studenten und sind deshalb fast so faul wie wir.
Und fast immer m\"ussen Nachklausuren angeboten werden, f\"ur Studenten, die durchgefallen sind oder nicht teilnehmen konnten.
Aber w\"ahrend die Vorlesungen meistens in der dritten April- oder Oktoberwoche anfangen f\"angt das Semester p\"unklich am Ersten an.
Und deshalb sind die Nachklausuren meistens am Ende der Semesterferien, aber im neuen Semester. Der
ge\"ubte Munchkin braucht also seine Klausuren nicht alle auf einmal zu
schreiben, sondern lernt am Anfang \emph{und} am Ende der Semesterferien.
Ansonsten gilt: \textbf{Konzentriere dich auf wenige Klausuren, statt in allen zu versagen.}
\emph{Bla, bla, lernt rechtzeitig, bla\dots}

\textbf{Freiversuche - Rerolls f\"ur Klausuren:}
Und jetzt kommen die kleinen Feinheiten des Regelwerks, die jeden Munchkin interessieren sollten.
Ger\"uchteweise haben manche Studenten Klausuren mitgeschrieben und kl\"aglich versagt. Daran ist noch nichts besonderes.
Aber die f\"ur diese Studenten war es so als h\"atten sie die Klausur nie
geschrieben und haben sich einfach beim n\"achsten mal wieder in die Klausur
gesetzt. Und noch viel besser. Andere die grad so bestanden hatten. Sa{\ss}en
in der Klausur und konnten nochmal mitschreiben und haben am Ende eine bessere
Note gehabt. Wie geht das?

Naja, eigentlich ist nichts besonderes daran. Wenn du innerhalb der
Freiversuchsfrist eine Klausur schreibst, kannst du beim ersten Mal
Durchfallen, ohne dass das als Fehlversuch gez\"ahlt wird.
Die Freiversuchsfrist ist in der Bachelorordnung festgelegt und ist an den Studienverlauf angepasst.
Das soll euch dazu motivieren die Klausur zu beim erstem Mal oder in der Nachklausur mitzuschreiben.
Das funktioniert aber nur in den Basismodulen so. In den Vertiefungs- und Anwendungsfachmodulen funktioniert das leider nicht.
Aber die Basismodule machen \"uber die H\"alfte des Studiums aus, also ist das gar nicht so schlecht.

Und wer wider erwarten die Klausur besteht und mit seiner Note nicht zufrieden
ist, kann die Klausur nochmal schreiben. Dazu muss man sich nach der Klausur
f\"ur die n\"achste Klausur anmelden. Das kann man aber auch nur in den
Basismodulen und insgesamt nur f\"unf mal. Da es aber nur 9 Klausuren in den Basismodulen gibt,
ist das immer noch viel. Aber die genauen Regeln bekommt ihr noch rechtzeitig in der Studienorientierung erz\"ahlt.

\textbf{Studienorientierung:}
\textbf{
Die Studienorientierung ist soooo wichig dass wir einen extra Artikel geschrieben haben. Macht die im ersten Semester und alles wird gut. Ausserdem haben wir diesen Absatz extra fettgedruckt!}

\textbf{FAIL - das Howto:}
Manchmal muss man einfach alle Br\"ucken hinter sich lassen. Und manchmal will man nicht gehen, sondern rausfliegen. Und so gehts:

\textbf{FAIL - die sparsame Methode:}
Bezahl deine Studiengeb\"uhren nicht. So einfach ist das. Die bezahlt man sonst immer im Januar oder im Juli. Aber wer wirklich raus will findet in der Sparsamkeit eine wirksame Methode.

\textbf{FAIL - die faule Methode:}
Mach in den ersten drei Semestern weniger als 15CP. Das sind so ungef\"ahr zwei Module von zehn. Dann musst du nur noch die Anfragen vom Pr\"ufungsamt ignorieren und keine Fristverl\"angerung beantragen. Dann bist du raus.

\textbf{ULTRAFAIL - die einzig wahre Methode:}
Wenn du nicht nur rausfliegen willst, sondern gar nicht mehr Informatik
studieren k\"onnen willst, solltest du dreimal durch eine Pr\"ufung in einem
Basismodul durchfallen (viermal, wenn du schon in der Freiversuchsfrist
anf\"angst). Wer eine Pr\"ufung dreimal nicht besteht, hat ''endg\"ultig nicht
bestanden''. Und wenn das so etwas grundlegendes wie Programmierung ist, kannst
du nicht mehr Informatik studieren, auch nicht woanders in Deutschland. Auch
hier musst du darauf achten, alle Beratungsgespr\"ache zur\"uckzuweisen.

\textbf{N\"utzliche Tips:}
\emph{bla, bla, \"Ubungsabgaben, yadda, yadda, Lernen, bla, blubber, Zeiteinteilung, bla, regelm\"a{\ss}ig da sein, trololo, Durchhalteverm\"ogen, bla, generische Motivationsrede}

\textbf{Und nun?}
Naja, da sind noch mehr Seiten in der Don't Panic! 42. Die k\"onnt ihr auch
lesen. Wichtig ist vor allem sich an der Uni einzuleben, wenn man erfolgreich
studieren will. Und das sieht f\"ur jeden anders aus. Manche machen ihr Studium
schnell und andere lassen sich Zeit. Und wichtiger als die Skills, die ihr hier
bekommt, ist die Erfahrung. Wenn ihr hier fertig seit, m\"usst ihr eh wieder
neue Sachen lernen.






\spaltenende
\newpage
%%%%%%%%%%%%%%%%%%%%%%%%%%%%%%%%%%%%%%%%%%%%%%%%%%%%
% STO
\subsubsection{STO}
Bereits \textbf{im ersten Semester} wird den Bachelor-Studierenden die Teilnahme an der Studiumsorientierungveranstaltung “Einführung in das Studium” (STO) empfohlen. Diese Veranstaltung ist der Pflichtanteil des Ergänzungsmoduls im \textbf{Bachelor Informatik} und \textbf{muss} aus diesem Grund besucht werden. Dafür gibt es 2 CPs.

Da STO gar nicht anstrengend ist, \textbf{keine Abschlussprüfung} enthält und das Studium sehr stark erleichtert, ist die Teilnahme daran bereits im ersten Semester sehr sinnvoll.

Die Veranstaltung besteht aus dem Vorlesungsanteil, wo euch eher allgemein erzählt wird, wie man am effizientesten studiert und dem sogenannten Peer-Mentoring.

Der Grundgedanke des Peer-Mentorings ist es, euch den Einstieg ins Studium zu erleichtern, denn oftmals sind viele Erstsemester überfordert, wissen einfach nicht wo sie welche Informationen herbekommen oder trauen sich einfach nicht zu fragen. Oftmals handelt es sich dabei um die einfachsten Dinge.

Ihr werdet am Anfang eures Studiums in mehrere Kleingruppen unterteilt. Diese bestehen aus jeweils etwa 10 Studenten (Mentees) und einem erfahrenen Studenten aus einem höheren Semester (Mentor).

\begin{center}
	\includegraphics[scale=1.0]{comics/groups-meeting-light-icon}
\end{center}

Im Laufe des Semesters wird sich jedes Mentorium etwa alle 2-3 Wochen einmal treffen. Dort werden euch dann alle Fragen rund ums Studium beantwortet. Außerdem steht es euch jederzeit frei euren Mentoren einfach eine E-Mail zu schreiben. Sie werden euch jederzeit gerne helfen.

Das Mentoring soll euch hierbei auch schon eine weitere Möglichkeit geben, um in kleinen Kreisen neue Leute kennen zu lernen und gegebenenfalls Lerngruppen zu bilden, denn ihr werdet mit der Zeit bemerken, dass Studium und Schule sich sehr unterscheiden und man hier allein nicht alles schaffen kann.

Ihr werdet lernen, dass es hier viel wichtiger als in der Schule ist, eigenverantwortlich zu arbeiten und selbstständig zu werden, denn ihr habt im Studiengang Informatik in den meisten Fällen keine Anwesenheitspflicht. Auch die Übungen die ihr regelmäßig in den Vorlesungen ausgeteilt bekommt, solltet ihr alle selbst machen. Keiner wird sich darum kümmern, ob ihr sie macht oder nicht.

\begin{center}
	\includegraphics[scale=0.8]{comics/study-icon}
\end{center}

Wie Vieles nun ganz genau abläuft und geregelt ist, was ihr genau wann und wo zu tun habt,
bzw. wann und wie ihr euch für irgendwelche Prüfungen anmelden könnt oder wie ihr euer Studium generell plant und alles was für euch noch so wichtig ist, erfahrt ihr im Mentorium.

Das Mentorium behandelt also im allgemeinen organisatorische Themen, aber ihr könnt das Ganze auch als Kummerkasten betrachten. Was im Mentorium besprochen wird, bleibt auch im Mentorium. Also ist euer Mentor schonmal eine Art Vertrauensperson, die versuchen wird, auch individuell auf euch einzugehen, denn jeder Mensch geht anders mit den Begebenheiten um, wie man lernt und wie man seine Zeit plant. Dabei wird euer Mentor jedoch nicht sagen wie ihr zu lernen habt oder eure Zeit einteilen müsst, sondern wird euch lediglich helfen, den für euch richtigen Weg oder Kompromiss selbst zu finden.

\begin{flushright}Markus S., modified by Pavel \end{flushright}
\begin{center}
	\includegraphics[scale=0.8]{comics/time_management.png}
\end{center}
\newpage
%%%%%%%%%%%%%%%%%%%%%%%%%%%%%%%%%%%%%%%%%%%%%%%%%%%%
% Formen
\subsubsection{Veranstaltungs- und Prüfungsformen}
Im folgenden werden wir verschiedene Veranstaltungs- und Prüfungsformen, die dir während deines Studiums über den Weg laufen können, kurz vorstellen, damit du dich darauf einstellen kannst, was alles auf dich zukommen kann. Die unten aufgezählten weiteren Kriterien können euch bei den Modulprüfungen begegnen. In jedem Fall müssen die genauen Kriterien für die zu erbringenden Leistungen spätestens in der ersten Veranstaltungswoche von dem Professor bekanntgeben werden, der die Veranstaltung hält.


\paragraph{Protokolle} In den Praktika musst du meistens wöchentlich eine oder ein paar Aufgaben lösen und diese sehr ordentlich und ausführlich protokollieren. Nur wer alle (oder fast alle – je nachdem) Versuchsprotokolle mit einem „ok“ gekennzeichnet bekommen hat, bekommt einen Praktikumsschein. Es kann aber auch möglich sein, dass zusätzlich noch mehr verlangt wird: Das Bestehen einer Klausur, das Teilnehmen an einem Seminar. Der Phantasie sind da kaum Grenzen gesetzt.


\paragraph{Hausarbeit} Hausarbeiten sind in der Informatik unüblich, in anderen Fächern wie zum Beispiel Pädagogik oder Politologie in Seminaren aber aufgrund der hohen Teilnehmerzahlen durchaus gebräuchlich. Eine Hausarbeit kannst du dir als einen mittelgroßen Aufsatz vorstellen, der nicht mit einem mündlichen Vortrag oder Referat verbunden ist. Und was „mittelgroß“ bedeutet, hängt unter anderem davon ab, wie lang oder wie kurz du ein bestimmtes Thema bearbeiten kannst. Im Zweifelsfall wird dir aber dein Veranstalter einen Tipp geben.


\paragraph{Referat} Referate sind in der Informatik ebenfalls unüblich, hier musst du in der Regel einen kurzen Vortrag über ein bestimmtes Thema halten, manchmal ist zusätzlich eine schriftliche Zusammenfassung zu leisten.


\paragraph{Seminar} Hier ist in der Regel eine schriftliche Ausarbeitung und ein Vortrag (mit anschließender Diskussion) zu einem bestimmten Thema zu leisten. Der Aufwand sollte nicht unterschätzt werden, da je nach Thema einiges an Literatur recherchiert und durchgearbeitet werden muss.

\paragraph{Vorlesung} Hier trägt der Veranstalter zu einem bestimmten Thema vor, dass dann auch in Übungen, falls angeboten, vertieft wird. Du kannst dir Vorlesungen als Frontalunterricht vorstellen, bei dem du zuhören darfst und solltest. Wenn du Glück hast, darfst du sogar ab und zu eine Frage stellen. Ganz engagierte Dozenten stellen sogar ab und zu eine Frage an die Hörer, um sicherzustellen, dass sie ihm auch gedanklich folgen. Vorlesungen an der Universität verlaufen aber oft sehr viel schneller als die Kurse in der Schule. Das heißt: \textbf{Passt du heute nicht auf, weißt du morgen schon nicht mehr, worum es überhaupt geht,} und aufzupassen fällt dir immer schwerer, und übermorgen gehst du vor lauter Frust schon gar nicht mehr hin. Die Übungszettel sind eine Möglichkeit für dich, sich mit dem Stoff der Vorlesung nochmal auseinanderzusetzen. Immerhin bist du nicht gezwungen, die Aufgaben allein zu lösen, sondern du darfst dich mit deiner Lerngruppe an die Zettel setzen, und mit mehreren Leuten hat man ja oft auch mehr Ideen. 


\paragraph{Mündliche Prüfung, „Fachgespräch“} Mündliche Fachgespräche werden oft in Vorlesungen Angeboten, wenn zu erwarten ist, dass sich nicht allzu viele Studis prüfen lassen werden. Ein 
solches Gespräch soll 20 bis 40 Minuten dauern. Am besten, du schaust dir vorher mal Prüfungsprotokolle an (beispielsweise auf der Homepage der Fachschaft angeboten), damit du einen Eindruck vom Prüfungsstil und den Lieblingsfragen des Prüfers bekommst.


\paragraph{Klausur} Eine Klausur ist in vielen Basismodulen vorgesehen. Sie dauert zwei bis vier Stunden, zum Bestehen musst du einen bestimmten Anteil der möglichen Punkte auch wirklich erreichen. Meistens liegt dieser Anteil so grob bei 50 \%, es kann je nach Fach und Schwierigkeit des Stoffs aber auch weniger sein!


\paragraph{Übungen} Jede Woche bekommst du einen Zettel mit Aufgaben, die du bis zur folgenden Woche lösen sollst. In den  „Tutorien“ oder „Übungen“ werden sie dann besprochen (dazu weiter unten noch mehr). Übungen sind etwas anders als Hausaufgaben. Einerseits solltest du sie zu Hause lösen. Andererseits bekommst du aber auch keine 6 eingetragen, wenn du sie nicht machst. Du musst dir noch nicht einmal eine gute Ausrede einfallen lassen.

Oft werden die letzten beiden Kriterien, Klausur und Übungen, kombiniert: Es gibt Übungsaufgaben, und die Punkte, die du damit erreichst, werden schon für die Klausur angerechnet; oder sie dienen dir nur zu deiner Vorbereitung auf die Klausur.


\paragraph{Tutoriumsleitung} Im Bachelor-Studiengang ist es möglich, durch das Leiten eines Tutoriums das Ergänzungsmodul zu komplettieren. Dadurch hast du nicht nur die Möglichkeit, das Institut für Informatik direkt zu unterstützen, sondern kannst außerdem noch wertvolle Erfahrungen im Bereich der Soft Skills sammeln.


\paragraph{Tutorien} Und wie ist das mit den Tutorien? Einmal in der Woche findet eine Veranstaltung bei eurem Tutor statt. Das ist auch ein Studi, der mit seinem Studium schon weiter ist und sich jetzt bereit erklärt hat, euch ein wenig beim Studieren zu helfen. Während eines solchen Tutoriums werdet ihr die Übungsaufgaben besprechen; meistens wird einer von euch eine Aufgabe vorrechnen. Das hat nichts damit zu tun, euch „vorführen“ zu wollen. Es sind aber zwei verschiedene Dinge, eine Lösung auf einen Zettel zu malen und eine Lösung anderen laut und mündlich, mit Hilfe einer Tafel, zu erklären. Die Möglichkeit, das zu üben, solltet ihr euch nicht entgehen lassen; ihr profitiert davon nicht nur für Seminarvorträge! Auch wenn ihr von eurem Tutor Punkte auf eure Lösungsversuche bekommt (oder viel zu oft nicht bekommt), ist es für euch keine Schande zuzugeben, dass ihr etwas nicht verstanden habt und euren Tutor danach zu fragen. Dafür sind Tutorien da!

Es ist wohl möglich, existierende Lösungen abzuschreiben und sich in den Tutorien wie eine Maus zu verkriechen und bloß nicht aufzufallen. Es haben auch schon Leute auf diese Weise einen Übungsschein (ohne Klausur) erhalten. Ihr wisst schon, gelernt will der Stoff trotzdem werden, denn in der mündlichen Prüfung gibt es kein Abschreiben und Verstecken.


Oft aber verlaufen Tutorien ganz anders. Der Tutor schreibt an und wirbelt herum, reißt sich ein Bein aus und redet sich ’nen Wolf, und alle Teilnehmer sitzen gelangweilt im Seminarraum, gucken an die Decke und bohren in der Nase. Was glaubt ihr, könnt ihr in einem solchen Tutorium lernen? Na? Wie sinnvoll und interessant und hilfreich ein Tutorium ist, ist zu einem großen Teil davon abhängig, wie die Teilnehmer es gestalten. Macht was draus!

\begin{flushright}Oliver und Sebastian\end{flushright}


	\begin{center}
		\includegraphics[scale=1.0]{bilder/rm_pano}
	\end{center}
\subsection{Sonstige Studiengänge}
	\subsubsection{MA Info}
	\spaltenanfang
Nachdem du es gepackt hast, einen Bachelor mit einer Note von nicht schlechter als 3.0 zu erwerben, kannst du dich für den Masterstudiengang an der Goethe-Universität Frankfurt entscheiden.
Die meisten Masterstudenten haben ja schon einmal gezeigt, dass sie
erfolgreich studieren k\"onnen. Aber da es ja mal hie{\ss} „Man bekommt sein Diplom
daf\"ur, dass man die Studienordnung gelesen hat.“ geben wir nochmal einen
kurzen \"Uberblick.

\textbf{Schwerpunkte:}
Zun\"achst sollte beachtet werden, dass es so einen Master bei uns in vier neuen unterschiedlichen Geschmacksrichtungen gibt. Aber statt alle zu kosten, muss man sich f\"ur einen der sogenannten \emph{Schwerpunkte} entscheiden:
\begin{noindEnumerate}
\item Master mit Anwendungsfach
\item Master mit vertieftem Anwendungsfach
\item Master in allgemeiner Informatik
\item Master mit Spezialisierung
\end{noindEnumerate}

Wenn du im Bachelor kein Anwendungsfach studiert hast, \textbf{musst} du den Master mit Anwendungsfach machen.

Wenn du schon ein Anwendungsfach hattest, kannst du dich entweder im selben Fach weiter vertiefen, oder ein ganz neues Anwendungsfach belegen.

Wer der Meinung ist, dass alles au{\ss}er Info doof ist, kann den Master in allgemeiner Informatik oder den Master mit Spezialisierung machen. 

\textbf{Auflagen:}
Wer zwischen Bachelor und Master die Uni wechselt, kann f\"ur das Masterstudium Auflagen bekommen.
Das sind Veranstaltungen mit Grundlagen, die vorrausgesetzt wurden, die man im Bachelor aber nicht nachweisen konnte.
Das hei{\ss}t im Klartext, dass bis zu 30CP an Modulen, die euch dann gesagt werden, in den ersten 14 Monaten gemacht werden m\"ussen, ohne, dass sie angerechnet werden. Aber das sollte kein Problem sein, das sind immerhin Grundlagen, und oft nur ein oder zwei Veranstaltungen, die ihr dann machen m\"usst. Wenn ihr den Bachelor in Frankfurt gemacht habt, wird euch das sowieso nicht passieren.

\textbf{Aufbau:}
Insgesamt geht das Masterstudium über vier Semester und besteht aus 120 CP.
Egal für welche der vier Formen des Masterstudiums du dich entscheidest, am Ende musst du bei allen im Abschlussmodul eine Masterarbeit schreiben, die 30 CP bringt und gewöhnlich über ein Semester geht (6 Monate).
Die Informatikmodule sind in drei Gebiete aufgeteilt: „Informatik der Systeme“, „Grundlagen der Informatik“ und „Angewandte Informatik“. Je nach gewähltem Schwerpunkt musst du aus jedem der Gebiete eine bestimmte Anzahl an CP erbringen. Au{\ss}erdem muss man ein Seminar und ein Praktikum machen. Wer allgemeine Informatik studiert, macht hier 84-87CP, alle anderen nur 60-63CP.
Wer sich f\"ur ein Anwendungsfach oder die Spezialisierung entschieden hat, muss darin 24CP machen.
Zusätzlich sind in jedem Schwerpunkt Ergänzungsmodule zu belegen, die 3 bis 6~CP bringen und meistens durch Soft-Skill Workshops erreicht werden.

\textbf{Masterordnung:} Wer mehr wissen will, sollte sich die Masterordnung anschauen.\footnote{siehe \url{http://www.cs.uni-frankfurt.de/images/pdf/informatik/master2015/ordnung_master_2015.pdf}} Das sieht wie immer nach mehr aus, als es ist. Haltet einfach nach bunten Bildern Ausschau.

\spaltenende

	\subsubsection{MA Bioinformatik}
	
	\subsubsection{MA Wirtschaftsinformatik}
\newpage
	\subsubsection{Informatik L3}
	\spaltenanfang
Seit dem Wintersemester 1997/98 kann man in Frankfurt Informatik als Lehramtsfach im L3-Studiengang (Gymnasiales Lehramt) studieren – mit sehr guten Berufsaussichten, denn viele entschließen sich nicht dazu und viele hören schon bald wieder auf\ldots Warum eigentlich? So ein Info-Lehramtsstudium ist doch ganz einfach. Zunächst einmal stellen wir die für das Lehramtstudium relevanten Ämter vor:


LSA - \textit{Das Landesschulamt}
Das Landesschulamt (LSA) (ehemals: AfL / Amt für Lehrerbildung)
Das LSA ist dem Hessischen Kultusministerium direkt untergeordnet und ist verantwortlich für die Ausbildung von Lehrkräften aller Fachrichtungen und Schulformen in ganz Hessen.
Du wirst im Regelfall mit dem LSA direkt nur zweimal zu tun haben: Bei deinem „Orientierungspraktikum“ und bei deinem „Betriebspraktikum“, dazu später mehr.
Kontakt mit dem LSA kannst du am besten über deren Homepage (http://lehrerbildung.lsa.hessen.de, dann links auf „Lehramtsstudierende“ ) und per E-Mail an den zuständigen Sachbearbeiter aufnehmen.


SBS-Büro \textit{Büro für Schulpraktische Studien}

Das SPS-Büro der ABL (Akademie für Bildungsforschung und Lehrerbildung) übernimmt innerhalb der Goethe-Universität die Koordination der sogenannten Schulpraktischen Studien (SPS), auch hierzu später mehr. Du findest das SPS-Büro momentan im Juridicum auf dem Campus Bockenheim im 10. OG.
Du meldest dich beim SPS-Büro für die Schulpraktischen Studien an und im Gegensatz zu allen anderen Ämtern an der Universität muss dem SPS-Büro eine Änderung deines Studiums explizit schriftlich mitgeteilt werden (z.B.: bei einem Fachwechsel). Die Anmeldefristen für die SPS findest du auf der Homepage des SPS-Büros (www.abl.uni-frankfurt.de/40729270/Schulpraktische-Studien ). Diese Fristen sind absolut verbindlich und es gibt keinen Spielraum – also hier besonders genau sein!


ZPL – \textit{Zentrales Prüfungsamt für Lehramtsstudiengänge}
Das ZPL befindet sich momentan am Campus Bockenheim im Juridicum. Es ist dafür verantwortlich die von dir abgelegten Prüfungen (abgeschlossene Module) zu registrieren und die Zwischenprüfung, sowie die Meldung zur Ersten Staatsprüfung zu verwalten.
Zu Beginn deines Studiums meldest du dich zur Zwischenprüfung beim ZPL an (Informationen bekommst du entweder auf der Homepage des ZPL, oder bei einer Einführungsveranstaltung). Du legst die Zwischenprüfung automatisch ab, wenn du in vollständig abgeschlossenen Modulen insgesamt 90 CP beim ZPL in Form von kopierten (!) Modulscheinen eingereicht hast.
Die Zwischenprüfung kannst du ablegen, sobald du die benötigten CP gesammelt hast (hierbei gibt es genauere Auflagen, die du der Studienordnung entnehmen kannst), sie sollte allerdings spätestens zwei Semester vor der Meldung zur Ersten Staatsprüfung eingereicht werden, damit eventuelle formale Probleme behoben werden können.\\

Neben diesen Ämtern möchten wir dir einige Informationen zu einigen wichtigen Begriffen und Punkten im Lehramtsstudium geben.

\paragraph{Die Schulpraktischen Studien (SPS)}
Die SPS sind der praktische Teil der universitären Lehramtsausbildung. Du wirst in deinem Studium zwei SPS-Veranstaltungen absolvieren, eine bildungswissenschaftliche und eine fachwissenschaftliche. Die SPS setzen sich in der Regel aus einem Vorbereitungsseminar in einem Semester, einem 5- wöchigen Schulpraktikum in den Semesterferien und einer Nachbereitungsveranstaltung im folgenden Semester zusammen.

\paragraph{Modulscheine}
Im Gegensatz zu den meisten anderen Studiengängen ist das Lehramtsstudium noch nicht vollständig digitalisiert. Du kannst dir auf der Seite des ZPL sog. „Modulscheine“ ausdrucken, in welche die Dozenten die Prüfungsnoten eintragen müssen.
In der Regel ist es so, dass zu einem Modul mehrere Veranstaltungen gehören, in diesem Fall ist es empfehlenswert in der zweiten Veranstaltung nur eine Kopie des bereits zum Teil ausgefüllten Scheins abzugeben, denn ein Verlust ist für dich mit erheblicher Arbeit verbunden.

\paragraph{Die fachspezifischen Anhänge und die Studienordnung}
Die Studienordnung für Lehramtsstudiengänge setzt sich aus der
"Studienordnung" an sich und den "fachspezifischen Anhängen" zusammen. Die Studienordnung regelt generelle Formalia, wie zum Beispiel den Umfang der Studium, die benötigeten CP und die Abläufe der Ersten Staatsprüfung und der Zwischenprüfung.
Die fachspezifischen Anhänge hingegen beschränken sich auf ein Fach (z.B.: L3 Informatik) und beinhalten eine Übersicht über die zu belegenden Module.
Sowohl die Studienordnung, als auch die fachspezifischen Anhänge finden sich auf der Homepage des ZPL und sind leicht über eine Suchmaschine unter dem Stichwort "fachspezifische Anhänge Uni Frankfurt" zu finden.
\textbf{Zu Beginn des Studiums empfiehlt sich auf jeden Fall eine Lektüre der Studienordnung sowie der fachspezifischen Anhänge, zumindest sollten sie quergelesen werden!}
Innerhalb der spezifischen Anhänge findest du einen Studienverlaufsplan, also einen exemplarischen Plan, in welcher Reihenfolge und in welchem Semester du welche Veranstaltungen belegen kannst. Die Vorgaben sind allerdings nicht verpflichtend, du hast dadurch allerdings den Vorteil, dass keine Überschneidungen entstehen (dürften).

\paragraph{Orientierungspraktikum (OP)}
Das OP muss durch das LSA bestätigt worden sein, bevor du dich für die ersten SPS anmelden kannst. Es muss in einer pädagogischen Institution (oft: Kindergarten, Jugendbetreuung) stattfinden und 5 Wochen (120 Stunden) dauern. Auf der Homepage des LSA befindet sich ein Vordruck für einen Praktikumsbericht, der ausgefüllt werden muss – auf der letzten Seite des Berichts ist ein Formular von dir und vom Betrieb auszufüllen, zudem muss vom Betrieb eine (unformale) Bestätigung ausgestellt werden.
Für das OP können Tätigkeiten aus FSJ oder ähnlichem anerkannt werden, in diesem Fall muss nur eine Bescheinigung des Betriebs beiliegen sowie die letzte Seite ausgefüllt werden. Der Bericht entfällt in diesem Fall.

\paragraph{Betriebspraktikum (BP)}
Das Betriebspraktikum umfasst 8 Wochen bei "branchenüblicher Arbeitszeit" (in Ordnung sind auch zwei mal 4 Wochen, auch in unterschiedlichen Betrieben). Es muss in einem Betrieb stattfinden, der knapp gesagt, nichts mit (pädagogisch-) sozialen Tätigkeiten zu tun hat. Auch hierzu findest du auf der Homepage des LSA einen Vordruck für einen Praktikumsbericht. Es wird eine Bestätigung des Betriebs benötigt.
Als Betriebspraktikum können Nebenjobs in nicht-pädagogischen Betrieben anerkannt werden, die über einen längeren Zeitpunkt gemacht wurden. In diesem Fall benötigst du eine Bestätigung des Betriebs und musst den Bericht dennoch schreiben!
Das Betriebspraktikum muss vor der Meldung zur Ersten Staatsprüfung eingereicht werden.
\spaltenende

\newpage

\subsection{Bachelor-/Masterordnung}

\subsection{Teilzeitstudium}
	
\newpage
\section{Die Uni}
	\subsection{Wo ist was?}
		\subsubsection{Bockenheim}
			\begin{center}
			\texttt{ It's dangerous to go alone! Take this. }
			
			\fbox{
			\includegraphics[width=0.9\linewidth]{bilder/KarteB}}
			\end{center}
		\pagebreak
			Wichtig für die Vorlesungen und Übungen:\\
			\fbox{
			\begin{tabular}{rrl}
				Hörsäle: & H 1 - H 16  & Teil des Hörsaalgebäudes über dem Cafe.\\
				{} & H I - H VI  & Andere Teil des Hörsaalgebäudes, welcher nicht\\
				{} & {} & über dem Cafe ist.\\
				{} & Magnushörsaal & In der Informatik\\
				{} & {} & {}\\
				Seminarräume: & SR9, SR11 & Informatik EG\\
				{} & 307 & Informatik 3.Stock\\
				{} & NM...& Diese Räume sind in der neuen Mensa\\
				{} & \multicolumn{2}{l}{Alle anderen dreistellingen Zahlen sind im Matheturm}\\
			\end{tabular}}
			\ \\\\ 
			Sonstige Interessante Orte:\\
			\fbox{
				\begin{tabular}{rp{11cm}}
					Cafe Struvelpeter: & Hier gib es Getränke und kaltes Essen. Du findest es im\\
					{} & Hörsaalgebäude.\\
					Cafeteria: & Verkauft warme Gerichte und gehört zum Studentwerk.  \\
					{} & Ihr findet die Cafeteria in der neuen Mensa.\\
					Leipziger Straße: & Falls ihr aus gegebenen Anlässen keine Lust mehr auf Mensa essen habt, dann gibt es hier alles was das Herz begehrt(Nicht nur Essen).
				\end{tabular}
			}
		
		\subsubsection{Westend}
			Das Westend ist für euer Studium erst interessant, wenn ihr ein Anwendungsfach der Geisteswissenschaften gewählt habt oder ihr Fragen oder Probleme mit dem HRZ oder der Uni habt. Außerdem ist das Westend ein perfekter Ort für eine Studentensafari. Nirgendwo sonst gibt es einen Lebensraum, wo die Reviere so unterschiedlicher Studenten aufeinander treffen. \\
		
		\subsubsection{Riedberg}
			Genauso wie das Westend, ist der Riedberg erst mit der Wahl eines Anwendungsfaches interessant. Bis auf der Tatsache, dass wir uns von Anfang an dort wohlfühlen.
		\subsubsection{Niederrad und Ginnheim}
			Orte, wo die meisten von uns nie sein werden. In Ginnheim befinden sich die Unisportanlagen und in Niederrad die Medizin.
	
		
	\subsection{QIS/LSF und dessen Funktion}
	\newpage
		\subsection{Goethe-Card + Semesterticket}
		\begin{wrapfigure}[8]{r}{0.48\textwidth}
     \vspace{-5mm}
  \begin{center}
     \includegraphics[scale=0.5]{bilder/goethecard}
  \end{center}
\end{wrapfigure}

Nach der erfolgreichen Immatrikulation an unserer Universität bekommst du im Studien-Service-Center eine schicke Chipkarte mit deinem Foto und einigen bildhaften Logos und Beschriftungen drauf. Möglicherweise erfährst du auch gleich, dass sie \emph{Goethe-Card} heißt und als Studienausweis dient. Warum braucht man überhaupt so etwas, wenn man bereits einen ordinären Ausweis hat?

Der wichtigste Grund ist die Tatsache, dass man damit sofort feststellen kann, ob du zum aktuellen Zeitpunkt an der Goethe-Universität studierst: In diesem Fall wurde das blau aufgetragene Gültigkeitsdatum unten zu diesem Zeitpunkt noch nicht überschritten. 

Falls du jetzt einen Blick auf deinen Studienausweis wirfst, wirst du bemerken, dass dort als Enddatum der letzte Tag des laufenden Semesters steht. Daher wirst du, solange du bei uns bleibst, einmal alle 6 Monate diesen Eintrag updaten müssen. Das geht über einen überall auf dem Uni-Gebiet platzierten Automaten, die \emph{Validierer} genannt werden. Sollte man sein Studium erfolgreich beendet oder abgebrochen haben, wird der Validierer das Enddatum der Gültigkeit nicht ändern.

Auf der Goethe-Card sind außerdem dein Foto, dein Name und auch deine persönliche Identifikationsnummer (\emph{Matrikelnummer}) in der Universität aufgedruckt\footnote{Deine Matrikelnummer ist 7-stellig und entspricht den letzten 7 Ziffern der 12\hbox{-}stelligen Zahl auf der Goethe-Card.}. Diese Angaben helfen nicht nur den Profs, dich eindeutig zu identifizieren, sondern auch dir, deine während der Prüfung vergessene Matrikelnummer schnell zu finden.

Aber das ist noch nicht alles, was du mit der Goethe-Card machen kannst:

\begin{itemize}
	\item Du kannst darauf mittels spezieller Geldautomaten-ähnlichen Geräten \textbf{Geld aufladen}, um damit in der Mensa für das Essen oder auch in der Uni-Bibliothek für die Verwendung des Kopierers zu bezahlen.
	\item Du kannst sie als einen \textbf{digitalen Schlüssel} für die super modernen \textbf{Schließfächer} am Campus Westend verwenden
	\item Du kannst damit \textbf{kostenlos} den \textbf{Palmengarten} besuchen.
	\item Du kannst damit \textbf{Bücher} in der \textbf{Uni-Bibliothek} ausleihen.
	\item Du kannst damit \textbf{kostenlos} bestimmte \textbf{Museen} besuchen.
\end{itemize}

Und - last but not least - du kannst damit \textbf{kostenlos} in allen öffentlichen Verkehrsmitteln außer ICE, IC und EC in Hessen und teilweise in angrenzenden Tarifgebieten fahren! 

Das Semesterticket gilt im Gebiet des RMV, NVV, RMV-VRN-Übergangsgebiet und des VGWS. Auf der nächsten Seite findest du eine Karte mit dem skizzierten Geltungsbereich des Semestertickets (Achtung: Stand 2014). Mehr und aktuellere Infos findest du auf den Seiten des AStA Frankfurt:\\
\url{http://www.asta-frankfurt.de/angebote/geltungsbereich-des-semestertickets}.

Au{\ss}erdem: Der Verlust der Goethe-Card kann recht schmerzlich sein - zur Zeit liegt die Gebühr bei Verlust für die Neuausstellung (im Servicecenter des HRZ am Campus Westend) bei 35 Euro. Ist die Goethe-Card defekt bzw. nicht lesbar, kostet euch der Ersatz natürlich nichts.

Man kann mit der Goethe-Card gerade als Informatiker viel Spa{\ss} haben. Der RFID Chip in der Goethe-Card ist sehr gut dokumentiert, einfach auszulesen und zu beschreiben. Suche hierzu Mifare. 

Weitere Informationen zur Goethe-Card kannst du bei Interesse hier finden:\\
\url{http://www.rz.uni-frankfurt.de/44160530/Goethe-Card}

Ach ja, wenn du lieber an der frischen Luft unterwegs bist: Du darfst nach Registrierung auch die /emph{Call a Bike} Leihräder der Deutschen Bahn nutzen. Für die ersten 45 Minuten jeder Fahrt sogar kostenlos (vereinfacht gesagt). Auch hier - mehr Infos beim AStA:\\
\url{http://asta-frankfurt.de/aktuelles/teil-2-wie-leihe-ich-callbike-aus}
		\includepdf{tabellen/semesterticket}
		\subsubsection{Internet}
		\subsubsection{Softwarelizenzen}
		\subsubsection{Bücherreien}
	\subsection{Aufbau der Unipolitik}
\section{Die Informatik}
	\subsection{Robocup-AG}

\label{fachschaftsarbeit}
\section{Die Fachschaft}
	\subsection{Wer sind wir?}
		\glqq\textit{ Wir befinden uns im Jahre 2015 n.Chr. Die ganze Informatik ist von der Faulheit besetzt... Die ganze Informatik? Nein! Ein von unbeugsamen Freiwilligen bevölkerter Raum hört nicht auf, dem Eindringling  Widerstand zu leisten.} \grqq \\
		Spaß beiseite. Wir sind die aktive Fachschaft der Informatik. Aktiv, weil alle Studenten Teil der Fachschaft sind. \\
		Grundsätzlich kann man aber über uns sagen, dass wir ein versprengter Haufen von verrückten, abgedrehten Nerds, Geeks und anderer Lebensformen in den Augen von Außenstehenden sind, dies aber teilweise auf Vorurteilen  
	beruht. Welche davon stimmen und welche nicht, dürft ihr gerne selber herausfinden. Wir bezeichnen uns aber gerne als normal. 	
	\subsection{Was machen wir}
		Das grundlegende Vorurteil gegenüber eine Informatik-Fachschaft besteht darin, dass wir ständig am Trinken und am Rollenspiele spielen sind. \\
		Damit die Wahrheit ans Licht kommt, müssen wir aber sagen, ja das stimmt teilweise (welcher Teil nun stimmt, bleibt wieder euch überlassen). Aber das ist nur ein kleiner Teil unserer Arbeit.\\
		Der andere große Teil betrifft hauptsächlich den Universitätsalltag.\\ Viel Zeit verbringen wir z.B. mit Gremienarbeit, wo wir derzeit die studentischen Vertreter stellen. Zu diesen Gremien zählen derzeit:
		\begin{itemize}
			\item I-Rat
			\item FBR
			\item Lust-Ausschuss
			\item Prüfungsausschüsse
			\item QSL-Mittelausschuss
			\item FSR
			\item FSK	
		\end{itemize}
		Zu unseren weiteren Aufgaben zählt auch, dass wir anderen Studenten helfen. Dies kann auf unterschiedlichste Weise passieren:
		\begin{itemize}
			\item Bei Fragen zu organisatorischen Themen können wir selber Antworten geben oder auf Personen verweisen, die euch da besser weiter helfen können.
			\item Falls ihr berechtigte Kritik an Vorlesungen und/oder Übungen nicht selbst äußern wollt oder sie ignoriert wird, sind wir dafür da, diese Kritik an die verantwortliche Person weiter zu tragen, wozu uns mehr Möglichkeiten zur Verfügung stehen.
			\item Wir organisieren die OE für Erstsemestler, damit ihr in der ersten Woche nicht vor lauter ungewohnten Sachen erschlagen werdet.
			\item Wir erstellen dieses Heft, damit ihr wichtige Themen nachschlagen könnt.
			\item Wir organisieren Sommerfest und Weihnachtsfeier!
		\end{itemize}
		
	\subsection{Du willst zu uns?}
		\glqq Wir grillen gerade und haben ein wenig zu trinken da. Willst du nicht zu uns stoßen?\grqq { }Bei vielen Fachschaftlern hat diese Masche schon funktioniert. Doch gerüchteweise gibt es unter den Studenten auch immer ein paar ''Rationalisten'' die ''Argumente'' brauchen. Und weil ''Wir sind cool!'' nicht immer ausreicht, sind hier ein paar Kommentare von und über uns:
		\begin{itemize}
			\item Wir sind cool!
			\item Social Engeneering lernen, wo besser als an der Uni?
			\item Die Gerüchteküche sind wir.
			\item Form die Uni nach eurem Bild.
			\item Lern die Professoren kennen.
			\item Mach die Uni zu deinem Zuhause and der Uni.
			\item Lern die Studenten kennen, die das dir dein 2h Problem in 2min lösen.
			\item Werde Student, der die 2h Probleme in 2min löst.
			\item CP für Gremienarbeit.
			\item Das Direktorat fragt dich.
			\item Entscheide für wen Ausnahmen gemacht werden.
			\item Firmenpolitik begegnet dir überall. Lerne die Kunst an der Uni.
			\item Bekomme Schlüssel
			\item Fachschaftsfeiern
		\end{itemize}

	
	
\end{document}