
\subsection{Wer sind wir?}
    \glqq\textit{Wir befinden uns im Jahre \the\year{} n.Chr. Die ganze Informatik ist von der Faulheit besetzt... Die ganze Informatik? Nein! Ein von unbeugsamen Freiwilligen bevölkerter Raum hört nicht auf, dem Eindringling  Widerstand zu leisten.} \grqq

    Spaß beiseite. Wir sind die aktive Fachschaft der Informatik. Aktiv, weil alle Studenten Teil der Fachschaft sind. Die ist n\"amlich als alle Studenten eines Fachbereichs definiert.

    Nochmal, das hei{\ss}t, dass \emph{ihr alle Fachschaft seit}. Aber nur mache werden aktiv und machen was. %Und das sind wir.
    Grundsätzlich kann man über uns sagen, dass wir ein versprengter Haufen von verrückten, abgedrehten Nerds, Geeks und anderer Lebensformen sind, deren Existenz in den Augen von Außenstehenden schwer nachzuvollziehen ist.
    Aber das sind alles sowieso nur Vorurteile, die wir nicht von uns weisen. Welche davon stimmen und welche nicht, dürft ihr gerne selber herausfinden. Wir bezeichnen uns aber gerne als normal.



\subsection{Was machen wir?}
    Es wird gerne behauptet, der Sinn einer Informatik-Fachschaft sei, st\"andig zu trinken und Rollenspiele zu spielen.
    Da wir uns als Wissenschaftler vorurteilsfrei der Wahrheit verschrieben haben, m\"ussen wir sagen: ja, das stimmt teilweise.
    Gem\"utliches Beisammensein ist aber nur ein Teil unserer Arbeit.

    Der andere große Teil betrifft hauptsächlich den Universitätsalltag.
    Studenten sind in viele Prozesse der Entscheidungsfindung an der Uni
    mit eingebunden.
    Zum Beispiel verbringen wir viel Zeit mit Kommissionen und Gremienarbeit als studentische Vertreter.
    Die haben dann die unterschiedlichsten Namen und Aufgaben.
    Die habt ihr ja alle bei der W\"orterbuchattacke am Anfang mitbekommen.
    Aber unser inoffizielles OE Motto ist ''Wir glauben an Wiederholungen!'' also frischen wir euren Cache im Kopf nochmal auf:
    \begin{itemize}
        \item I-Rat - die offizielle Ger\"uchtek\"uche
        \item FBR - wo man zusieht, dass keiner am Fachbereich Mist macht.
        \item LuSt-Ausschuss - weil's Spa{\ss} macht
        \item Prüfungsausschüsse - sind halt wichtig, und so.
        \item QSL-Mittelausschuss - Geld verteilen muss auch sein.
        \item FSR - das sind wir offiziell.
        \item FSK - und das sind wir und alle anderen Fachschaften.
    \end{itemize}
    Au{\ss}erdem sind wir daf\"ur da, dass wir anderen Studenten helfen. Dies kann auf unterschiedlichste Weise passieren:
    \begin{itemize}
        \item Bei Fragen zu organisatorischen Themen können wir selber Antworten geben oder auf Personen verweisen, die euch da besser weiter helfen können.
        \item Falls ihr berechtigte Kritik an Vorlesungen und/oder Übungen nicht selbst äußern wollt oder sie ignoriert wird, sind wir dafür da, diese Kritik an die verantwortliche Person weiter zu tragen, wozu uns mehr Möglichkeiten zur Verfügung stehen.
        \item Wir organisieren die Orientierungsveranstaltung für Erstsemestler, damit ihr in der ersten Woche nicht vor lauter ungewohnten Sachen erschlagen werdet.
        \item Wir erstellen dieses Heft, damit ihr wichtige Themen nachschlagen könnt.
        \item Wir organisieren Sommerfest und Weihnachtsfeier!
    \end{itemize}
    Im Prinzip machen wir also alles\dots\hspace{1cm} Zumindest was Studenten so tun.




\subsection{Du willst zu uns?}
    \glqq Wir grillen gerade und haben ein wenig zu trinken da. Willst du nicht zu uns stoßen?\grqq { }Bei vielen Fachschaftlern hat diese Masche schon funktioniert. Doch gerüchteweise gibt es unter den Studenten auch immer ein paar ''Rationalisten'' die ''Argumente'' brauchen. Und weil ''Wir sind cool!'' nicht immer ausreicht, sind hier ein paar Kommentare von und über uns:
    \begin{itemize}
        \item Wir sind cool!
        \item Social Engeneering lernen, wo besser als an der Uni?
        \item Die Gerüchteküche sind wir.
        \item Form die Uni nach eurem Bild.
        \item Lern die Professoren kennen.
        \item Mach die Uni zu deinem Zuhause and der Uni.
        \item Lern die Studenten kennen, die dir dein Zweistundenproblem in zwei Minuten lösen.
        \item Werde Student, der die Zweistundenprobleme Probleme in zwei Minuten löst.
        \item CP für Gremienarbeit.
        \item Das Direktorat fragt dich!
        \item Entscheide für wen Ausnahmen gemacht werden.
        \item Firmenpolitik begegnet dir überall. Lerne die Kunst an der Uni.
        \item Bekomm Schlüssel.
        \item Fachschaftsfeiern.
    \end{itemize}
    Bleibt nur noch zu sagen, dass ihr keine Besonderen F\"ahigkeiten braucht - die Fachschaft ist ja keine Casting Show.
    Manche von uns reden nicht gerne, andere k\"onnen nicht gut schreiben und sind dabei weder am Trinken noch am Grillen.
    Wie gesagt, wir sind halt nur die Leute die an der Uni rumh\"angen, nur das wir uns auch f\"ur die Uni an sich interessiert haben.

