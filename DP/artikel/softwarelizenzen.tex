\spaltenanfang
Wenn du f\"ur dein Studium Programme f\"ur Windows entwickeln lernen willst,
dir aber als armer Student ein solches Luxusprodukt nicht leisten kannst,
kannst du zur RBI gehen und am Microsoft Imagine Programm
teilnehmen. Denn daf\"ur haben wir extra Lizenzvereinbarungen, so dass diverse
Software f\"ur Studenten kostenlos zur Verf\"ugung gestellt werden kann.
Eines dieser Softwareprodukte ist Windows.

Wer also eine Windowslizenz haben will, muss dazu in die RBI, die ja bekanntlich im Keller der Informatik sitzt,
und dort zu dem Herren in Raum \textbf{014a}, der die Lizenzen verwaltet. Dazu
braucht man einen RBI-Account, den man bekommt, wenn man das Anmeldeformular
korrekt ausgef\"ullt in den Briefkasten vor Raum 011 geworfen hat (au{\ss}er
man war beim Vorkurs, dann hat man den Account schon da bekommen.)

Mehr Infos gibt es auf der RBI-Webseite unter: \texttt{\url{http://www.rbi.informatik.uni-frankfurt.de/RBI/de/service/microsoft-imagine}}


\spaltenende

