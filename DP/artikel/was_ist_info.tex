\spaltenanfang
Du hast bestimmt eine gute Vorstellung davon, was Informatik ist. Immerhin willst du das ja studieren. Aber deine Vorstellung von Informatik wird sich wahrscheinlich in den n\"achsten Semestern deutlich \"andern.
Um das zu verdeutlichen, bedenken wir kurz das folgende Szenario:

Du verbringst viel Zeit vor deinem Computer und bist Experte im Umgang mit den neuesten Produkten aus dem Hause MS? Du kennst dich bestens aus im WWW und streifst Nachts leidenschaftlich von Chatroom zu Chatroom? Als begabter Hardware Tüftler bist du mit deiner selbst übertakteten Grafik Hardware ein Meister aller 3D Spiele und hast in World of Warcraft Rang und Namen? Außerdem ist Mathematik eh viel zu langweilig und überflüssig, braucht doch eh keiner, oder? Als eingefleischter Programmierer reicht es dir eh mal ab und zu den ein oder anderen Schnipsel PHP zu programmieren?

Willkommen in der Informatik, wahrscheinlich bist du hier ... falsch.


Spätestens jetzt solltest du schlecht gelaunt denken:
Was soll der Blödsinn? Informatik hat doch mit Computern, Datenverarbeitung 
und Internet zu tun? Um jemanden zu zitieren (von dem ihr in eurem Studium noch
einiges lernen werdet – nur Geduld): „Computer Science is no more about
computers than astronomy is about telescopes.“ Folgendes werdet ihr also (im
Wesentlichen) nicht im Informatik Studium lernen:

\begin{itemize}
     \item EDV Grundlagen.
     \item Wie kann ich Webseiten gestalten?
     \item Wie werde ich mit Sozialen Netzen reich?
     \item Was ist die beste Hardware?
     \item Wo bekomme ich die her?
     \item Wie baue ich mir damit meinen PC zusammen?
     \item Und wie vernetze ich diesen?
     \item Und wenn ich schon dabei bin: Wie kann ich an der Uni drucken, ins WLAN oder mein Passwort \"andern.
\end{itemize}

Das lernt ihr selbst in eurer Freizeit. Vieles davon wird im Studium sogar vorausgesetzt.

Also, was ist Informatik jetzt wirklich?

Informatik ist eine Wissenschaft. Sie besteht aus Mathematik, Algorithmen,
Datenstrukturen, Technik (aha – also doch?) und ihrer Anwendung. Sie
beschäftigt sich hauptsächlich mit Problemen und (sofern es sie gibt) deren
Lösungen. Wir von der Fachschaft wollen euch natürlich nicht für dumm
verkaufen. Und deshalb seid ihr ja auch an der Uni gekommen. Damit ihr das ganze Drumherum auch noch lernen k\"onnt.

\spaltenende
