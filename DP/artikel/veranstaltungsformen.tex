\spaltenanfang
Selbst in unserem skurrilen Pseudorollenspiel, in das wir mit dem Streben nach dem Bachelor verwickelt werden, existieren, wie in einem ''echten'' und ''coolen'' Rollenspiel, verschiedene Prüfungen, die zu bewältigen sind. Leider sind die Prüfungen im Studium wesentlich echter und weniger cool. Anstatt Drachen zu töten um mächtige Amulette zu erringen, ist das in der Uni ''etwas'' unspektakulärer. Trotzdem behaupten wir: Die Bewältigung beider Arten von Aufgaben erzeugen aber das selbe Maß an Euphorie! *hust* 
Kommen wir nun zu den Veranstaltungsformen.

\textbf{Vorlesung: } Veranstaltungen solcher Art werden von geübten Studenten
zum Verschlafen oder Einschlafen verwendet. Wie kommt es dazu? Vorlesungen
haben einen oder mehrere wöchentlich stattfindene Termine, die meist ein- bis
zweistündig sind. Dort sitzen bis zu 300 Studenten in einem Hörsaal und
lauschen den weisen Worten eines Dozenten. In der Informatik sind, nicht wie in
anderen Studiengängen, die Vorlesungen ohne Anwesenheitspflicht. Liegen
Vorlesungen in einem Zeitraum zwischen 8 Uhr und 12 Uhr, so ist die Anzahl der
Besucher der Vorlesungen eher gering (siehe Anfang des Absatzes). Solche
Veranstaltungen, so unangenehm es auch sein mag, sich da hin zu quälen,
erleichtern das Lernen. Nichts ist besser als eine Veranstaltung zu besuchen,
in der ein komischer Kauz dir alles erzählt, was du selber lernen sollst.
Außerdem steht dir die Freiheit zur Verfügung, ihm Fragen zu stellen und ihn
sogar zu korrigieren, was eine Menge Spaß macht. Ist in einem Modul eine
Vorlesung als Veranstaltung vorhanden, schließt man das Modul meistens mit
einer Klausur als Prüfung ab. Manchmal gibt es auch eine Alternative zur
Klausur: Die mündliche Prüfung. Da sitzt man dann alleine mit dem Dozenten der
Vorlesung in einem Raum und er fragt dir Löcher in den Bauch. Bevor ihr diese
Gelegenheit wahrnehmt, informiert euch bei Höhersemestrigen, ob man damit nicht
Suizid begeht.

\textbf{Übungen und Tutorien: }
In anderen Studiengängen wird zwischen Übungen und Tutorien unterschieden. Nicht hier in der Informatik. Eine Übung ist nichts anderes als eine meist wöchentlich stattfindene Veranstaltung in der sogenannte Übungsaufgaben besprochen werden. Die Studenten werden ebenfalls von einem ihres gleichen, der das Modul, zu dem die Übung gehört selbst schon erfolgreich abgeschlossen hat, betreut. Ebenfalls hat man hier wieder den Spaß, Fragen stellen zu dürfen und zu klugscheißen. Übungen und Vorlesungen gehen meistens Hand in Hand. Die Übungsaufgaben sind mit dem bisher erlangten Wissen durch die Vorlesung zu bearbeiten und meist eine Woche später abzugeben. Zwei Mathe Module beinhalten eine Studienleistung, die besagt, dass man mindestens 50\% der Übungspunkte erreicht haben muss, um an der Abschlussprüfung teilnehmen zu dürfen. So ein Scheiß. Aber seht's mal positiv! So viele Übungspunkte nach hause geholt zu haben bedeutet auch, sich etwas für die Prüfung vorbereitet zu haben. Andere Module, wie z.B. HW2, setzen nur eine Studienleistung vorraus, nämlich genau die 50\% der Übungspunkte. Ihr lest richtig: Geschenkte CP.

\textbf{Praktikum: }
Ein Praktikum ist die Veranstaltung, in der man praktisch arbeitet. Kann man wohl aus dem Namen rauslesen. Je nach Praktikum hat man aber mehr oder weniger Aufgaben. Einige Praktika verlangen einfach knallhart, knallharte Aufgaben zu bearbeiten, andere verlangen noch zusätzlich Protokolle. Praktika können auch mit anderen Veranstaltungen und Prüfungen gekoppelt sein. Manche WiMis und Dozenten sind wahnsinnig einfallsreich und kreativ was das angeht. Im Modulkatalog, der auch in den Spielregeln (Die Bachelorordnung) enthalten ist, findest du genau diese Information. 
%Also kleiner Ratschlag: Wenn ihr auf Überraschungsprüfung nicht so steht, dann lunst mal da rein.

\textbf{Seminar: }
Du hast Lust, selbständig zu arbeiten und zu recherchieren? Du möchtest das Selbsterlernte als schriftliche Ausarbeitung bewerten lassen? Du möchtest deine Ergebnisse vor anderen Studenten, wissenschaftlichen Mitarbeitern und Professoren in einer Präsentation vorstellen und hast Lust auf eine anschließende Diskussionrunde? Dann ist ein Seminar genau das Richtige für dich! Falls das überhaupt nicht dein Ding ist, dann hast du gelitten, denn du musst mindestens ein Seminar in den Vertiefungsmodulen bestanden haben. Aber ein Seminar kann man schon überstehen und wenn man die alten Hasen im Bachelor nach Seminaren ausquetscht, sind diese meistens Seminaren gegenüber sehr positiv gestimmt.
\spaltenende
