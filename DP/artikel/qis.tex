Das QIS ist DAS Internetportal-Dingens, um Krempel zu erledigen.
Um welchen Krempel handelt es sich? Das erfahrt ihr jetzt:

\textbf{1. Vorlesungsverzeichnis} (Veranstaltung $\rightarrow$ Vorlesungsverzeichnis)
Das Vorlesungsverzeichnis beinhaltet alle Veranstaltungen, die in dem OBEN RECHTS ausgewählten
Semester angeboten werden und ist, so lange man dort nicht rumpfuscht, auf das aktuelle Semester eingestellt.
Benutzt wird es 1. wenn man nachschauen möchte, was man so in diesem oder nächsten Semester besuchen möchte oder
2. wenn man keine Ahnung hat, wann und wo man nochmal hin soll.

\textbf{2. Prüfungsanmeldung} (Meine Funktion $\rightarrow$ Prüfungsverwaltung)
Habt ihr eure TAN-Liste, die ihr mit eurer GoetheCard ausgeteilt bekommen habt, weggeschmissen?
Wenn ja, dann habt ihr Pech, denn um die Online-Annmeldung über das QIS-Portal durchzuführen,
braucht ihr genau diese. Ihr klickt euch von eurem Fachbereich über euer Studium zur Prüfung hindurch, für die ihr euch anmelden wollt,
und wählt dann den gewünschten Termin aus, haut die TAN-Liste auf den Tisch und gebt den sechsstelligen Pin ein, nach dem gefragt wird.

\textbf{3. Notenspiegel und bestandene Module} (Meine Funktionen $\rightarrow$ Prüfungsverwaltung)
Klickt auf Notenspielgel und dann auf Info. Der Rest erklährt sich von selbst.

Fehlt nur noch, wo ihr dieses Wunderportal findet:\\
\begin{center}\large\textbf{\url{http://qis.server.uni-frankfurt.de}}\end{center}
