\spaltenanfang
\paragraph{Deine Fachschaft}~\\
Findest du oft im Fachschaftsraum hinter dem Lernzentrum.
Du musst nur durch einen der beiden Glasbr\"ucken.\\
\begin{center}
\url{http://fs.cs.uni-frankfurt.de/}\\
\emailfachschaft
\end{center}

\paragraph{Dein Fachbereich}~\\
Fachbereich Informatik und Mathematik (12)\\
Institut für Informatik\\
Robert-Mayer-Straße 11–15\\
60054 Frankfurt\\

\begin{tabular}{ll}
Institutssekretariat & 798 – 23325 \\
Informatik-Bibliothek: & 798 – 22287\\
Informatik-Prüfungsamt: & 798 – 28279\\
Informatik-Studienberatung: & \\
~~~Prof. G. Schnitger & 798 – 28326
%~~~und die Fachschaft & 798 – 23933\\
\end{tabular}

\paragraph{Informationssystem}~\\
Verwaltung Deiner persönlichen Daten rund ums Studium:\\
\begin{center}\url{http://qis.server.uni-frankfurt.de}\end{center}

%\begin{center}\url{http://go.uni-frankfurt.de}\end{center}
%Vorlesungsverzeichnis:



\paragraph{Der AStA}~\\
Der AStA ist die zentrale Studentenvertretung. Siehe auch Wörterbuch $ \rightarrow $ AStA.
Der ernennt verschiedene ''Referate'' die sich um spezielle Belange von Studierenden k\"ummmern.
\begin{center}\url{http://asta-frankfurt.de}\end{center}


\paragraph{AStA-Büro}~\\
Studierendenhaus, Mertonstr.\\
26–28, Raum B1\\
Telefon: 798 – 23181\\
Mo,Di,Do,Fr: 9:30 -- 13:00 Uhr\\
Mo,Di,Do: 13:30 -- 15:00 Uhr


\paragraph{Rechts- und BaföG-Beratung}~\\
Studierendenhaus Raum B7\\
798 – 23175 oder 798-33098\\
Bockenheim:\\
Montag: 10.15 - 11.00 Uhr \\  
Freitag: 15.30 - 16.15 Uhr \\
Beratungscenter AStA Haus Campus Westend:\\
Dienstags 16.00 - 16.45 Uhr \\
Mittwochs 9.15 - 10.00 \\
Donnerstags 17.45 - 18.30 Uhr \\

\paragraph{Sozialreferat}~\\
Studierendenhaus Raum B7\\
Beratung: Mo 12:00 -- 13:00 Uhr\\
Mi 12:00 -- 13:00 Uhr\\
Fr 13:00 -- 14:00 Uhr

\paragraph{Nightline Frankfurt}~\\
Die Nightline Frankfurt ist ein anonymes Zuhörtelefon von Studierenden für Studierende.
Montags, Mittwochs und Freitags ist die Nightline von 20 bis 00 Uhr unter 069/798 17 238 für alle erreichbar, die sich etwas von der Seele reden möchten. 


\paragraph{BAföG}~\\
Wer wieviel BAföG bekommt, können wir hier natürlich nicht schreiben, aber durch die letzten Gesetzesänderungen lohnt sich eigentlich für Jeden der Weg ins BAföG-Amt.

\paragraph{Amt für Ausbildungsförderung}~\\
Neue Mensa, 4. Stock, Vorzimmer 410\\
Bockenheimer Landstraße 133\\
60325 Frankfurt\\
Sprechzeiten: Mo Di Fr 10:00 - 12:00 Uhr, Mo Di Mi Do 13:00 - 15:00 Uhr
Telefonsprechzeiten: Mo-Fr 8:00-10:00 Uhr\\

\paragraph{Wohnen}~\\
Studentenwerk Frankfurt/Wohnheimabteilung\\
Bockenheimer Landstr. 133\\
Raum 319 u. 320\\
60325 Frankfurt\\
Telefon: 798 – 23021\\
%\emailwohnen

\spaltenende
