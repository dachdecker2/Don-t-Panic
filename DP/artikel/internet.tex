Ach ja, das Internet von dem alle immer reden...\\

\spaltenanfang
Eigendlich sollte ja alles ganz einfach sein.
Die Uni war schon am Internet bevor es das WWW gab und sollte deshalb mehr Erfahrung als der Rest der Welt haben.
Aber was stattdessen passiert ist, ist dass das Netz an der Uni gewachsen ist und inzwischen die Komplexit\"at eines Lebewesens gepaart mit der Zuverl\"assigkeit eines Informatikstudentens morgens um 08:00 s.t. erreicht hat. Besonders in der Informatik, wo jeder einen eigen Webserver betreiben kann, betreibt jeder einen eigenen Webserver. Das Resultat ist eine uneinheitliche Mischung, in der jede Veranstaltung eigene Rituale hat, \"Ubungsanmeldungen \"uber das Internet abzuwicklen oder die Folien der Vorlesung zu ver\"offentlichen. Ausserdem ist die Netzwerkinfrastruktur uneinheitlich und die unterschiedlichen WLAN SSIDs, mit denen man sich verbinden kann, kommen einem wie ein Dschungel aus elektromagnetischer Strahlung vor. Sich dann noch Passworte f\"ur Vorlesungen, unterschiedliche Uni-Mailadressen und all die verschiedenen Accounts zu merken macht die Verwirrung komplett.  Und damit wurde das Internet zu dem was es heute ist: \textbf{Dem ersten Endgegner der Informatik}.

\textbf{Eure Accounts:}
Grunds\"atzlich bekommt jeder Student an der Uni Frankfurt einen Account vom \textbf{HRZ}, dem \textit{Hochschul RechenZentrum}\footnote{\url{http://www.rz.uni-frankfurt.de}}.
\"Uber diesen Account k\"onnt ihr euch zu Klausuren und Veranstaltungen anmelden, eure Studiendaten abfragen, E-Mails schreiben und lesen und habt einen WLAN-Zugriff \"uberall da an der Uni, wo WLAN grade funktioniert. Als eingeschriebene Studenten habt ihr eure Zugangsdaten und eine TAN Liste zugeschickt bekommen. 

Aber als Informatiker habt ihr auch die M\"oglichkeit euch einen Account im speziellen Informatikrechenzentrum, dem \textbf{RBI} (\textit{RechnerBetrieb Informatik})\footnote{\url{http://www.rbi.cs.uni-frankfurt.de}} anlegen zu lassen. Da k\"onnt ihr auch E-Mails benutzen, Windows- und Softwareentwicklungs Lizenzen bekommen und k\"onnt sogar bis zu 500 Seiten im Semester kostenlos drucken. Au{\ss}erdem stehen in der Informatik Computer, die ihr nur mit einem RBI-Account benutzen k\"onnt. Wer beim Vorkurs war, kennt das ja schon.

F\"ur manche Veranstaltungen bekommt ihr eventuell noch extra Accounts. Zum Beispiel, wenn ihr Hochleistungscomputer programmieren sollt, kann euch der Veranstalter erlauben mit seinem Spielzeug zu spielen und gibt euch extra Accounts. Aber normalerweise braucht ihr erstmal nur die beiden Zug\"angen der beiden oben genannten Rechenzentren aus.

\textbf{WLAN:}
Sich an der Uni mit dem WLAN zu verbinden ist im Prinzip einfach.
Trotzdem sollte man vorher alle Rituale, die g\"ottliches Wohlwollen hervorrufen mit h\"ochster Sorgfalt abhalten,
denn die WLAN Ausleuchtung der Uni ist nicht \"uberall gew\"ahrleistet.

Es gibt uniweit drei unterschiedliche Netze.\\
\textbf{FREIFLUG} ist das langsamste (und unsicherste) Netzwerk, aber daf\"ur kommt man auch fast immer rein. Dazu wirst du erst auf eine interne Uniseite umgeleitet und musst dich da mit deinen HRZ Daten anmelden, bevor du ins Internet kommst. Das ist dann unversch\"usselt, was in der Informatik nicht zu emfehlen ist.\\
FLUGHAFEN ist ein verschl\"usseltes uniweites Netzwerk, aber nicht so gut wie:\\
\textbf{eduroam} ist ein Netwerk, mit dem du als Student sogar an vielen anderen Universit\"aten weltweit ins Internet kommst (Achtung, damit dies auch weltweit funktioniert, muss der Login wie folgt sein: \textit{<deinHrzAccountName>@uni-frankfurt.de} - auch wenn Du diese Email-Adresse nicht hast!).
Es ist ein 802.1x Netz f\"ur das du aber vorher ein Zertifikat\footnote{\url{https://www.pki.dfn.de/fileadmin/PKI/zertifikate/deutsche-telekom-root-ca-2.der}} brauchst (ist auf machen Systemen (auch mobil) schon vorinstalliert).\\
Schlie{\ss}lich gibt es noch WLAN-RBI-VPN von der RBI, mit dem du dich \"uber OpenVPN verbinden kannst.

Wie du dich genau mit dem WLAN verbindest kannst du auf den Seiten des jeweiligen Rechenzentrums nachlesen und ist von deinem Betriebssystem abh\"angig. Aber wer braucht als Informatiker schon Internet, Papier und Bleistifte sind wichtiger als Computer.






\spaltenende
