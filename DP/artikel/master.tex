\spaltenanfang
Nachdem du es gepackt hast, einen Bachelor mit einer Note von nicht schlechter als 3.0 zu erwerben, kannst du dich für den Masterstudiengang an der Goethe-Universität Frankfurt entscheiden.
Die meisten Masterstudenten haben ja schon einmal gezeigt, dass sie
erfolgreich studieren k\"onnen. Aber da es ja mal hie{\ss} „Man bekommt sein Diplom
daf\"ur, dass man die Studienordnung gelesen hat.“ geben wir nochmal einen
kurzen \"Uberblick.

\textbf{Schwerpunkte:}
Zun\"achst sollte beachtet werden, dass es so einen Master bei uns in vier neuen unterschiedlichen Geschmacksrichtungen gibt. Aber statt alle zu kosten, muss man sich f\"ur einen der sogenannten \emph{Schwerpunkte} entscheiden:
\begin{noindEnumerate}
\item Master mit Anwendungsfach
\item Master mit vertieftem Anwendungsfach
\item Master in allgemeiner Informatik
\item Master mit Spezialisierung
\end{noindEnumerate}

Wenn du im Bachelor kein Anwendungsfach studiert hast, \textbf{musst} du den Master mit Anwendungsfach machen.

Wenn du schon ein Anwendungsfach hattest, kannst du dich entweder im selben Fach weiter vertiefen, oder ein ganz neues Anwendungsfach belegen.

Wer der Meinung ist, dass alles au{\ss}er Info doof ist, kann den Master in allgemeiner Informatik oder den Master mit Spezialisierung machen. 

\textbf{Auflagen:}
Wer zwischen Bachelor und Master die Uni wechselt, kann f\"ur das Masterstudium Auflagen bekommen.
Das sind Veranstaltungen mit Grundlagen, die vorrausgesetzt wurden, die man im Bachelor aber nicht nachweisen konnte.
Das hei{\ss}t im Klartext, dass bis zu 30CP an Modulen, die euch dann gesagt werden, in den ersten 14 Monaten gemacht werden m\"ussen, ohne, dass sie angerechnet werden. Aber das sollte kein Problem sein, das sind immerhin Grundlagen, und oft nur ein oder zwei Veranstaltungen, die ihr dann machen m\"usst. Wenn ihr den Bachelor in Frankfurt gemacht habt, wird euch das sowieso nicht passieren.

\textbf{Aufbau:}
Insgesamt geht das Masterstudium über vier Semester und besteht aus 120 CP.
Egal für welche der vier Formen des Masterstudiums du dich entscheidest, am Ende musst du bei allen im Abschlussmodul eine Masterarbeit schreiben, die 30 CP bringt und gewöhnlich über ein Semester geht (6 Monate).
Die Informatikmodule sind in drei Gebiete aufgeteilt: „Informatik der Systeme“, „Grundlagen der Informatik“ und „Angewandte Informatik“. Je nach gewähltem Schwerpunkt musst du aus jedem der Gebiete eine bestimmte Anzahl an CP erbringen. Au{\ss}erdem muss man ein Seminar und ein Praktikum machen. Wer allgemeine Informatik studiert, macht hier 84-87CP, alle anderen nur 60-63CP.
Wer sich f\"ur ein Anwendungsfach oder die Spezialisierung entschieden hat, muss darin 24CP machen.
Zusätzlich sind in jedem Schwerpunkt Ergänzungsmodule zu belegen, die 3 bis 6~CP bringen und meistens durch Soft-Skill Workshops erreicht werden.

\textbf{Masterordnung:} Wer mehr wissen will, sollte sich die Masterordnung anschauen.\footnote{siehe \url{http://www.cs.uni-frankfurt.de/images/pdf/informatik/master2015/ordnung_master_2015.pdf}} Das sieht wie immer nach mehr aus, als es ist. Haltet einfach nach bunten Bildern Ausschau.

\spaltenende
