%\includegraphics[width=\linewidth]{comics/FuturamaComics/Comicsfinal/jointheFachschaftfull}
Schaut man in das hessische Hochschulgesetz (HHG), so erfährt man, dass sich „die Fachschaft aus allen Studierenden eines Fachbereichs“ zusammensetzt. Fragt man auf dem Campus rum, so erfährt man vielleicht, dass „Fachschaftler“ Leute sind, die an der Uni (politisch) aktiv sind. Aber was macht so ein „Fachschaftler“ denn nun eigentlich genau? Fangen wir doch einfach mal mit dem an, was du gerade in der Hand hältst:

\begin{itemize}
 \item Irgend jemand hat diese Ausgabe der „Don‘t Panic!“ erstellt, Ideen entwickelt, Bilder gesetzt und Texte geschrieben. Das waren Fachschaftler.

 \item Irgend jemand hat Freiwillige als OE-Tutoren rekrutiert, Frühstück besorgt, sich um Sponsoren bemüht und diese ganze Orientierungseinheit organisiert. Das waren Fachschaftler.
 
 \item Irgend jemand bereitet (teils mit der Hilfe der Professoren) Sommer-, Weihnachts-, Halloween- und sonstige Feste vor und sorgt dafür, dass man Essen und Trinken kaufen kann. Das sind Fachschaftler.

 \item Irgend jemand versucht tagtäglich, deine Interessen gegenüber Professoren und anderen Bösewichten durchzusetzen. Das sind Fachschaftler.

 \item Irgend jemand ist für dich da, wenn es in Veranstaltungen beschissen läuft, unfaire Bedingungen vorhanden sind, du Fragen zu Studium, Prüfungsmodalitäten oder diversen Veranstaltungen hast. Das sind Fachschaftler.

 \item Irgend jemand leiht dir ein Ohr, wenn du eine Studienberatung willst. Das sind Fachschaftler.

 \item Irgend jemand antwortet dir, wenn du Fragen an die E-Mail-Adresse \emailfachschaft schickst. Das sind Fachschaftler.
\end{itemize}

Okay, damit wäre schonmal grob umrissen, was wir Fachschaftler so tun. Aber was heißt das denn nun genau?


Nehmen wir doch mal das Beispiel „nicht geregeltes Anwendungsfach“. Auch wenn Du hoffentlich bald eine recht breite Auswahl an Anwendungsfächern haben wirst, möchtest du vielleicht doch etwas anderes machen. Egal, ob du nun erst zum Prüfungsamt gehst, um dir die Prozedur erklären zu lassen, oder gleich zu uns kommst, früher oder später landet dein Antrag in einem Gremium, genannt „Institutsrat“ oder auch „Direktorium“. In diesem Gremium werden alle informatikspezifischen Angelegenheiten von Professoren, wissenschaftlichen, administrativ-technischen Mitarbeitern und Studierenden, also Fachschaftlern, diskutiert und beschlossen. Im Zweifelsfall erklären also deine Fachschaftler hier den Professoren, warum deine Wahl eines Anwendungsfachs sinnvoll ist.


Ein anderes Beispiel wäre der Lehr- und Studienausschuss (kurz LuSt). In diesem Gremium wird unter anderem das Lehrangebot erstellt. Auch hier sitzen Professoren mit deinen Fachschaftlern zusammen und achten darauf, dass die Professoren auch ihre Lehrverpflichtung erfüllen, dass in den einzelnen Bereichen genug Lehre vorhanden ist, dass es genug Praktika und Seminarplätze gibt, alles in allem also, dass dein Studium halbwegs studierbar bleibt. Fachschaftler schauen sich auch immer wieder um, was außerhalb so alles los ist, und machen Vorschläge, welche Externen man dazu einladen könnte, um bei uns eine Veranstaltung durchzuführen.


Fachschaftler sind es auch, die neue Entwürfe für die Bachelor-Ordnung durchsehen und kommentieren. Denn man muss sowas einfach aus Studierendensicht durchgehen, um zu merken, dass das Ding so nicht studierbar ist.


%\renewcommand{\labelitemi}{\textendash}

Und nicht zu vergessen, sind es auch Fachschaftler, die in den ganzen Gremien deine Beschwerden weitergeben und deine Wünsche vertreten. Wie du siehst, gibt es für Fachschaftler an allen Ecken und Enden etwas zu tun, und um all das zu tun, brauchen wir DICH!

\begin{itemize}

\item Wir brauchen DICH, um uns Bescheid zu geben, wenn in einer Veranstaltung etwas nicht so läuft, wie es sollte.

\item Wir brauchen DICH im nächsten Jahr, um als OE-Tutor die neuen Erstsemester zu betreuen.

\item Wir brauchen DICH, damit du im kommenden Frühjahr zur Wahl gehst.
\item Und wir brauchen DICH, damit einfach mal bei unseren Gremientreffs vorbeikommst, und mal schaust, ob du uns nicht unterstützen willst.
\end{itemize}

Also, wenn du Lust hast, dann schau doch einfach mal donnerstags gegen 16:00 Uhr bei uns vorbei. Und wenn du Fragen hast oder es Probleme gibt, dann zögere nicht, uns anzusprechen. Wenn im neuen Fachschaftsraum (direkt hinter dem Lernzentrum) niemand anzutreffen ist, dann vielleicht in der Bibliothek. Oder einfach per E-Mail: \emailfachschaft

\renewcommand{\labelitemi}{$\bullet$}

\begin{flushright} Alexander \end{flushright}


