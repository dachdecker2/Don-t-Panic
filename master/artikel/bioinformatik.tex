Nun seid ihr den ersten Schritt gegangen, und habt euch für den Bachelor Bioinformatik Studiengang an der Johann Wolfgang Goethe Universität eingeschrieben. Damit ihr euch nach den ersten zaghaften Blicken in der Bachelor Ordnung nicht wieder ausschreibt, werde ich versuchen, euch diesen Ordnungswulst zusammengefasst und verständlich wiederzugeben. Es sei allerdings angemerkt, dass Änderungen in der Ordnung noch gemacht werden könnten, die euch betreffen können. Zunächst einmal gilt für euren Bachelor Studiengang, wie für alle Anderen auch, dass er modular aufgebaut ist, und in 6 Semestern mit der Bachelorarbeit zum Ziel führt. Ein Modul umfasst im Allgemeinen eine oder mehrere Lehrveranstaltungen. Es kann außerdem sein, dass zur Anmeldung einer Modulabschlussprüfung bestimmte Auflagen (z.B. der Abschluss eines anderen Moduls, oder der Nachweis einer Studienleistung) erfüllt werden müssen. Weil all dieses noch viel zu trivial ist, kommt noch hinzu, dass Prüfungen von Informatikveranstaltungen nach der Informatik-Prüfungsordnung, und Prüfungen von Bioveranstaltungen von der Biologie-Prüfungsordnung geregelt sind.

Zu der Prüfungsordnung der Informatik zählen: Programmierung 1 (Studienleistung), Programmierung 2 (Studienleistung), Modulabschlussprüfung Programmierung (Thematisch: Programmierung 1 \& 2), Grundlagen der Programmierung für Bioinformatiker Praktikum, Algorithmentheorie, Datenstrukturen, Algorithmen und Modelle der Bioinformatik, Diskrete Modellierung, Analysis und lineare Algebra, Mathe 2 - alternativ Stochastik und Numerik.

Für diese aufgezählten Module gilt: Ihr habt 3 Versuche, wobei der letzte Versuch mündlich sein sollte. Solltet ihr hiernach durchfallen, seit ihr durch den Bachelor Bioinformatik final durchgefallen und dürft in ganz Deutschland keinen Bioinformatik oder Bioinformatik-nahen Studiengang an einer deutschen Hochschule mehr studieren. Es empfiehlt sich also, sich nur dann anzumelden, wenn man auch der Meinung ist, man könne die Prüfung auch bestehen. Eine Ausnahme bilden jedoch Numerik, Mathe 2 und Stochastik. Solltet ihr z.B. in Numerik final durchfallen, und in Stochastik noch nicht eure finale Prüfung hinter euch haben, so seit ihr nicht durch den Bachelor in Bioinformatik durchgefallen, ihr habt jedoch keine Möglichkeit mehr euch in Numerik prüfen zu lassen, und müsst nun Stochastik oder Mathe 2 bestehen. Das Gleiche gilt natürlich auch umgekehrt. Wenn ihr jedoch durch alle 3 durchfallt, seid ihr durch den Bachelor final durchgefallen. Wenn Ihr durch eine Modulabschlussprüfung durchfallen solltet, habt ihr 14 Monate Zeit diese zu wiederholen. Bei den Modulen: Modulabschlussprüfung in Programmierung, Algorithmentheorie, Datenstrukturen, Modellierung, Analysis und lineare Algebra, Stochastik und Numerik handelt es sich um so genannte „Basismodule“. Die Informatik Prüfungsordnung schreibt vor, dass diese Module jedes Semester angeboten werden müssen. Es heißt jedoch nicht, dass hierfür extra nochmal eine Vorlesung stattfinden muss. Der durchgenommene „Stoff“ kann sich also durchaus auch auf das Semester davor beziehen. Für Studienleistungen gilt: Ihr habt soviel Versuche wie ihr wollt. Es gibt keine Limitierungen.

%\includegraphics[width=\linewidth]{comics/FuturamaComics/Comicsfinal/Bioinfocpbafg1}

Zu der Biologie Prüfungsordnung zählen: Struktur und Funktion der Organismen, Struktur und Funktion der Organismen Praktikum, Organische Chemie, Biochemie, Mikrobiologie und Pflanzenphysiologie, Organische Chemie Praktikum, Zellbiologie, Neurobiologie, Molekularbiologie und Genetik, Spezialisierung I, Strukturelle Bioinformatik, Grundlagen der Bioinformatik, Spezialisierung II.

%\includegraphics[width=\linewidth]{comics/FuturamaComics/Comicsfinal/Bioinfocpbafg2}
%\hspace *{1cm}

Für diese aufgezählten Module gilt: Ihr habt auch 3 Versuche. Der letzte Versuch muss jedoch nicht mündlich sein. Wenn er schriftlich ist, wird die Klausur im Falle einer finalen Prüfung von 2 Prüfenden bewertet. Wenn ihr durch eine Modulabschlussprüfung durchfallt müsst ihr diese Prüfung zum nächsten Prüfungstermin wiederholen.

%\includegraphics[width=\linewidth]{comics/FuturamaComics/Comicsfinal/Bioinfocpbafg3}
%\hspace *{1cm}

 Sollte die Prüfung Zulassung für ein Modul im folgenden Semester sein, wird vom Fachbereich Biowissenschaften eine Wiederholungsmöglichkeit vor Beginn des jeweiligen Semesters angeboten.
%\hspace *{1cm}

Solltet ihr bei einer kumulativen Modulprüfung (z.B. in Mikrobiologie und Pflanzenphysiologie) in einer Teilprüfung durchfallen (z.B. Mikrobiologie), so müsst ihr nur die Teilprüfung wiederholen, durch die ihr durchgefallen seid (in unserem Fall Mikrobiologie). Die Gesamtmodulabschlussnote ergibt sich aus dem addierten Mittelwert beider Teilprüfungen.

Eine Ausnahme bilden: Präsentationstechniken, Teammanagement und Führungskompetenz, Wahlpflichtmodul und Abschlussmodul (die Bachelorarbeit). Präsentationstechniken, Teammanagement und Führungskompetenz und das Abschlussmodul können entweder im Bereich der Informatik oder im Bereich der Biologie absolviert werden. Bei dem Wahlpflichtmodul habt ihr sogar noch eine größere Auswahl. Hier könnt ihr frei aus den Veranstaltungen von Biowissenschaften, Informatik und Mathematik, Biochemie, Pharmazie und Chemie oder Physik wählen. Solltet ihr nur ein Fach wählen, müsst ihr 9 CP erbringen. Ihr könnt auch 2 Fächer wählen, jedoch müsst ihr dann jeweils mindestens 6 CP erbringen. Wenn ihr in eurem Wahlpflichtmodul eine Prüfung ablegt, und darauf eine Note bekommt, fließt diese nicht in die Notenberechnung eures Bachelor Zeugnisses ein, ihr könnt jedoch auf Antrag beim Prüfungsausschuss Bioinformatik die Note mit in die Bachelor Urkunde aufnehmen lassen.
%\hspace *{1cm}

Für Teammanagement und Führungskompetenz gibt es folgendende Möglichkeiten, wie ihr die 4CP bekommen könnt: Tutoriumsleitung (PRG 1 oder 2) oder Strufo (Praktikumsbetreuung), IT-Projektmanagement, Ringvorlesung Informatik + Gesellschaft. 

%\begin{center}
%\includegraphics[width=6.5cm]{comics/FuturamaComics/Comicsfinal/Bioinfocpbafg4}
%\end{center}

Solltet ihr nach 6 Semestern alle Prüfungen erfolgreich absolviert haben , wird euer Abschlussmodul für die Berechnung der Gesamtnote CP mäßig doppelt so hoch gewichtet, wie es angegeben ist. Die anderen Module werden anhand ihrer CP anteilig angerechnet (also CP * Note + CP * Note usw. / Alle Module (in CP), auf die Ihr eine Note gekriegt habt, und die in diese Berechnung mit eingerechnet wurden).

%\includegraphics[width=\linewidth]{comics/FuturamaComics/Comicsfinal/Bioinfocpbafg5}

\begin{flushright} Tim, modified by Stefan \end{flushright}