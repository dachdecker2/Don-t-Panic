Bibliothek, woran denkt man, wenn man diesen Begriff hört? Vermutlich an einen dunklen großen Raum mit vielen hohen Regalen aus Eichenholz mit einer ganzen Menge verstaubter Bücher darin, kleine Lesepulte mit grünen Lampen, eine alte Bibliothekarin, die am Eingang sitzt und über ihre Lesebrille schaut und an eine Stille, die ruhiger ist als fünf Minuten vor dem Urknall. Nun ja, auf manche Bibliothek mag das vielleicht zutreffen, aber ganz bestimmt nicht auf unsere Institutsbibliothek. Aber werfen wir zunächst einmal einen Blick ins Lexikon:\\

\begin{flushright}
\textit{Bi $\cdot$ bli $\cdot$ o\textasciiacute thek, Bi\textbar bli\textbar o\textbar thek~ \textless f.; -, -en\textgreater}

\textit{[grch. bibliotheke biblion „Buch“ + theke „Behältnis“] Büchersammlung, Bücherei; Raum od. Gebäude, in dem diese aufbewahrt wird. Die ältesten Bibliotheken des abendländischen Kulturkreises sind die babylonischen Sammlungen von Keilschrift-Tontafeln. Im griechischen Altertum dann Buchsammlungen in Alexandria und Pergamon. Im frühen Mittelalter waren Klöster die alleinigen Sammelstätten der Literatur, aus denen sich später dann die Universitätsbibliotheken entwickelten. [\dots]}
\end{flushright}

Und wenn man heute in so manche Unibibliothek geht, hat man immer noch das Gefühl, in einem Kloster zu sein. Aber wie schon gesagt, bei uns sieht das anders aus.\\

Die Bibliothek des Instituts für Informatik befindet sich im ersten Stock des Institutsgebäudes in der Robert-Mayer-Str. 11–15. Durch die große, rote Tür geht man in einen hellen, angenehmen Raum, der in der Regel von 09:00 Uhr bis 18:00 Uhr geöffnet ist. Hier findet man sicher nicht viele verstaubte Bücher (auch keine Tontafeln mehr, und Mönche sieht man auch eher selten), denn hier stehen viele der Bücher, die ihr für euer Studium braucht, und es lohnt sich immer, mal einen Blick hineinzuwerfen. Das sind natürlich vor allem Bücher aus den verschiedenen Bereichen der Informatik, aber auch Mathebücher, ein paar Physik- und Elektrotechnikbücher sowie viele der wichtigsten Fachzeitschriften. In der Regel findet man auch die von den Professoren zu den Veranstaltungen angefertigten Skripte als Kopiervorlage in der Bibliothek (obwohl es die Skripte oft auch im Internet gibt und ihr sie ausdrucken könnt).\\

In der Bibliothek steht Euch eine große Auswahl von Büchern aus allen Teilbereichen der Informatik zur Verfügung. Viele Studenten suchen die Bib auch zum Lernen und Arbeiten auf, wo die passenden Fachbücher sofort greifbar sind. Wenn ihr Euch jedoch in Lerngruppen zusammensetzen und unterhalten wollt, solltet ihr aus Rücksicht auf die anderen Studenten unser \textbf{Lernzentrum} nutzen. Wenn ihr durch den Haupteingang des Informatikgebäudes kommt, findet ihr es gleich links im Erdgeschoss.\\

Außer den Präsenzexemplaren kann man die Bücher auch für zwei Wochen ausleihen und mit nach Hause nehmen. Dazu müsst ihr in der Bibliothek Eure \textit{Goethe-Card} vorlegen (siehe Glossar auf S. \pageref{glossar}).\\

Wenn ein Prof. am Anfang der Vorlesung eine ganze Liste mit Büchern vorschlägt, ist es immer ganz gut, sich nicht gleich alle zu kaufen, sondern sich erstmal die Bücher zu leihen, mal reinzuschauen und damit zu arbeiten. So kann man feststellen, ob man mit dem Buch überhaupt etwas anfangen kann. Es sind zwar nicht immer alle der vorgeschlagenen Bücher in der Bibliothek zu finden und natürlich auch nicht 300 bis 400 Exemplare, so dass jeder eins mit nach Hause nehmen kann, aber meistens findet man ein Buch, das einem weiterhilft.\\

Sollte man mal ein Buch nicht auf Anhieb finden, so kann man im Internet auf der Bibliothekshomepage oder direkt auf den Recherche-Rechnern in der Bibliothek danach suchen, wer lieber mit dem eigenen Notebook sucht, kann auch das hauseigene WLAN nutzen; wenn das auch nicht hilft oder man eigentlich gar nicht so genau weiß, was man sucht, kann man ruhig auch mal einen von den Menschen fragen, die dort arbeiten.\\

Andere Bibliotheken, die noch hilfreich sein könnten:

%%%%%%%%%%
\begin{description}
\item[Die Bereichsbibliothek der Mathematik.] Im Mathegebäude, im obersten, dem vierten, Stock. Hier stehen, welch’ Wunder, vor allem Mathebücher und -skripte. Hat den Charme eines Krankenhausflures, eine der Bibliotheken, in der man besser keine Stecknadel fallen lässt, bestimmt direkt aus einer Klosterbibliothek hervorgegangen (s.o.).

\item[Die Universitätsbibliothek.] Direkt an der U-Bahn-Haltestelle Bockenheimer Warte, nicht zu übersehen. Hier gibt es viele Bücher mit Themen aus anderen Fachbereichen. Die großen Säale sind gut geeignet, um sich zum stillen Lernen zurückzuziehen.

\item[Die Deutsche Bibliothek.] Sie ist die zentrale Archivbibliothek der Bundesrepublik Deutschland. Sie sammelt alle deutschsprachigen Veröffentlichungen des In- und Auslandes seit 1945, ferner Übersetzungen deutscher Werke, fremdsprachige Werke über Deutschland und Exilliteratur der Jahre 1933 -- 1945.

\item[Die Technische Bibliothek.] Zu finden in der Waldschmidtstraße 39, kann man sie von der U-Bahnstation Zoo nach einem dreiminütigen Fußmarsch erreichen. In Sachen Unhöflichkeit ist sie unübertroffen, und die Ausleihbeschränkung auf maximal fünf Medien macht sie auch nicht sympathischer, geschweige denn die Öffnungszeiten (Mo, Do: 13 -- 19 Uhr; Di, Mi, Fr: 13 -- 17 Uhr). Aber man findet dort Bücher, die man in den oben genannten Bibliotheken eventuell nicht findet.

\item[Die Bibliotheken der anderen Fachbereiche.] Wo diese sind und was ihr da so findet, erfahrt ihr am besten von den Angehörigen der entsprechenden Fachbereiche.
\end{description}
%%%%%%%%

\begin{flushright} Björn\end{flushright}