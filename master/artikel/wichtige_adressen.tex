\subsection*{Deine Fachschaft}

\url{http://fs.cs.uni-frankfurt.de/}\\
\url{http://fs.cs.uni-frankfurt.de/forum/}\\
\emailfachschaft

\subsection*{Dein Fachbereich}

Fachbereich Informatik und Mathematik (12)\\
Institut für Informatik\\
Robert-Mayer-Straße 11–15\\
60054 Frankfurt\\

\begin{tabular}{ll}
Institutssekretariat & 798 – 23325 \\
Informatik-Bibliothek: & 798 – 22287\\
Informatik-Prüfungsamt: & 798 – 28279\\
Informatik-Studienberatung: & \\
~~~Prof. G. Schnitger & 798 – 28326\\
%~~~und die Fachschaft & 798 – 23933\\
\end{tabular}

\subsection*{Informationssystem}

Verwaltung Deiner persönlichen Daten rund ums Studium:

\begin{center}\url{http://go.uni-frankfurt.de}\end{center}

Vorlesungsverzeichnis:

\begin{center}\url{http://qis.server.uni-frankfurt.de}\end{center}


\subsection*{Der AStA}

Die Langform lautet Allgemeiner Studenten-Ausschuss; politisch korrekt nennt er sich auch meistens Allgemeiner Studierenden-Ausschuss. Der AStA ist die (leider nur von einer kleinen Minderheit) gewählte studentische Vertretung der gesamten Studierendenschaft, also aller Studierender der Universität. Der AStA hat verschiedene „Referate“, die sich mit speziellen Bedürfnissen von Teilen der Studierendenschaft oder auch besonderen Angeboten für eben selbige beschäftigen.

\paragraph{AStA-Büro}~\\
Studentinnenhaus, Mertonstr. 26–28, Raum B1\\
798 – 23181\\
Öffnungszeiten: Mo - Fr 9:30 -- 13:00 Uhr

\paragraph{Rechts- und BaföG-Beratung}~\\
Raum B7\\
798 – 23175 oder 798-33098\\
Bockenheim:\\
Mo, Di, Fr.: 10.00 - 12.00 Uhr
Mo, Di, Mi und Do : 13.00 - 15.00 Uhr


\paragraph{Sozialreferat}~\\
Raum B7\\
Beratung: Mo 12:00 -- 13:00 Uhr\\
Mi 12:00 -- 13:00 Uhr\\
Fr 13:00 -- 14:00 Uhr


\paragraph{BAföG}~\\
Wer wieviel BAföG bekommt, können wir hier natürlich nicht schreiben, aber durch die letzten Gesetzesänderungen lohnt sich eigentlich für Jeden der Weg ins BAföG-Amt.

\paragraph{Amt für Ausbildungsförderung}~\\
Neue Mensa, 4. Stock, Vorzimmer 410\\
Bockenheimer Landstraße 133\\
60325 Frankfurt\\
Sprechzeiten: Mo Di Fr 10:00 - 12:00 Uhr, Mo Di Mi Do 13:00 - 15:00 Uhr
Telefonsprechzeiten: Mo-Fr 8:00-10:00 Uhr\\

\subsection*{StuGuG}
Referat für Studienguthaben\\
Gräfstraße 39 – 4. Obergeschoss\\
Sprechzeiten: Mo + Di 14:00 - 16:30 Uhr\\
Mi 9:30 - 12:30\\
Do + Fr 14:00 - 16:00 Uhr\\
Telefon: 798-22683/28899/22206/28385\\
\emailstugug

\subsection*{Wohnen}
Wenn ihr einmal Hotel Mama verlassen wollt, könnt ihr versuchen, im Studentenwohnheim einen Platz zu erhalten. Einzige wirkliche Voraussetzung dafür ist, dass ihr immatrikuliert seid. Berücksichtigt werden Bewerber nicht, wenn sie schon älter als 30 Jahre sind, schon länger als 14 Semester studieren, ein abgeschlossenes Hochschulstudium haben.\\
~\\
Studentenwerk Frankfurt/Wohnheimabteilung\\
Bockenheimer Landstr. 133\\
Raum 319 u. 320\\
60325 Frankfurt\\
Telefon: 798 – 23021 Telefax: 798 – 23029\\
\emailwohnen

\paragraph{Allgemeine Informationen}~\\
Telefon: 01801-STUDENTENWERKF\\
\emailinfo\\
Eine sehr umfangreiche Liste mit Kontaktadressen, die euch bei der Wohnungssuche weiterhelfen können, findet ihr unter:
\textbf{http://www.studenten-wg.de}
oder\\
http://www.studentenwerkfrankfurt.de/?id=93
