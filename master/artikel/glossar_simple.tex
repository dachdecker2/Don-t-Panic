%%%%%%%%%%%%%%%%%%%%%%%%%%%%

\label{glossar}

\begin{description}

 
\item[AStA] Allgemeiner Studierendenausschuss. Einmal im Jahr wählen
die Studierenden das \textbf{StuPa}, und dieses wählt sich einen
Vorsitz, den AStA. Von dort aus werden beispielsweise eure
studentischen Beiträge, also 9,50 \euro{} von den 344,98 \euro{}, die
ihr zu Anfang des Semesters überwiesen habt, zu einem Teil den
Fachschaften zugeteilt und zu einem anderen, deutlich größeren Teil
für zentrale studentische Aufgaben verwendet.


\item[B.Sc.] Bachelor of Science. Der (Jung-)Geselle der Wissenschaft.
Kleiner Bruder vom Master (siehe \textbf{M.Sc.}).


\item[Bachelorordnung] Die Bachelorordnung legt exakt fest, wie ihr zu eurem
Studienabschluss kommen könnt. Wann ihr das macht hängt allein von eurer eigenen
Motivation ab.  Der Abschnitt zur Studienorganisation soll euch in
halbwegs verständlichem Deutsch klar machen, wie euer Studium aussehen könnte.
Beim Abschnitt Prüfungsorganisation werden mehr die harten Prüfungsregeln
festgelegt. \textbf{Die Bachelorordnung sollte jeder Student im Verlauf seines
Grundstudiums mal zur Hand nehmen und wenigstens lesen,} verstehen
wäre natürlich noch besser! Fragen kann man im \textbf{Prüfungsamt} stellen (oder bei uns). Außnahmen regelt der \textbf{Prüfungsausschuss}.
Die aktuelle Ordnung kann man hier herunterladen:
\url{http://www.cs.uni-frankfurt.de/images/pdf/informatik/bachelor2/bachelorordnung\_neu.pdf}


\item[Bockenheim, Campus] Die Betonwüste, in der du dich gerade
aufhältst. Früher der einzige und zentrale Campus der Frankfurter
Universität, heute nur der Rand. Hier befinden sich neben der Informatik nur noch
Mathematiker, Gesellschaftswissenschaftler
und die Verwaltung der Uni. Bis 2017 soll der Campus aufgegeben werden.
Solltet ihr euer Studium in Regelstudienzeit abschließen, müsst ihr euch aber über den Umzug keine Gedanken machen.


%\item[Burschenschaften] Es gibt sie noch: Studentenverbindungen,
%„schlagend“ oder nicht, „farbentragend“ oder nicht. Wir raten euch
%jedoch zur Vorsicht; das elitäre Gehabe, ihren Mitgliedern Wohnungen,
%Arbeitsstellen und anderes auf vetternwirtschaftliche Weise
%zuzuschanzen und auch die bei diesen Organisationen oft anzutreffende
%Deutschtümelei sind uns höchst suspekt.

\item[CP] Credit Points. Berechnungseinheiten des ECTS (European
Credit Point Transfer System), die hochschulübergreifend angerechnet
werden können sollten. Umrechnung: 1 CP $\approx30$ Stunden
Arbeitsaufwand. 180 CP = 1 Bachelor. Die beste Übersetzung bis jetzt:
Glaubwürdigkeitspunkte.


\item[Dekan] Der Dekan ist ein Professor des Fachbereichs. Er wird vom
Fachbereichsrat gewählt und „leitet die Geschäfte des Fachbereichs“
für einen Zeitraum von drei Jahren, d.h. er vertritt den Fachbereich
nach außen, und er führt auch den Vorsitz im Fachbereichsrat. Er darf
viele Entscheidungen auch selbst treffen, für die früher ein Beschluss
des \textbf{FBR} notwendig war.


\item[Dekanat] siehe \textbf{Dekan}, \textbf{Prodekan} und
\textbf{Studiendekan}. Daneben wird der Begriff „Dekanat“ für das
Sekretariat eines Fachbereichs benutzt, dort wird ein Fachbereich
verwaltet.


\item[Direktorium] Eigentlich der Institutsrat, aber das hört sich
wohl nicht so toll an. Es handelt sich um ein Gremium, das die Belange
eines Instituts behandelt und teilweise, wo nicht der  Fachbereichsrat
gefragt ist, entscheidet.


\item[Direktor, geschäftsführender] Wird auch als GD abgekürzt. GD ist
kürzer als „Der mit dem Institutsrat tanzt“; er hat den Vorsitz im
Direktorium und repräsentiert das Institut nach aussen.


\item[Direktorat] Hierbei handelt es sich um das Pendant zum Dekanat,
nur eben auf Institutsebene. Sämtliche Angelegenheiten der Informatik
kann man zunächst im Direktorat der Informatik regeln. Hier könnt ihr
auch die \textbf{Bachelorordnung} erhalten.


\item[Evaluierung/Evaluation] Studierende müssen per Gesetz in die
Evaluierung, sprich die Reflexion und Bewertung von Lehrveranstaltungen
einbezogen werden. Das ist im Großen und Ganzen auch der einzige
Weg, den Studierende haben, um ihre Meinung zu einer Veranstaltung
kund zu tun. Die Alternativen wären nicht ernstzunehmende Nörgelei
oder die Fachschaft anzusprechen. Dennoch habt ihr dabei eine
einigermaßen anonyme Möglichkeit zu Kritik und Lob und die Professoren
bekommen tolle Statistiken. Wenn ihr also zur Mitte  oder zum Ende
eines Semesters Evaluierungs-Fragebögen bekommt, nehmt sie
ernst und beantwortet die Fragen. Selbst wenn ihr eine Veranstaltung
abgebrochen habt (gerade dann!), ist es wichtig zu wissen, warum und
was besser gemacht werden müsste.


\item[Fachschaftsraum (neuer Fachschaftsraum)]  Der große Raum, den
man erreicht, indem man vom Lernzentrum über eine der Glasgänge geht. 
Hier findet Donnerstag Nachmittag ab etwa 16:00 Uhr das
Gremientreffen statt, und auch sonst kann man hier unter Umständen
einen hilfsbereiten Fachschaftler antreffen. 


\item[Fachschaftstreffen] Das einzige regelmäßige Treffen der
Studierendenvertretung in der Informatik. Er findet jeden Donnerstag
um 16:00 Uhr im \textbf{neuen Fachschaftsraum} der Informatik statt.
Wir besprechen hier alle wichtigen und manchmal auch unwichtigen
Dinge, die am Institut und am Fachbereich geschehen, nicht geschehen,
geschehen sollten oder besser nicht geschehen wären. Wer sich für
Fachschaftsarbeit interessiert, ist beim Gremientreff genau richtig.
Außerdem ist die Fachschaft Informatik natürlich das netteste, was man
überhaupt in der Robert-Mayer-Straße 11–15 finden kann.


%\item[Fakultätentag (Informatik)] Eine bundesweite Konferenz von
%Abgesandten der Informatikfachbereiche und -institute. Hier sitzen
%nicht nur Professoren, sondern auch studentische Vertreter, die
%regelmäßig von der KIF entsandt werden. Hier werden Themen
%debattiert, die später oft zu bundesweiten Übereinkünften erhoben
%werden, wie z.B. Rahmenrichtlinien für Bachelor-/Master-Studiengänge
%und Kreditpunktesysteme für Informatikstudiengänge. Der Fakultätentag
%ist zwar kein gesetzgebendes Organ, aber seine Vorschläge werden oft
%von den jeweiligen Ministerien übernommen oder als Grundlage
%verwendet.


\item[FBR] Der Fachbereichsrat. Es handelt sich um ein Gremium, in dem
7 Professoren, 3 Studierende, 2 Wissenschaftliche Mitarbeiter und 1
Administrativ-technischer Mitarbeiter sitzen. Hier wird von den
gewählten Vertretern aller \textbf{Statusgruppen} über zentrale
Belange wie z.B. die Bachelorordnung, Verwendung von Geldern, das
Lehrangebot und vieles mehr entschieden. Die Studenten wählen dabei ihre Vertreter direkt.


\item[Fischerräume] Die Rechnerräume, deren Eingang sich hinter dem
Magnushörsaal auf der Emil-Sulzbach-Straße versteckt, nennen sich so,
weil dort früher mal eine Firma dieses Namens ansässig war. Hier kann
man Programmieraufgaben lösen, sich mit Linux vertraut machen und sich
mit dem Gedanken anfreunden, dass später vorausgesetzt wird, wie man
mit solchen Kisten umgeht. Ansonsten kann man die Fischerräume im
Sommer auch wunderbar als Backofen verwenden.


\item[FS/FS-Inf] Die Fachschaft bzw. die Fachschaft Informatik. Der
Begriff „Fachschaft“ wird aber auch in einem wesentlich engeren Sinn
für die wenigen Leutchen benutzt, die sich ganz aktiv um die Belange
der Studierenden kümmern. Siehe auch den Artikel
„Fachschaftsarbeit“ (Seite \pageref{fachschaftsarbeit}).

\item[Goethe-Card] Diese kleine Plastikkarte erfüllt gleich mehrere
Aufgaben: sie dient euch als Studierendenausweis, Bibliotheksausweis
und als Semesterticket. Der Zeitraum, für den die Goethe-Card gültig
ist, wird mit Spezialtinte auf die Karte aufgedruckt. Nach der
Rückmeldung könnt Ihr den Aufdruck erneuern, indem ihr einen der dafür
vorgesehenen Automaten verwendet (z.B. im Gebäude \textit{Neue
Mensa}). Durch den enthaltenen Chip werden Informationen zu den von
Euch ausgeliehenen Büchern gespeichert, außerdem könnt ihr sie als
Schlüssel für Schließfächer verwenden. Ihr könnt die Goethe-Card mit
Geld ``aufladen'', und mit diesem Guthaben die von der Uni
aufgestellten Kopierer nutzen, oder in der Mensa bezahlen.
%(falls man keine Scheu vor dem Essen hat!)
Last but not least habt ihr durch die Goethe-Card freien Eintritt im Palmengarten.
Plastikhüllen, um Eure
Karte vor Beschädigungen zu schützen, gibt es im \textit{AStA}-Büro.
Bei Einführung der Goethe-Card gab es erhebliche Kontroversen um die
Sicherheit der persönlichen Daten, die auf der Karte gespeichert sind.
Wer absolut sicher gehen möchte, dass seine Daten nicht ungewollt aus
der Ferne ausgelesen werden, kann beim AStA auch eine Metallhülle
erwerben, die vorgeblich dagegen schützt. Eigentlich sollten aber auch
Aluminiumfolie oder Antistatikfolie gegen RFID schützen.


%\item[HHG] Das Hessische Hochschulgesetz. Es regelt die allgemeinen
%Dinge, die für alle Hochschulen in Hessen gelten. Da Bildung ja
%Ländersache ist, hat jedes Bundesland ein eigenes Hochschulgesetz,
%und diese Gesetze weisen auch gravierende Unterschiede auf, obwohl
%sämtliche Gesetze die Regelungen des \textbf{HRG} nicht verletzen
%dürfen. So gibt es z.B. in Deutschland Landesgesetze, nach denen
%Studiengebühren erhoben werden müssen und andere, die selbige
%verbieten. Ebenso darf man in manchen Ländern nur geringfügig länger
%studieren als die Regelstudienzeit, und in anderen Ländern gibt es
%mehr Freiheit. Auch die Vertretung der Studierenden durch gewählte
%Vertreter ist in einigen Ländern einfach nicht vorgesehen.


\item[HiWis] Hilfswissenschaftler. Das sind eure \textbf{Tutoren} in
den Übungen, die studentischen Mitarbeiter in der Bibliothek oder in
der RBI. Diesen günstigen Arbeiterschwärmen kann man sich meist
schon nach ein paar Semestern anschließen.


%\item[HRG] Das Hochschulrahmengesetz. In ihm sind die Richtlinien
%festgeschrieben, die die Bildungs- und Hochschulpolitik auf
%Bundesebene den Ländern als Rahmen gibt, innerhalb dessen sie agieren
%müssen. Damit das überhaupt möglich ist, bei so verschiedenen
%Auffassungen von Bildungspolitik in den Ländern, kann das HRG nur ein
%sehr grobes Raster vorgeben.


\item[H I– VI, H 1-16, HA, HB, HH] Die Hörsäle. Die römisch
nummerierten sind die moderneren und vor allem größeren im Bauteil E,
dem Hörsaalgebäude, während sich die arabisch nummerierten auf der
anderen Seite des Gebäudeteils E über dem ehemaligen Café
Struwwelpeter befinden. Die Hörsäle HA, HB und HH findet ihr im
Hauptgebäude. Daneben gibt es noch den \textbf{Magnus-Hörsaal} und die
\textbf{NM xxx}.


\item[HRZ-Account] Jeder Studierende erhält vom
\textbf{\underline{H}ochschul\underline{r}echen\underline{z}entrum}
einen Account. Durch diesen lassen sich verschiedene
Dienste der Uni Frankfurt nutzen. Leider sind die Accountnamen
inzwischen pseudonym und schlecht zu merken. Der wichtigste
Nutzungsgrund liegt darin, dass ihr über den Zugang die
persönlichen Daten Eures Studiums verwalten und euch zu Prüfungen
anmelden könnt, siehe dazu auch \textbf{QIS-LSF}. Ansonsten braucht
ihr ihn für WLAN (an fast allen Unis Deutschlands), eventuell E-Mail,
FTP und andere Dinge. Die Zugangsdaten erhaltet ihr zusammen mit der
\textbf{Goethe-Card}. Nicht zu verwechseln mit dem
\textbf{RBI-Account}.

\item[Institutsrat] Das zentrale Gremium, das die institutsinternen
Angelegenheiten der Informatik vorentscheiden darf/soll/kann/muss
(Unzutreffendes wird je nach Gelegenheit gestrichen), wird auch
\textbf{Direktorium} genannt.


%\item[KIF] Nein, hierbei handelt es sich nicht um ein
%fehlgeschlagenes Experiment zur Bewusstseinserweiterung, sondern um
%die Konferenz der Informatik-Fachschaften. Hier treffen sich zweimal
%im Jahr Informatik-Fachschaftler (und auch ehemalige) aller
%deutschsprachigen Hochschulen. Dabei wird miteinander diskutiert, es
%werden gemeinsam Stellungnahmen erarbeitet, und nebenbei wird auch
%mal ein Biergarten besucht. Die KIF wird wechselnd von einer
%Informatik-Fachschaft organisiert und durchgeführt. Von der KIF
%werden auch die studentischen Mitglieder des \textbf{Fakultätentags
%(Informatik)} bestimmt.

\item[Kommilitonen] Das sind die Leute, die mit dir studieren. Ein Kollektiv
voller lernbegeisterter Studenten, die dir dabei helfen all den Stoff, den du
verschlafen hast, mit dir nachzuarbeiten. Und da ihr es als Einzelgänger im
Studium schwer haben werdet, gehören eure Kommilitonen zu euren wichtigsten
Ressourcen.

\item[Lernzentrum] Befindet sich im Erdgeschoss des
Informatikgebäudes, gleich links wenn ihr zum Haupteingang
hineinkommt. Hier könnt ihr versuchen zu lernen, sollte ihr euch
wieder erwarten doch inmitten von Diskussionen konzentrieren können.
Im Lernzentrum sind die Skripte der momentan laufenden Basismodule
vorhanden. Außerdem habt ihr hier die Möglichkeit einen Mitarbeiter
der Uni um Hilfe bei euren Aufgaben zu bitten. Es gibt sogar
Brettspiele zum Ausleihen. An das Lernzentrum sind die
''Student-Lounge'' genannten Räume der Fachschaft angeschlossen.


%\item[LuSt-Ausschuss] Der Ausschuss für Lehr- und
%Studienangelegenheiten am Institut für Informatik hat unter anderem
%die folgenden Aufgaben: Fortschreiben der Studienordnungen, ggfs.
%Erstellen einer neuen Studienordnung; Zuteilung der Tutoren-Stellen;
%Koordination bei der Erstellung des Lehrangebots, Unterbreitung von
%Vorschlägen für Lehraufträge; Erarbeitung und Koordination von
%Anwendungsfachregelungen und das Finden von Lösungen für alle
%Probleme, die Lehre und Studium betreffen.


\item[Magnus-Hörsaal] „Magnus“ kommt aus dem lateinischen und bedeutet
groß. Warum dieser Hörsaal, übrigens der einzige (!) der Informatik,
so heißt, obwohl er euch wohl gerade ziemlich klein vorkommt, weiß
niemand verlässlich. Ein Gerücht erzählt von einem Chemiker dieses
Namens, den die Physikalische Chemie auf diese Weise geehrt hat, als
sie vor der Informatik in diesem Gebäude untergebracht war.

% Mentoren scheint es nicht mehr zu geben...
%\item[Mentoren] Irgendwann demnächst bekommt ihr einen Professor als
%„Mentor“ zugeteilt. Er soll für euch ein Ansprechpartner in allen
%studienrelevanten Fragen sein. Achtet am besten auf die Aushänge vor
%dem \textbf{Direktorat} und bei den jeweiligen Professuren.


\item[Modul] Ein Modul ist eine Lehreinheit, das die fachlich sinnvoll
aus ein bis mehreren Lehrveranstaltungen zusammengesetzt ist. Ein
Modul wird innerhalb eines Semesters oder auch über 2 Semester
veranstaltet.


\item[M. Sc.] Master of Science. Der Meister der Wissenschaft. Großer
Bruder vom Bachelor (siehe \textbf{B.Sc.}).


\item[NM xxx] Das sind die Seminarräume im ersten Stock der
\textbf{Neuen Mensa}, \textsl{xxx} steht hierbei für die jeweilige
Raumnummer.


%\item[Per Anhalter durch die Galaxis (The Hitchhiker's Guide to the
%Galaxy, H2G2)] Vierteilige Trilogie in fünf Bänden von Douglas Adams,
%die nicht nur für Informatikstudierende interessant ist. Leider
%verstarb Douglas Adams 2001; viel zu früh! In jüngster Zeit wurde man
%vielleicht durch die (leider unsägliche) Verfilmung von 2005 auf den
%Titel aufmerksam.


\item[Prodekan] Sowas wie ein stellvertretender Dekan. Der Dekan und
der Prodekan können auch eine Art „Aufgabenteilung“ unter sich
vereinbaren.


\item[Prüfungsamt] Im Prüfungsamt meldet man sich für Klausren, mündliche Prüfungen und Studenleistungen aller Art an.
Zwar geschehen inzwischen die meisten Anmeldungen über das \textbf{QIS-LFS}, aber das ist manchmal unzuverlässig.
% (Wurde ja auch von Informatikern gemacht\dots)


\item[Prüfungsausschuss] Der Prüfungsausschuss ist ein Ausschuss der sich mit
Prüfungen befasst (Sach bloß!). Hier siten Professoren und Studenten, die
Entscheidungen treffen, die das \textbf{Prüfungsamt} nicht treffe darf.  Vor
allem kann man hier Außnhameregelungen formlos beantragen. Dazu schreibt man
einfach einen Brief an den Ausschuss, in dem man kurz sagt was man will, und
wieso. Den kann man dann im \textbf{Prüfungsamt} einwerfen.


\item[Prüfungsprotokoll-Datenbank] Etwas, bei dem dir ruhig mulmig zu
Mute sein darf, ist eine mündliche Prüfung. Doch dem mulmigen Gefühl
kann abgeholfen werden: Schau dir doch einfach mal an, was den anderen
so passiert ist, als sie in ihrer Prüfung waren. Dafür gibt es eine
eigene Datenbank, die ihr unter
\url{http://myexam.gdv.cs.uni-frankfurt.de/} finden könnt. Mitmachen
ist dabei Pflicht, denn ohne Geben gibt es bei dieser Idee auch kein
Nehmen!

% NICHT MEHR AKTUELL \item[Q\&D] Quick \& Dirty, die mehr oder weniger
% regelmäßig erscheinende Zeitschrift über die Welt, unter besonderer
% Rücksichtnahme auf die Universität, den Fachbereich und das
% Institut; hauptsächlich also über euch. Mit viel Arbeit verbunden,
% daher sind freiwillige Mithelfer immer gerne gesehen.


\item[QIS-LSF] Das elektronische Informationssystem der Uni Frankfurt.  Da
''Qualitätssteigerung der Hochschulverwaltung im Internet durch Selbstbedienung
- Lehre Studium Forschung'' offensichtlich zu lang ist, haben viele den Namen
zugunsten des Akronyms längst verdrängt.  Ihr könnt euch das Portal als die
Schaltzentrale Eures Studiums vorstellen. Hier könnt Ihr das
Vorlesungsverzeichnis für das laufende oder auch zukünftige Semester einsehen.
Einige Fachbereiche geben eine kommentierte Liste mit den Veranstaltungen des
Semesters heraus, mit Anforderungen, Anmeldefristen und eventuell sogar schon
Literaturtipps.
%Da ihr als Studenten prinzipiell das Recht habt, an
%allen Veranstaltungen teilzunehmen (von begrenzter Platzzahl und
%Bedingungen zum Scheinerwerb abgesehen), ist es vielleicht mal
%interessant, sich umzusehen, was in anderen Fachbereichen so los ist.
Für das Studium erhaltet Ihr Zugangsdaten für einen
\textbf{HRZ-Account}. Mit diesen könnt ihr Euch einloggen und Eure
ganz persönlichen Daten verwalten. Ihr könnt \textsl{unter anderem}
das Studienguthaben kontrollieren, Stundenpläne zusammenstellen, Euch
online für Seminare anmelden, und nicht zuletzt eure
Studienbescheinigungen/Stammdatenblatt abrufen. Früher bekam man seine
Semesterunterlagen mal per Post zugeschickt, heute ``dürft'' Ihr sie
euch selber ausdrucken, weil die Uni damit Kosten spart. Zum
Vorlesungsverzeichnis gelangt Ihr über
\url{http://qis.server.uni-frankfurt.de/}, zur Verwaltung der
persönlichen Daten geht es am schnellsten über
\url{http://go.uni-frankfurt.de}



\item[RBI] Rechner-Betriebsgruppe Informatik. Und jetzt ist Schluss
mit lustig! In den \textbf{Fischerräumen} gibt es eine ganze Menge
Rechner der RBI, die ihr benutzen dürft. Und es gibt ein paar Geräte,
die zwar nicht an jedem Rechner installiert sind, die ihr aber
trotzdem benutzen dürft: DVD-R-Brenner, Scanner und Drucker. Und zwar
alles für euch noch kostenlos. Das heißt aber, dass ihr die Sachen
auch mit einem großen Maß an Verantwortungsgefühl und einem Auge für
eure Mit-Studis benutzen müsst! Da in früheren Jahren manche Leute
bergeweise Überflüssiges gedruckt haben, wurde die Anzahl der Seiten,
die Ihr pro Semester drucken dürft, von der RBI beschränkt auf 500
Seiten pro Semester. Ebenfalls möchten wir euch ausdrücklich davor
warnen, Internettauschbörsen jeglicher Art in der RBI zu benutzen. Wer
schon glaubt, dass er unbedingt 'vom Laster gefallene' Filme etc.
benötigt, der soll sie sich bitte \textbf{nicht} in der RBI besorgen.
Die Geräte sind dazu da, dass ihr euer Studium betreiben könnt, bitte
geht also sorgfältig mit ihnen um. Wer sich nicht daran hält (und
meint, schlauer zu sein als die Administratoren), dem kann sein
Rechnerzugang auch wieder entzogen werden!% \emph{Big brother is
%watching you!}
%\item[RBI] Rechner-Betriebsgruppe Informatik. In Frankfurt hat die
%Informatik ein eigenes kleines Rechenzentrum, das sich bemüht den
%chaotischen Bedürfnissen der Informatiker nachzukommen. Zwar weiss
%niemand so genau, was die eigendlich machen sollen, aber es gibt eine
%ganze Menge Sachen, für die man an der freundlichen Herren der RBI
%vorbei muss. Als Student ist das für den RBI-Account, Windows-Lizenzen
%oder zur Beamerausleihe. Wenn man später mal im Institut arbeitet, hat
%man eventuell durch die Schlüsselvergabe oder Arbeitsgruppenserver
%mit ihnen zu tun. Mit den Admins lässt sich meistens reden, wenn man
%höflich ist. Leute die am Drucker herumspielen, weil sie glauben das
%Papier sei alle. User, die Hinweise ignorierren, die auf dem Drucker
%sind.  Allgemein wenn Hinweise und die FAQ ignoriert werden oder
%irgendjemand an der Hardware rumpfuscht. Beschuldigungen für
%selbstverursachte Probleme.  Irgendjemand, der meint auf ihre Monitore
%schauen zu dürfen.  Arrogantes Auftreten beim Stellen von Fragen.  Und
%als letztes, wenn Hilfe für Probleme mit privaten Computern verlangt
%wurde, die nicht das geringste mit den Unisystemen zu tun haben.
%Wenn ihr all diese Dinge vermeidet, sollten auch so Sachen wie
%Softwareinstallationen oder das Anheben der Quota, wenn man grade eine
%Bachelor- oder <Masterarbeit schreibt, kein Problem sein.


\item[RBI-Account] Als Informatiker könnt ihr neben eurem
\textbf{HRZ-Account} auch noch einen Account in der \textbf{RBI}
bekommen. Damit habt ihr Zugang zu den Rechnern der Fischerräume und
kommt mit Glück auch in das Informatik-WLAN rein. Im Paket drin sind
eine weitere E-Mail Adresse (mit sinnvollerem Benutzernamen),
Webspace, und 500 Seiten zum drucken.  Außerdem werdet ihr für manche
Praktika einen Zugang brauchen und da der Antrag eine Weile dauert,
solltet ihr ihn möglichst innerhalb der ersten Woche
beantragen.\textbf{WARNUNG:} Hacken, illegales Filesharing und andere
Dinge, die die Admins nerven oder dazu führen, dass sie sich mit
Juristen beschäftigen müssen, haben ernste Konsequenzen.
Bedenkt, dass das ein Rechenzentrum ist, in dem sich
Leute seit über 20 Jahren mit Informatikstudenten herumschlagen und
die dazu noch HiWis aus euren Reihen einstellen.
%Wer Mist macht, kann exmatrikuliert werden.



\item[Riedberg, Campus] Gelegen am Niederurseler Hang, soll er einmal
alle naturwissenschaftlichen Fachbereiche beherbergen -- schon jetzt
ist vieles dort, was vegetiert und stinkt. Zum Beispiel Chemie,
Pharmazie, Geowissenschaften, Physik und alle, die sich nicht wehren
konnten. Teilweise so abgelegen, dass fast jede Art ihn zu erreichen
unangenehm ist (war früher noch schlimmer). Unsere Empfehlung für
Bockenheim-Riedberg: Straßenbahlinie 16 und U-Bahn U9.



%\item[Senat] Der Senat ist das Organ, das – auf Universitätsebene –
%dem Fachbereichsrat entspricht. Er trifft alle Entscheidungen, die
%die Universität im Allgemeinen betreffen. Er besteht aus neun Profs,
%drei Studis, drei WiMis und drei SoMis. Sein Vorsitzender ist der
%Universitäts-Präsident. Aufgrund der Anzahl der Mitglieder können dem
%Senat nicht aus allen Fachgebieten Vertreter angehören. Da ist es
%natürlich schon fraglich, wessen Interessen hier vertreten werden.


\item[Skript] Entweder eine frühere Mitschrift eines fleißigen
Studenten (eher selten anzutreffen) oder die Kopien der vom Professor
benutzten Folien, gelegentlich auch ein extra hierfür entworfener
Text. Ist entweder in der entsprechenden Fachbereichs-Bibliothek, auf
der Internet-Präsenz des jeweiligen Dozenten oder gar nicht zu finden.
Dabei sollte man jedoch zwei Dinge beachten: 1. Das Lesen des Skriptes
ersetzt nur sehr selten die Teilnahme an der Veranstaltung. 2. Wer ein
komplettes Skript auf dem Drucker im RBI ausdruckt, läuft Gefahr, von
vor dem Drucker wartenden Kommilitonen gesteinigt zu werden, oder
zumindest als DAU (Dümmster anzunehmender User) des Monats nominiert
zu werden; vgl. \textbf{RBI}.


\item[SoMis] Sonstige Mitarbeiter. Gemeint ist damit die Statusgruppe
der administrativ-technischen Mitarbeiter, was sich aber immer noch
kein Mensch merken kann. Enthalten ist darin alles, was nicht
Professor, Student oder WiMi ist. Da diese Gruppe früher Sonstige
Mitarbeiter hieß, erklärt sich diese Abkürzung wohl von alleine.


\item[Statusgruppen] Die Statusgruppen der Universität sind
Professoren, Studierende, Wissenschaftliche Mitarbeiter (WiMi) und die
administrativ-technischen Mitarbeiter (SoMis).


\item[Student Lounge] Die \textbf{Fachschaftsräume} hinter dem Lernzentrum, die
normalerweise offen sind und von allen Studierenden als eine Erweiterung vom
\textbf{Lernzentrum} oder auch als ein Platz, wo man sich auf Sofas entspannen
könnte, betrachtet werden kann. Auch hier solltet ihr allgemeine Regeln des sozialen Zusammenlebens beachten.
Also haltet den Raum sauber, entsorgt euren Müll und haltet euch an die Hausordnung.


\item[Studiendekan] Auch ein Mitglied der „Drei Dekane“. Er ist im
Besonderen zuständig für Fragen von Lehre und Studium.


\item[Studien-Service-Center] Gelegen im Gebäude vor der \textit{Neuen
Mensa}. Hier helfen euch nette Mitarbeiter, wenn es bei den
Formalitäten eures Studiums hakt, z.B. bei der Rückmeldung zum
nächsten Semester. Auch alle Arten von Anträgen (z.B. zum
Teilzeitstudium, s.S. \pageref{teilzeitstudium}) können hier gestellt
werden.

%\item[StuGuG] Studienguthaben-Gesetz, \textbf{Studiengebühren}. Ab WS
%06/07 wurden allgemeine Studiengebühren eingeführt, sodass jeder
%Student pro Semester 500 \euro{} zusätzlich zu den Semestergebühren
%löhnen musste. Zum Sommersemester 2008 wurde das Gesetz wieder
%gekippt, und man muss die 500 \euro{} vorerst nicht mehr zahlen. Man
%darf gespannt sein, wie der Rechtsstreit um Studiengebühren letztlich
%ausgeht.


\item[StuPa] Das StuPa ist, wie der Name „Studentenparlament“ ja auch
sagt, das Parlament der Studierenden. Und wie es sich für jedes
ordentliche Parlament gehört, wird auch im StuPa Politik gemacht:
Deshalb hält man die StuPa-Sitzungen auch kaum aus. Studentische
Interessen stehen leider eher selten im Vordergrund. Welche
politischen Hochschulgruppen in das StuPa reingewählt werden, könnt
ihr versuchen in Wahlen zu beeinflussen.


\item[SWS] Abkürzung für Semesterwochenstunde, sprich die Zeit, die
eine Veranstaltung im Verlauf des Semesters wöchentlich dauert. Die
private Vor- bzw. Nacharbeit dauert normalerweise noch das doppelte
dazu. Also: 1 SWS = 3 Stunden fürs Studium wochentlich.


\item[Tutor] Wie vieles andere hat es zwei Seiten, Tutor zu sein.
Zuerst die unangenehme: Der Stundenlohn von 8,50 \euro{} und 10
\euro{} falls ihr eine Bachelor habt, ist bundesweit festgeschrieben,
so gut wie unveränderbar und eine bodenlose Unverschämtheit. Jetzt die
angenehme: Durch die intensivere und etwas anders motivierte
Herangehensweise kannst du den Stoff nochmal vertiefen; auch wenn die
entsprechende Basismodulprüfung bereits vorbei sein sollte, ist das
nicht zu unterschätzen! Dein Arbeitsplatz ist direkt an der Uni: kein
elendes Pendeln mehr! Weiterhin hast du jetzt die Möglichkeit,
Studierenden die Hilfe zurückzugeben, die du von deinen Tutoren
bekommen hast – oder sogar noch bessere! Du erhältst die Möglichkeit,
durch die Tutorentätigkeit direkten Einblick in eine Professur zu
gewinnen, und die Chance, einen Gedanken daran zu verschwenden, die
wissenschaftliche Karriere als beruflichen Weg ins Auge zu fassen. Und
nicht zuletzt macht es auch eine Menge Spaß, Tutor zu sein. Wirklich!


\item[Umzug] Die Universität plant, den Standort Bockenheim
aufzugeben. Aktuell ist vorgesehen, dass die Informatik quasi ''das
Licht ausmacht'' und 2017 als letzter Fachbereich den Campus
Bockenheim verlässt. (Genaues weiß man aber nicht, da schon in der 3. Sitzung der FBR (in den 80ern) über den ''Umzug an den Niederurseler Hang'' geschprochen wurde.)
Danach soll sie am \textbf{Campus Riedberg}
residieren.% Wo wir dann genau hinkommen weiß noch keiner so genau.
Wer nicht gerade Mathematik oder ein sogenanntes
gesellschaftswissenschaftliches Fach als Anwendungsfach studieren
möchte, wird schon jetzt in den Genuss des Reisens kommen: die anderen
Fächer sitzen mittlerweile am \textbf{Campus Westend} oder am
\textbf{Campus Riedberg}.
%Wer sein Studium zügig betreibt, wird den
%Umzug aber aller Voraussicht nach nicht mehr miterleben.


\item[Verwaltungsgebühren] Nach Verkündigung durch den hessischen
Ministerpräsidenten wurde ''Verwaltungskostenbeitrag'' in Höhe von 50,00
\euro{} pro Student eingeführt.
% Leider kommt das Geld nicht den Universitäten zu gute.
%; für dieses und das nächste Jahr sindStellenkürzungen so gut wie
%sicher.

\item[Wahlen] Irgendwann wird ein Brief zu euch nach hause kommen, in dem ihr
ein paar Kreuze machen könnt, um ihn danach wieder in einem Briefkasten zu
werfen. Und wenn ihr das gemacht habt, habt ihr gewählt. Denn damit wir uns auch
weiterhin für euch mit Ämtern ärgern, mit Professoren prügeln und mit Dekanen
duellieren können, brauchen wir den Rückhalt der Studenten. Dazu könnt ihr uns
in verschiedene Räte wählen, in denen wir eure Interessen vertreten. Ihr könnt
euch natürlich auch selber wählen lassen. Wenn ihr Fragen dazu habt, kommt
einfach zu uns.


%\item[VV] Vollversammlung. VV ist, wenn der AStA oder die Fachschaft
%einlädt, um über besonders wichtige Dinge zu diskutieren und
%Entscheidungen zu treffen. Zur Teilnehmerrate sei kurz erwähnt, dass
%zur letzten vom AStA einberufenen VV mit dem Thema
%„Studiengebühren/Streik“ zwar mehr als 40 000 Studierende eingeladen
%waren, sie dann aber trotzdem auf dem Campus stattfinden konnte.


\item[Westend, Campus] Ein jüngerer Campus der Frankfurter
Universität. Im früheren Gebäude der IG-Farben (Poelzig-Bau) sind die
Sprach- und Geisteswissenschaftler untergebracht. In den jüngst
entstandenen, protzigen Neubauten residieren seit 2008 die
Wirtschaftswissenschaftler und Juristen. Außerdem findet man hier
neben Bäumen und diversen Grünflächen auch die bessere Mensa. Zu
erreichen von Bockenheim aus per Bus mit Linie 36 oder 75, oder mit
zusätzlichem 5-minütigen Fußmarsch durch die U-Bahn-Linien 1, 2, 3 und
8 (Station \textit{Holzhausenstraße}).


\item[WiMis] Wissenschaftliche Mitarbeiter. Sie sind an einer
Professur angestellt, um dort zu promovieren, also um irgendwann
einmal einen Doktortitel zu erhalten. Bis dahin forschen sie fleissig
und unterstützen den Professor bei Lehrveranstaltungen. Gerade in
Übungen werdet ihr ab und zu mit einem WiMi zu tun haben.

\end{description}

\begin{flushright}Überarbeitung des Glossars durch folgende Fachschaftler: Sebastian, 
Pavel, Jonathan und Johannes\end{flushright}

