%\vspace{5mm}
\section*{\begin{turn}{10} \textbf{WILLKOMMEN!} \end{turn} }
%\vspace{5mm}
Ei Guude!\\

Da seit ihr also. Die Neuen. Und ihr seit hier aus einem Grund. Einem spezifischen Grund. Ihr wollt was lernen.
Und das werdet ihr. An der Uni lernt ihr, so wie wir das gelernt haben,
\"uber jegliche Form der Ahnungslosigkeit hinwegzut\"auschen. Und wir helfen euch dabei!\\

Daher pr\"asentiert die Fachschaft Informatik euch voller Stolz und Selbstverherrlichung die \emph{Don't Panic! 42}.

Und damit fangen wir auch mit der ersten Lektion in Sachen Hochschulkultur an: Selbstbeweihr\"aucherung.
Damit ihr ganz genau wisst, auf wen ihr die Schuld schieben k\"onnt sind hier alle, die an der Don't Panic! mitgearbeitet haben:
\begin{itemize}
	\item Alexandra Herrmann - Kein Typ
	\item Jonathan Cyriax Brast - So ein Typ
	\item Joshua Sole - Noch ein Typ
	\item Linda Homeier - Typin
	\item Simon Pruy - Auch ein Typ
	\item Wiebke Belitz-Hellwich - Auch kein Typ
	\item Johannes Schöpp - Komischer Typ
	\item \LaTeX - stark typisiert
	\item Und aus historischen Gr\"unden all unsere Vorg\"anger:
	Pavel Safre, Grzegorz Lato, Markus Palcer, Tim F\"oller, Sabrina Brandt, Markus Strobel, Michael Bals, Christoph Burschka, Sebastian Behr und Igor Geier.
\end{itemize}


Außerdem möchten wir xkcd (\url{http://xkcd.com}) für die tollen Comics danken.

Es war viel Arbeit. Wir hoffen ihr wisst das zu schätzen.
Ansonsten wünschen wir euch jetzt aber einen guten Einstieg in euer Studium!

\begin{flushright}Die Fachschaft\end{flushright}

\section{Impressum}
\spaltenanfang
\textbf{Studentische Vertretung der Lehreinheit Informatik der Johann Wolfgang Goethe-Universität Frankfurt am Main}\\
Robert-Mayer-Straße 11 - 15\\
D-60325 Frankfurt am Main\\
\url{www.fsinf-frankfurt.de}\\
\url{www.fsinf-forum.de}\\
\emailfachschaft\\
%~\newline
Don‘t Panic! 42\\
OE SS 2015\\
Oft überarbeitete und erweiterte Auflage\\
Version der Ausgabe: 2.0\\
April 2015\\
Erscheinungsweise: jedes Semester\\
Auflage: 250\\
Druck und Bindung: HRZ Druckzentrum

\spaltenende
