Damit du dein Studium auch erfolgreich beenden kannst, musst du irgendwann auch Prüfungen ablegen.
Du wirst warscheinlich schon im ersten Semester deine ersten Klausuren schreiben wollen, um die Veranstaltungen abzuschließen.
Im Bachelorstudiengang wird zwischen zwei verschiedenen Prüfungen unterschieden: \emph{Prüfungsleistungen} und \emph{Studienleistungen}.
Die erste Frage, die sich stellt ist, was denn der Unterschied zwischen beidem ist.
Nun, im ersten Semester wirst du den als einzigen Unterschied spüren, dass du für die Studienleistungen kein Formular ausfüllen musst während du für Prüfungsleistungen ins Prüfungsamt rennen musst.
Im zweiten Semester wirst du merken, dass du für \emph{Grundlagen der Programmierung} und die beiden Technikveranstaltungen noch einmal eine mündliche Prüfung ablegen musst.
Warum das?

Nun, wenn du dir die Modulübersicht ansiehst, wirst du merken, dass \emph{PRG-1} und \emph{PRG-2} zusammen ein Modul bilden.
Das gleiche gilt für die Paare \emph{EDGI} und \emph{HWR} sowie \emph{GL-1} und \emph{GL-2}.
In allen Veranstaltungen außer diesen sechs ist die Prüfung am Ende des Semesters eine \emph{Prüfungsleistung}, die auch \emph{Modulabschlussprüfung} genannt wird.
Dieser Name macht deutlicher, was gemeint ist:
Die Prüfung schließt das Modul komplett ab und die Note wird später in deine Bachelornote verrechnet.
In den sechs oben genannten Veranstaltungen ist die Prüfung nicht eine Modulabschlussprüfung, da das Modul ja aus beiden Veranstaltungen besteht.
Mit der Klausur hier zeigst du lediglich, dass du den Stoff aus dem Semester verstanden hast.
Um diese ``Doppelmodule'' abzuschließen musst du eine zusätzliche mündliche Prüfung machen.
Diese mündliche Prüfung ist dann die Modulabschlussprüfung, deren Note in deinen Bachelor zählt.
Um zu der mündlichen Prüfung zugelassen zu sein, musst du aber mindestens eine der Klausuren bestanden haben.

Nun habe ich sehr viele verwirrende Dinge erzählt, die du erst einmal verdauen musst, aber \textbf{Don't Panic}: so schwer, wie es klingt ist das nicht.
Spielen wir mal eine Prüfungsanmeldung durch.
Gehen wir mal davon aus, dass du im ersten Semester zwei Prüfungen machen möchtest - sagen wir einfach mal in Programmieren und Mathe.
Aus der Modulübersicht oder der Liste oben kannst du entnehmen, dass Programmieren eines der Doppelmodule ist, für die du eine Studienleistung schreibst.
Hier brauchst du dich also nicht im Prüfungsamt anzumelden.
Das entsprechende Formular findest du direkt am Eingang ins Prüfungsamt.
Meist möchten die Profs aber wissen, wie viele Studenten zur Klausur zu erwarten sind, also solltest du dich in die Liste eintragen/online eintragen oder welches Verfahren der Prof verwendet, um das herauszufinden.
In Mathe (oder genauer: \emph{Analysis und Lineare Algebra für Informatiker}) wirst du eine Modulabschlussprüfung schreiben, für die du dich rechtzeitig (=mindestens 4 Wochen vorher) im Prüfungsamt anmelden musst.
Du gehst also zum Prüfungsamt, um dich anzumelden. 
Nach einer kleinen Wartezeit bist du dann endlich an der Reihe und betrittst das Prüfungsamt, um dein Formular abzugeben und dein Gewissen beruhigen zu können, dass du eine Sache weniger vergessen kannst.
Dann kommt der Super-Gau:
``Haben sie schon das Prüfungsverfahren eröffnet?'', fragt die Dame im Prüfungsamt.

Jaaa - hier geht es los.
Bevor du dich zu deiner ersten Prüfung anmelden kannst, musst du bereits den Antrag stellen, dass du die Bachelorprüfung machen darfst.
Das entsprechende Formular findest du direkt am Eingang ins Prüfungsamt.
Dort musst du unter anderem angeben, ob du bereits in einem ähnlichen Studiengang immatrikuliert warst und ob du von diesem Studium irgendetwas annerkannt bekommen möchtest.
\textbf{Don't Panic}!
Du kannst das Formular aber auch mit deiner ersten Anmeldung zu Modulprüfungen abgeben und meist nimmt die Dame im Prüfungsamt deine Anmeldungen auch schon entgegen, bevor du dein Prüfungsverfahren eröffnet hast. 
Du musst das nur vor den Prüfungen noch erledigen - frag einfach nach, wenn es hier haken solle.

Springen wir ein paar Wochen in die Zukunft. 
Die Prüfungen sind geschrieben und du erwartest gespannt die Ergebnisse.
Täglich grast du die Homepages ab, die von den Profs extra für die Veranstaltungen eingerichtet wurden und endlich steht das Ergebnis auch online.
Wie zu erwarten hast du in Programmieren gerockt, nur in Mathe hat es nicht ganz so gereicht - verdammt!
Was jetzt?
Zuerst - wie immer: \textbf{Don't Panic}!
Andersherum wäre es zwar etwas weniger schlimm gewesen, da du Studienleistungen ja beliebig oft wiederholen kannst, aber...
Ja, natürlich.
Du hast richtig gehört.
\textbf{Die Studienleistungen} in den sechs oben genannten Vorlesungen \textbf{kannst du so oft wiederholen wie du willst}, ohne dass dir Nachteile entstehen.
Die Modulabschlussprüfungen, wie Mathe eine war, allerdings nicht.
Hier hast du für jedes Modul drei Versuche.
Erst, wenn du beim dritten Versuch (man nennt so etwas eine \emph{finale Prüfung}) durchfällst, dann bekommst du Probleme.
Welche Probleme?
Nun die Studienordnung verbietet dir, diese Prüfung erneut abzulegen.
Wenn du die Prüfung nicht für den Abschluss des Studiums benötigst, wie das zum Beispiel in den Vertiefungsmodulen der Fall ist, dann ist das nicht weiter tragisch.
Du musst lediglich ein anderes Vertiefungsmodul als Ersatz finden und schaffen.
Handelt es sich aber um ein Basismodul, ohne die du den Bachelor ja nicht schaffen kannst, dann ist dein Studium hier zuende.
Da du die Bedingungen zur Bachelorprüfung nicht mehr erfüllen kannst, wirst du automatisch exmatrikuliert.
Aber das heißt nicht, dass dein Studium gefährdet ist, nur weil du einmal Mathe nicht bestanden hast.

Springen wir mal ein halbes Jahr weiter, an das Ende des zweiten Semesters.
Hier hast du Datenstrukturen, Programmieren 2, Hardware und Mathe 2 gehört.
Gehen wir mal davon aus, dass du Programmieren 2 bestanden hast - die anderen Fächer interessieren uns jetzt erstmal nicht.
Nun hast du eine der Studienleistungen in \emph{B-PRG} bestanden und damit die Vorraussetzung, dieses Modul abzuschließen.
Du lässt dir also von einem der Profs, die du in den beiden Programmieren-Vorlesungen gesehen hast, einen Termin für eine mündliche Modulabschlussprüfung geben und meldest dich im Prüfungsamt an.
Ein bitterer Beigeschmack bleibt aber: 
Diese mündliche Prüfung hat beide Veranstaltungen zum Thema - auch die, deren Klausur du nicht bestanden hast.
Dennoch sollte die mündliche Prüfung für dich zu bewältigen sein.

Nun möchte ich noch ein paar Worte über Mathe 2 verlieren.
Außer AnaLinA, welches den Modulnamen \emph{M1} hat, hast du noch vier andere Mathemodule, deren Namen in der Form \emph{M2x} sind, wobei x für einen Buchstaben von a bis d steht.
Mathe 2 ist M2d und eine Einführung in die anderen drei.
Die Regelung für diese Module ist, dass du von den vier angebotenen zwei bestehen musst.
Also auch wenn alle vier als Basismodule zählen, brauchst du nur zwei von ihnen, um deinen Bachelor abschließen zu können.

Zum Schluss noch eine Warnung:
Wenn du nun denkst, du fährst sicherer, wenn du am Anfang keine Prüfungen ablegst (da duch dann auch nicht durchfallen kannst) dann kannst du immernoch Probleme bekommen.
\textbf{Du musst in den ersten drei Semestern mindestens zwei Module aus der folgenden Liste bestanden haben:}
\begin{itemize}
	\item Diskrete Modellierung
	\item Datenstrukturen
	\item Grundlagen der Programmierung
	\item Hardware
	\item Analysis und Lineare Algebra für Informatiker
\end{itemize}
Hast du das nicht, dann wird dir Post ins Haus flattern: 
Du musst eine Studienberatung besuchen, wenn du nicht rausfliegen möchtest.
Die Studienberatung ist ein Gespräch, in dem deine Zukunft in dem Studium besprochen wird. 
Du verlässt die Studienberatung in der Regel mit dem Zwang, bestimmte Prüfungen zu bestehen.
Diese Studienberatung steht auch immer dann an, wenn du vier Semester lang keine einzige Prüfung durchgeführt hast.

\begin{flushright}
	Markus
\end{flushright}