%Dass die Staatskassen leer sind, hat inzwischen auch die hessische Landesregierung mitbekommen, und hatte im Wintersemester 2006/07 in einer Hau-Ruck-Aktion überraschend und anscheinend ohne weiter darüber nachzudenken allgemeine Studiengebühren eingeführt. Zum Sommersemester 2008 wurden diese zusammen mit den bereits seit 2003 bestehenden Langzeitstudiengebühren wieder abgeschafft. Da dies für euch somit erst mal kein Thema ist, werden wir sie hier auch nicht behandeln. Schauen wir aber mal, was trotzdem an Kosten auf euch zukommt:

%~\newline
Eine genaue Aufschlüsselung der Semester Gebühren findet sich unter: https://www.uni-frankfurt.de/38935069/semesterbeitrag\\
Zum einen gibt es so genannte „Verwaltungsgebühren“ von 50 \euro{}. Diese sind in allen Semestern zu zahlen und werden gleich mit der Rückmeldegebühr berechnet. Von den 344,98 \euro{}, die du bezahlt hast, um hier studieren zu dürfen, sind also 50 \euro{} direkt in ein schwarzes Loch in Wiesbaden gesaugt worden (vom Rest geht der größte Teil übrigens für das Semesterticket drauf). In sozialen Härtefällen kannst du dir diese 50 \euro{}, die eigentlich jeder zu bezahlen hat, (teilweise) erlassen oder \textit{stunden} lassen. Stunden heißt, dass die Gebühren zu einem späteren Zeitpunkt fällig werden. Gleiches gilt auch bei Kindererziehung, Verwandten-Pflege oder als Opfer einer schweren Straftat.

Wenn du (mit Recht) von den ausführlichen und ungenauen Regelungen verwirrt bist, empfehlen wir dir, es im Zweifelsfall nach einer Beratung (zum Beispiel bei der Fachschaft, beim AstA oder im StuGuG-Referat) einfach mal zu probieren und dich vom Ergebnis überraschen zu lassen. Widerspruch einlegen kannst du hinterher immer noch. Um einen solchen eventuellen Nachlass in Anspruch zu nehmen, reicht es einen formlosen Antrag beim Studentensekretariat zu stellen. Wahrscheinlich wird von dir dann auch eine Selbstauskunft über deine finanzielle Situation verlangt. Beispiele für solch eine Selbstauskunft und so einen „formlosen Antrag“ findest du unter: http://www.asta.uni-frankfurt.de/informativ/gebuehrenberatung/

Auch wenn du dir jetzt noch sicher bist, es nicht zu brauchen: Sieh es als eine Versicherung für den Fall der Fälle, dass sich die Dinge nicht so entwickeln, wie du hoffst.


~\newline
Wenn du zwischen 12 und 24 Stunden pro Woche arbeitest, kannst du ein Teilzeitstudium beantragen. Das Studium läuft dann weiter wie gehabt, aber für zwei Semester im Teilzeitstudium wird dir nur ein Semester berechnet. Dies kann sich als nützlich erweisen, wenn eine zeitintensive Arbeit für dich zwingend erforderlich ist. Wichtig ist, dass du das Teilzeitstudium im laufenden Semester beantragst! Beachte jedoch, dass ein Teilzeitstudium Auswirkungen auf Krankenversicherung und BAföG haben kann. Nähere Informationen \textsl{und} ein entsprechendes Antragsformular hierzu findest du unter: http://www.uni-frankfurt.de/studium/verwaltung/teilzeitstudium/


Eine andere Möglichkeit sind Urlaubssemester, in denen aber keinerlei Prüfungsleistungen (auch Studienleistungen) abgelegt werden können. Ein solches Urlaubssemester kannst du zum Beispiel beantragen, wenn du aus gesundheitlichen Gründen nicht studieren kannst. Wieder gilt hier: das muss im laufenden Semester beantragt werden, später geht das in der Regel nicht mehr. Näheres hierzu findest du unter http://www.uni-frankfurt.de/studium/verwaltung/beurlaubung/


Und jetzt noch ein guter Tipp zum Schluss: Die Erfahrung zeigt, dass manch einer feststellen wird, dass Informatik nicht das richtige für ihn ist. Vielleicht ja auch du. Die meisten wechseln dann einfach ihr Fach. Leider merken sie das aber meist erst im dritten oder vierten Semester. Es ist zwar sicherlich sinnvoller, den Studiengang zu wechseln, statt zu versuchen, sich durch das falsche Fach zu quälen. Bitte überlege dir einen Fachwechsel aber sehr gut: so manches Fach, dass zu Studienbeginn noch als unüberwindbare Hürde erscheint, wirkt rückblickend überhaupt nicht mehr schlimm! Solltest du noch irgendwelche Fragen haben, so kannst du uns jederzeit ansprechen oder uns eine Email schicken: \emailfachschaft


\begin{flushright} Alexander, updated by Sebastian \end{flushright}
