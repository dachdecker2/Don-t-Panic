Irgendwann im Verlauf des ersten Semesters werdet ihr euch für die ersten Prüfungen anmelden. Damit ihr nicht in Zeitnot bei der Anmeldung geratet, hier ein kleiner Überblick wie ihr euch für welche Prüfungen anmeldet.

Schriftliche Modulabschlussprüfungen: Im ersten Semester werden die Modulabschlussprüfungen, die ihr machen werdet, wohl B-M1 (Lineare Algebra und Analysis für Informatiker) und B-MOD sein (Diskrete Modellierung). Allerspätestens vier Wochen vor dem Klausurtermin müsst ihr im Prüfungsamt den entsprechenden Zettel ausgefüllt abgeben. Da im Prüfungsamt der Betrieb kurz vor den Klausuren ziemlich hoch ist, sollte man sich früh genug um die Anmeldung kümmern und nicht erst am letztmöglichen Tag. Damit ihr aber überhaupt an eurer ersten Modulabschlussprüfung teilnehmen könnt, müsst ihr zuvor die Zulassung zur Bachelorprüfung beantragen. Zulassung zur Bachelorprüfung: Folgendes müsst ihr hierfür im Prüfungsamt abgeben:
\begin{itemize}
\item Immatrikulationsbescheinigung
\item Schriftliche Erklärung zur evtl. Immatrikulation in verwandten Studiengängen.
%\item Nachweis über die Zahlung der Prüfungsgebühr ( 75 \euro{})
\item Nachweise zu Studien- und Prüfungsleistungen für die eine Anrechnung begehrt wird
\item ggf. Erklärung, falls man einen Nachteilsausgleich gemäß § 21 in Anspruch nehmen will
\end{itemize}
Anmeldung für Studienleistungen: Im ersten Semester werdet ihr aller Wahrscheinlichkeit nach mit den Klausuren in PRG-1 und EDGI Studienleistungen erwerben, wenn ihr eben diese besteht. Die Informationen für die Anmeldung zur jeweiligen Klausur werden in der jeweiligen Vorlesung bekannt gegeben. Anmeldung zu einer mündlichen Modulabschlussprüfung: Für die mündliche Modulabschlussprüfung meldet ihr euch genauso an, wie für die schriftliche Modulabschlussprüfung. Also vier Wochen vorher das entsprechende Formular ausgefüllt abgeben. Damit ihr euch aber für eine mündliche Modulabschlussprüfung anmelden könnt, müsst ihr euch zunächst einen Termin beim dem Professor, der euch prüfen soll, holen. Dieser Professor muss euch dann auch das Formular unterschreiben. In Modulen, in denen die Vorlesung von mehreren Professoren gehalten wurde, wie z.B. in B-PRG könnt ihr euch einen Prüfer aussuchen. Da auch Professoren manchmal Urlaub machen, solltet ihr euch die Prüfungstermine rechtzeitig besorgen. Wenn ihr all diese Hürden für die Anmeldung gemeistert habt, bleibt nur noch eins: Viel Erfolg und Don‘t Panic!

\begin{flushright}Tim und Claudia\end{flushright}
