Wenn man am ersten Montag in seine erste Vorlesung geht, ist es wohl für die
meisten ein ungewohntes Gefühl. Gerade von der Schule, aus einem freiwilligen
sozialen Jahr oder aus einer Ausbildung gekommen, kommt es einem sehr komisch
vor, in einem relativ großen Hörsaal zu sitzen und sich den Vortrag eines
Professors anzuhören. Vor allem macht man sich bei einigen Veranstaltungen sehr
schnell Gedanken, warum man sich so früh in die Uni begibt, wenn man sich den
Inhalt der Vorlesung auch im Internet herunterladen und selbst lesen könnte. 
Sicherlich mag ein gemütliches Bett sehr verlockend sein, aber die Erfahrung
zeigt, dass der Lerneffekt wesentlich höher ist, wenn man seinen inneren
Schweinehund überwindet und regelmäßig in die Veranstaltungen geht und auch
versucht, den Stoff richtig nachzubereiten.  Die Zwanglosigkeit, die das
universitäre Studium mit sich bringt, stellt wohl jeden von uns anfangs vor die
Aufgabe zu lernen, sich selbstverantwortlich durch sein Studium zu kämpfen. Man
wird vor viele Fragen gestellt. Welche Veranstaltungen soll ich besuchen, mache
ich lieber mehr davon, oder ist es vielleicht besser, am Anfang nicht das
Maximum an Veranstaltungen zu besuchen und sich stattdessen lieber ein wenig
ausführlicher mit dem Stoff der anderen Fächer zu beschäftigen? Welche der
Übungen soll ich besuchen, wie passt das alles zusammen und wie schaffe ich es,
alles so zu legen, dass es für mich optimal ist? Und was ist überhaupt optimal
für mich?


Nun, diese Frage könnt nur ihr selbst beantworten. Wir können euch Tipps und
Anregungen geben und bieten auch gerne als Fachschaft an, euch bei
Problemen und Fragen hilfreich zur Seite zu stehen. Ein Rat von mir, nehmt
dieses Angebot an, wenn es nötig ist, ansonsten kann ich fürs erste nur sagen:
„Don’t Panic!“ Ihr werdet sehr bald sehen, dass man sich sehr leicht daran
gewöhnt, all dies zu „händeln“.


Wahrscheinlich werden viele von euch auch dazu gezwungen sein, neben dem
Studium zu arbeiten, sei es, weil es unbedingt nötig ist, um auch noch morgen
ein Dach über dem Kopf zu haben oder weil einfach nur ein lukrativer
Nebenverdienst lockt. Dies ist auch überhaupt nicht verwerflich, sondern kann
sogar sehr förderlich sein. Mein Tipp ist nur, vernachlässigt euer Studium
nicht allzu sehr, sondern versucht, möglichst die aussichtsreichen Chancen, die
man in Zukunft als Informatiker haben kann, zu nutzen.


Einige von euch werden wohl auch relativ schnell enttäuscht sein, weil das
Studium überhaupt nicht ihren Vorstellungen davon gerecht werden will. Auch an
dieser Stelle kann ich nur sagen: \textbf{Versucht euch durchzubeißen}. In
Einzelfällen kann auch die Überlegung sinnvoll sein, dass man hier nicht ganz
richtig ist. Dies ist kein Beinbruch und ihr wärt auch nicht die Ersten, die
die Entscheidung zu einem Studienwechsel treffen. Aber bitte überlegt es euch
gut und werft nicht gleich in den ersten Wochen die Flinte ins Korn, ärgert
euch aber auch nicht euer restliches Leben, weil ihr beim Einschreiben die
falsche Entscheidung getroffen habt.


Was ihr natürlich auch nicht vergessen solltet, ist die Tatsache, dass alleine
zu studieren keinen Spaß macht. Für die wenigsten von euch wird es zutreffen,
dass einsames Lernen zu Hause im stillen Kämmerlein effizienter ist als das
Lernen in der Gruppe. Versucht, euch mit einigen Leuten zusammenzufinden und
mit ihnen die Untiefen des Studiums zu meistern. Das Lösen der Übungsaufgaben
fällt einem in einer Gruppe natürlich wesentlich einfacher, wenn man sich
gemeinsam hinsetzt (damit ist aber nicht das kollektive Suchen nach Vorlagen
zum Abschreiben gemeint!), aber auch organisatorische Probleme lassen sich
leichter lösen. Einer alleine übersieht gerne mal einen Aushang oder eine
Liste, in die man sich eintragen muss oder kann sich zu einem bestimmten
Zeitpunkt nicht selbst eintragen. Außerdem kann eine Gruppe sehr motivierend
sein, wenn man sich gegenseitig zur Sau macht, weil man vielleicht mal die
anderen im Stich gelassen und sich viel lieber um 8 Uhr noch mal gemütlich im
warmen Bett umgedreht hat.


Einen ersten Schritt in diese Richtung sollt ihr während der OE machen, die,
außer euch wichtige Fakten zum Studium zu vermitteln, vor allem dafür da ist,
die Leute aus eurem Semester kennen zu lernen.
