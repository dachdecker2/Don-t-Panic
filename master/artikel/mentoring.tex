Bereits \textbf{im ersten Semester} wird den Bachelor-Studierenden die Teilnahme an der Studiumsorientierungveranstaltung “Einführung in das Studium” (STO) empfohlen. Diese Veranstaltung ist der Pflichtanteil des Ergänzungsmoduls im \textbf{Bachelor Informatik} und \textbf{muss} aus diesem Grund besucht werden. Dafür gibt es 2 CPs.

Da STO gar nicht anstrengend ist, \textbf{keine Abschlussprüfung} enthält und das Studium sehr stark erleichtert, ist die Teilnahme daran bereits im ersten Semester sehr sinnvoll.

Die Veranstaltung besteht aus dem Vorlesungsanteil, wo euch eher allgemein erzählt wird, wie man am effizientesten studiert und dem sogenannten Peer-Mentoring.

Der Grundgedanke des Peer-Mentorings ist es, euch den Einstieg ins Studium zu erleichtern, denn oftmals sind viele Erstsemester überfordert, wissen einfach nicht wo sie welche Informationen herbekommen oder trauen sich einfach nicht zu fragen. Oftmals handelt es sich dabei um die einfachsten Dinge.

Ihr werdet am Anfang eures Studiums in mehrere Kleingruppen unterteilt. Diese bestehen aus jeweils etwa 10 Studenten (Mentees) und einem erfahrenen Studenten aus einem höheren Semester (Mentor).

\begin{center}
	\includegraphics[scale=1.0]{comics/groups-meeting-light-icon}
\end{center}

Im Laufe des Semesters wird sich jedes Mentorium etwa alle 2-3 Wochen einmal treffen. Dort werden euch dann alle Fragen rund ums Studium beantwortet. Außerdem steht es euch jederzeit frei euren Mentoren einfach eine E-Mail zu schreiben. Sie werden euch jederzeit gerne helfen.

Das Mentoring soll euch hierbei auch schon eine weitere Möglichkeit geben, um in kleinen Kreisen neue Leute kennen zu lernen und gegebenenfalls Lerngruppen zu bilden, denn ihr werdet mit der Zeit bemerken, dass Studium und Schule sich sehr unterscheiden und man hier allein nicht alles schaffen kann.

Ihr werdet lernen, dass es hier viel wichtiger als in der Schule ist, eigenverantwortlich zu arbeiten und selbstständig zu werden, denn ihr habt im Studiengang Informatik in den meisten Fällen keine Anwesenheitspflicht. Auch die Übungen die ihr regelmäßig in den Vorlesungen ausgeteilt bekommt, solltet ihr alle selbst machen. Keiner wird sich darum kümmern, ob ihr sie macht oder nicht.

\begin{center}
	\includegraphics[scale=0.8]{comics/study-icon}
\end{center}

Wie Vieles nun ganz genau abläuft und geregelt ist, was ihr genau wann und wo zu tun habt,
bzw. wann und wie ihr euch für irgendwelche Prüfungen anmelden könnt oder wie ihr euer Studium generell plant und alles was für euch noch so wichtig ist, erfahrt ihr im Mentorium.

Das Mentorium behandelt also im allgemeinen organisatorische Themen, aber ihr könnt das Ganze auch als Kummerkasten betrachten. Was im Mentorium besprochen wird, bleibt auch im Mentorium. Also ist euer Mentor schonmal eine Art Vertrauensperson, die versuchen wird, auch individuell auf euch einzugehen, denn jeder Mensch geht anders mit den Begebenheiten um, wie man lernt und wie man seine Zeit plant. Dabei wird euer Mentor jedoch nicht sagen wie ihr zu lernen habt oder eure Zeit einteilen müsst, sondern wird euch lediglich helfen, den für euch richtigen Weg oder Kompromiss selbst zu finden.

\begin{flushright}Markus S., modified by Pavel \end{flushright}