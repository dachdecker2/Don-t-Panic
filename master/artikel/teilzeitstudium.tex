\label{teilzeitstudium}

Bevor man sich für ein Teilzeitstudium entscheidet, sollte man 3 Dinge tun:
\begin{enumerate}
\setlength{\parskip}{3pt}
	\item Diesen Beitrag lesen, und wenn das Teilzeitstudium dann noch in Frage kommt,
	\item Die betreffenden Stellen in der Teilzeit-Verordnung und 
	\item Studienordnung lesen.
\end{enumerate}

Das Teilzeitstudium an den Hochschulen des Landes Hessen ist zum 01. April 2010 neu geregelt worden. Grundständige Studiengänge können auch im Teilzeitstudium absolviert werden, wenn die Prüfungsordnung des Studiengangs, dies nicht ausschließt und für das entsprechende Fachsemester keine Zulassungsbeschränkungen bestehen.

Diese und teilweise die folgenden Bemerkungen sowie eventuelle Aktualisierungen sind im WWW zu finden.
\footnote{
 Unter:

\url{http://www2.uni-frankfurt.de/35793994/teilzeitstudium}

\url{http://www.cs.uni-frankfurt.de/images/pdf/informatik/bachelor2/bachelorordnung\_neu.pdf}

\url{http://goo.gl/TG0Wb}
}


\textbf{Warum ein Teilzeitstudium?} \newline
Auszug aus einem Informationsblatt der Uni Frankfurt:

\emph{``\ldots
Begründung für ein Teilzeitstudium: \linebreak
- Berufstätigkeit (auch selbständige Tätigkeit) mit einer wöchentlichen durchschnittlichen Arbeitszeit 14-28 Stunden für die Dauer von mind. 2 Semestern an Antragstellung. (Aktuelle Nachweise, wie Arbeitsbescheinigungen, Arbeitsverträge etc.)\linebreak
- Betreuung eines Kindes unter 10 Jahren, das im gleichen Haushalt lebt. (Geburtsbescheinigung.) \linebreak
- Pflege eines nahen Angehörigen. (Bescheinigung über die Pflegebedürftigkeit mit Zuordnung zur Pflegestufe, sowie amtlicher Nachweis über die Bestellung zur/zu Pfleger/in.) \linebreak
- Behinderung oder chronische Erkrankung. (Nachweis)\linebreak
- Zugehörigkeit zu einem A-, B- oder C-Kader oder vergleichbaren Förderstrukturen eines nationalen Spitzensportverbandes in den olympischen oder paralympischen Sportarten (Nachweis)\linebreak
- Aus einem anderen wichtigen Grund. (Bitte auf gesondertem Blatt begründen und ggf. belegen. \ldots{}''}


Wer also aufgrund persönlicher Verpflichtungen daran gehindert wird, ein Vollzeitstudium zu absolvieren, aber die notwendigen Fähigkeiten mitbringt, soll nicht benachteiligt werden und die Möglichkeit erhalten, einen Studienabschluss zu erhalten. Grundsätzlich gilt, dass jeder Grund gegenüber der Universität glaubhaft nachgewiesen werden muss.


\textbf{Fristen} \newline
Gibt es (wie immer) auch hier: Anträge für ein Sommersemester müssen bis 01. Mai, für ein Wintersemester bis zum 01. November eingereicht werden. 


\textbf{Einschränkung} \newline
Bei einem Wiederholungsantrag ist ein angemessener Studienfortschritt nachzuweisen (Bescheinigung Prüfungsamt). Bei modularisierten Studiengängen ist darüber hinaus ein Nachweis (Bescheinigung Prüfungsamt) erforderlich, dass während des vorangegangenen Teilzeitstudiums nicht mehr als 50\% der im Vollzeitstudium vorgesehenen Kreditpunkte oder Leistungsnachweise erworben wurden. 

Was heißt das?
Wenn man als Teilzeitstudent zu schnell studiert (oder aber eigentlich Vollzeitstudent ist) kann der Teilzeitstatus seitens der Uni wieder aberkannt werden. Damit gelten auch wieder kürzere Fristen für zu erbringende Leistungen (siehe unten).

\vspace{5mm}
\textbf{Beantragung} \newline
Ein Teilzeitstudium muss separat im Studentenservicecenter beantragt werden. Es gibt hierfür extra Formulare (entweder vor Ort erhältlich oder im Internet unter der oben angegebenen Adresse), die ausgefüllt und mit den entsprechenden Nachweisen wieder eingereicht werden müssen. Über den Antrag wird dann separat entschieden.

Ein Antrag gilt jeweils für 2 Semester. D.h. wenn man im Wintersemester anfängt und ein Teilzeitstudium beantragt, gilt der Antrag für dieses Wintersemester und das darauf folgende Sommersemester.

Vor der Beantragung ist  eine Fachstudienberatung mit dem zuständigen Professor durchzuführen und nachzuweisen.


\textbf{Studienverlauf} \newline
Aufgrund der persönlichen zeitlichen Beschränkungen können Teilzeitstudenten in der Regel nicht an allen Veranstaltungen teilnehmen.

Das erklärt auch, warum für Teilzeitstudenten andere (= verlängerte) Fristen für Prüfungen gelten.
(Siehe hierzu auch § 29 I Ziffer 5:  'Endgültiges Nichtbestehen der Bachelorprüfung' in der Studienordnung)

Grundlage dafür ist §4 II.  Anbei ein gekürzter Auszug daraus:

\emph{„...werden jeweils zwei im Teilzeitstudium absolvierte Semester bei der Berechnung der Meldefristen für die erstmalige Erbringung einer Prüfungsleistung als ein Fachsemester gezählt.  ...“}

Teilzeitstudenten müssen genau die gleichen Leistungen erbringen wie Vollzeitstudenten. Der wesentliche Unterschied ist aber der verlängerte Zeitraum, in dem diese Leistungen erbracht werden können. 

Das Studium muss nicht komplett als Teilzeitstudium absolviert werden. Während des Studiums kann zwischen Voll- und Teilzeitstudium gewechselt werden.

\paragraph{Pros und Cons}
Vorteile eines Teilzeitstudiums:
\begin{itemize}
	\item Durch die großzügigere zeitliche Gestaltung des Studiums wird es Menschen, die zeitlich stark eingeschränkt sind, ermöglicht, einen Studienabschluss zu erlangen.
\end{itemize}

Nachteile eines Teilzeitstudiums:
\begin{itemize}
	\item Längere Studienzeit
 	\item Durch die Studiengestaltung, müssen Lerngruppen und Kommilitonen, die gerade die gleichen Fächer hören, ca. alle 2 Semester neu gefunden werden (Die Kommilitonen, die man im ersten Semester kennen lernt, überholen einen naturgemäß und diejenigen, die einen während des Studiums einholen, habe evtl. bereits feste Gruppen gebildet, in die es schwerer hineinzukommen sein könnte.)
\end{itemize}


\textbf{Tip:} Organisiert euch in der Fachschaft.

Dort gibt es nicht nur Arbeit, sondern auch wertvolle Infos von erfahrenen Studenten. Da man als Teilzeitstudent immer wieder aus den gebildeten Lerngruppen herausgerissen wird, ist es gut, einen konstanten Bekanntenkreis zu haben. Außerdem wissen die dortigen Kommilitonen (und vielleicht auch bald du?) durch ihre Arbeit in den unterschiedlichen Gremien vielleicht das eine oder andere, das für das Studium nützlich ist. Und vielleicht trifft man dort noch andere Teilzeitstudenten, die gerne einmal ihre  Erfahrungen mit 'gleichgesinnten' austauschen möchten.

Ich wünsche allen einen guten Studienstart und viel Erfolg!
\begin{flushright}
	Michael
\end{flushright} 
