Im folgenden werden wir verschiedene Veranstaltungs- und Prüfungsformen, die dir während deines Studiums über den Weg laufen können, kurz vorstellen, damit du dich darauf einstellen kannst, was alles auf dich zukommen kann. Die unten aufgezählten weiteren Kriterien können euch bei den Modulprüfungen begegnen. In jedem Fall müssen die genauen Kriterien für die zu erbringenden Leistungen spätestens in der ersten Veranstaltungswoche von dem Professor bekanntgeben werden, der die Veranstaltung hält.


\paragraph{Protokolle} In den Praktika musst du meistens wöchentlich eine oder ein paar Aufgaben lösen und diese sehr ordentlich und ausführlich protokollieren. Nur wer alle (oder fast alle – je nachdem) Versuchsprotokolle mit einem „ok“ gekennzeichnet bekommen hat, bekommt einen Praktikumsschein. Es kann aber auch möglich sein, dass zusätzlich noch mehr verlangt wird: Das Bestehen einer Klausur, das Teilnehmen an einem Seminar. Der Phantasie sind da kaum Grenzen gesetzt.


\paragraph{Hausarbeit} Hausarbeiten sind in der Informatik unüblich, in anderen Fächern wie zum Beispiel Pädagogik oder Politologie in Seminaren aber aufgrund der hohen Teilnehmerzahlen durchaus gebräuchlich. Eine Hausarbeit kannst du dir als einen mittelgroßen Aufsatz vorstellen, der nicht mit einem mündlichen Vortrag oder Referat verbunden ist. Und was „mittelgroß“ bedeutet, hängt unter anderem davon ab, wie lang oder wie kurz du ein bestimmtes Thema bearbeiten kannst. Im Zweifelsfall wird dir aber dein Veranstalter einen Tipp geben.


\paragraph{Referat} Referate sind in der Informatik ebenfalls unüblich, hier musst du in der Regel einen kurzen Vortrag über ein bestimmtes Thema halten, manchmal ist zusätzlich eine schriftliche Zusammenfassung zu leisten.


\paragraph{Seminar} Hier ist in der Regel eine schriftliche Ausarbeitung und ein Vortrag (mit anschließender Diskussion) zu einem bestimmten Thema zu leisten. Der Aufwand sollte nicht unterschätzt werden, da je nach Thema einiges an Literatur recherchiert und durchgearbeitet werden muss.

\paragraph{Vorlesung} Hier trägt der Veranstalter zu einem bestimmten Thema vor, dass dann auch in Übungen, falls angeboten, vertieft wird. Du kannst dir Vorlesungen als Frontalunterricht vorstellen, bei dem du zuhören darfst und solltest. Wenn du Glück hast, darfst du sogar ab und zu eine Frage stellen. Ganz engagierte Dozenten stellen sogar ab und zu eine Frage an die Hörer, um sicherzustellen, dass sie ihm auch gedanklich folgen. Vorlesungen an der Universität verlaufen aber oft sehr viel schneller als die Kurse in der Schule. Das heißt: \textbf{Passt du heute nicht auf, weißt du morgen schon nicht mehr, worum es überhaupt geht,} und aufzupassen fällt dir immer schwerer, und übermorgen gehst du vor lauter Frust schon gar nicht mehr hin. Die Übungszettel sind eine Möglichkeit für dich, sich mit dem Stoff der Vorlesung nochmal auseinanderzusetzen. Immerhin bist du nicht gezwungen, die Aufgaben allein zu lösen, sondern du darfst dich mit deiner Lerngruppe an die Zettel setzen, und mit mehreren Leuten hat man ja oft auch mehr Ideen. 


\paragraph{Mündliche Prüfung, „Fachgespräch“} Mündliche Fachgespräche werden oft in Vorlesungen Angeboten, wenn zu erwarten ist, dass sich nicht allzu viele Studis prüfen lassen werden. Ein 
solches Gespräch soll 20 bis 40 Minuten dauern. Am besten, du schaust dir vorher mal Prüfungsprotokolle an (beispielsweise auf der Homepage der Fachschaft angeboten), damit du einen Eindruck vom Prüfungsstil und den Lieblingsfragen des Prüfers bekommst.


\paragraph{Klausur} Eine Klausur ist in vielen Basismodulen vorgesehen. Sie dauert zwei bis vier Stunden, zum Bestehen musst du einen bestimmten Anteil der möglichen Punkte auch wirklich erreichen. Meistens liegt dieser Anteil so grob bei 50 \%, es kann je nach Fach und Schwierigkeit des Stoffs aber auch weniger sein!


\paragraph{Übungen} Jede Woche bekommst du einen Zettel mit Aufgaben, die du bis zur folgenden Woche lösen sollst. In den  „Tutorien“ oder „Übungen“ werden sie dann besprochen (dazu weiter unten noch mehr). Übungen sind etwas anders als Hausaufgaben. Einerseits solltest du sie zu Hause lösen. Andererseits bekommst du aber auch keine 6 eingetragen, wenn du sie nicht machst. Du musst dir noch nicht einmal eine gute Ausrede einfallen lassen.

Oft werden die letzten beiden Kriterien, Klausur und Übungen, kombiniert: Es gibt Übungsaufgaben, und die Punkte, die du damit erreichst, werden schon für die Klausur angerechnet; oder sie dienen dir nur zu deiner Vorbereitung auf die Klausur.


\paragraph{Tutoriumsleitung} Im Bachelor-Studiengang ist es möglich, durch das Leiten eines Tutoriums das Ergänzungsmodul zu komplettieren. Dadurch hast du nicht nur die Möglichkeit, das Institut für Informatik direkt zu unterstützen, sondern kannst außerdem noch wertvolle Erfahrungen im Bereich der Soft Skills sammeln.


\paragraph{Tutorien} Und wie ist das mit den Tutorien? Einmal in der Woche findet eine Veranstaltung bei eurem Tutor statt. Das ist auch ein Studi, der mit seinem Studium schon weiter ist und sich jetzt bereit erklärt hat, euch ein wenig beim Studieren zu helfen. Während eines solchen Tutoriums werdet ihr die Übungsaufgaben besprechen; meistens wird einer von euch eine Aufgabe vorrechnen. Das hat nichts damit zu tun, euch „vorführen“ zu wollen. Es sind aber zwei verschiedene Dinge, eine Lösung auf einen Zettel zu malen und eine Lösung anderen laut und mündlich, mit Hilfe einer Tafel, zu erklären. Die Möglichkeit, das zu üben, solltet ihr euch nicht entgehen lassen; ihr profitiert davon nicht nur für Seminarvorträge! Auch wenn ihr von eurem Tutor Punkte auf eure Lösungsversuche bekommt (oder viel zu oft nicht bekommt), ist es für euch keine Schande zuzugeben, dass ihr etwas nicht verstanden habt und euren Tutor danach zu fragen. Dafür sind Tutorien da!

Es ist wohl möglich, existierende Lösungen abzuschreiben und sich in den Tutorien wie eine Maus zu verkriechen und bloß nicht aufzufallen. Es haben auch schon Leute auf diese Weise einen Übungsschein (ohne Klausur) erhalten. Ihr wisst schon, gelernt will der Stoff trotzdem werden, denn in der mündlichen Prüfung gibt es kein Abschreiben und Verstecken.


Oft aber verlaufen Tutorien ganz anders. Der Tutor schreibt an und wirbelt herum, reißt sich ein Bein aus und redet sich ’nen Wolf, und alle Teilnehmer sitzen gelangweilt im Seminarraum, gucken an die Decke und bohren in der Nase. Was glaubt ihr, könnt ihr in einem solchen Tutorium lernen? Na? Wie sinnvoll und interessant und hilfreich ein Tutorium ist, ist zu einem großen Teil davon abhängig, wie die Teilnehmer es gestalten. Macht was draus!

\begin{flushright}Oliver und Sebastian\end{flushright}