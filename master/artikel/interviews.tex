
Du verbringst viel Zeit vor deinem Computer und bist Experte im Umgang mit den neuesten Produkten aus dem Hause MS? Du kennst dich bestens aus im WWW und streifst Nachts leidenschaftlich von Chatroom zu Chatroom? Als begabter Hardware Tüftler bist du mit deiner selbst übertakteten Grafik Hardware ein Meister aller 3D Spiele und hast in World of Warcraft Rang und Namen? Außerdem ist Mathematik eh viel zu langweilig und überflüssig, braucht doch eh keiner, oder? Als eingefleischter Programmierer reicht es dir eh mal ab und zu den ein oder anderen Schnipsel PHP zu programmieren?

Willkommen in der Informatik, wahrscheinlich bist du hier ... falsch.


Spätestens jetzt fragen sich einige wahrscheinlich: Was soll der Blödsinn? Informatik hat doch mit Computern, EDV und Internet zu tun? Um jemanden zu zitieren (von dem ihr in eurem Studium noch einiges lernen werdet – nur Geduld): „Computer Science is no more about computers than astronomy is about telescopes.“ Folgendes werdet ihr also (im wesentlichen) nicht im Informatik Studium lernen:

\begin{itemize}
     \item EDV Grundlagen
     \item Wie kann ich Webseiten gestalten?
     \item Was ist die beste Hardware?
     \item Wo bekomme ich die her?
     \item Wie baue ich mir damit meinen PC zusammen?
     \item Und wie vernetze ich diesen?
     \item Und wenn ich schon dabei bin: Wie stelle ich die neueste Verschlüsselung für meinen 802.11x WLAN Router ein?
\end{itemize}

Das lernt ihr selbst in eurer Freizeit. Vieles davon wird im Studium sogar vorausgesetzt.

Informatik ist eine Wissenschaft. Sie besteht aus Mathematik, Algorithmen, Datenstrukturen, Technik (aha – also doch?) und ihrer Anwendung. Sie beschäftigt sich hauptsächlich mit Problemen und (sofern es sie gibt) deren Lösungen. Wir von der Fachschaft wollen euch natürlich nicht für dumm verkaufen. Wir haben für die diesjährige Dont Panic Interviews mit einigen Professoren von unserem Institut (die ihr mit Sicherheit auch bald kennen lernen werdet) geführt.

%\includegraphics[width=\linewidth]{comics/FuturamaComics/Comicsfinal/Studienwahl1voll}


\begin{flushleft}\underline{Interview mit Prof. Schnitger} \end{flushleft}

\begin{description}

\item[Fachschaft] 

Was bedeutet für sie der Begriff Informatik?

\item[Prof. Schnitger]
 
\textit{Wissenschaft der Information.}

\item[Fachschaft]

Was war ihre Motivation Mathematik zu studieren?

\item[Prof. Schnitger]

\textit{Es ist immer klar worum es geht: Abenteuer im Kopf.}

\item[Fachschaft]

Was fasziniert sie heutzutage an der Informatik?

\item[Prof. Schnitger]

\textit{Vieles beruht auf der Verarbeitung von Informationen: physikalische Prozesse, Steuerung durch Genom, ...}

\item[Fachschaft]

Welche Tipps können sie aus ihrer eigenen Zeit Erstsemestern für das Studium geben?

\item[Prof. Schnitger]

\textit{Arbeiten, arbeiten, ..., und Spaß haben.}

\end{description}



\begin{flushleft}\underline{Interview mit Prof. Schmidt-Schauß} \end{flushleft}

\begin{description}

\item[Fachschaft] 

Was bedeutet für sie der Begriff Informatik?

\item[Prof. Schmidt-Schauß]
 
\textit{Alles was mit  Berechnung, mit Rechnern und Informationsverarbeitung zu tun hat.}

\item[Fachschaft]

Was war ihre Motivation Mathematik zu studieren?

\item[Prof. Schmidt-Schauß]

\textit{Mathematik war mein Interessengebiet. Mit Computern hatte ich zum erstenmal im vierten Semester zu tun im Numerik Praktikum.}

\item[Fachschaft]

Was fasziniert sie heutzutage an der Informatik?

\item[Prof. Schmidt-Schauß]

\textit{Informatik bzw. Computer sind relevant in immer mehr Bereichen des Lebens, Technik und Wissenschaft.
Informatik ist eine junge Wissenschaft und es gibt dort noch zahlreiche offene und interessante Probleme zu lösen.}

\item[Fachschaft]

Welche Tipps können sie aus ihrer eigenen Zeit Erstsemestern für das Studium geben?

\item[Prof. Schmidt-Schauß]

\textit{Grundlagen der Informatik in jeder Form sind wichtig (nicht beschränkt auf die entsprechende Vorlesung). Später im Berufsleben gibt es so gut wie keine Zeit mehr, die Theorie, die jetzt leicht zu lernen ist, nochmals nachzuholen.}

\end{description}


\begin{flushleft}\underline{Interview mit Prof. Hedrich} \end{flushleft}

\begin{description}

\item[Fachschaft] 

Was bedeutet für sie der Begriff Informatik?

\item[Prof. Hedrich]
 
\textit{Informatik ist die Wissenschaft der Informationsverarbeitung im Allgemeinen.
Sie ist durch die rasante Entwicklung in der Computertechnik zu einer sehr
bedeutsamen Wissenschaft geworden und ewist damit sehr viele unterschiedlich
interessante Zweige auf. Von der Robotik bis hin zur Logik und Komplexitätstheorie.}


%\includegraphics[width=\linewidth]{comics/FuturamaComics/Comicsfinal/Studienwahl4voll}

\item[Fachschaft]

Was war ihre Motivation Elektrotechnik zu studieren?

\item[Prof. Hedrich]

\textit{Während des Studiums konnte ich mein Interesse für die Computer (ich hab
in der Schule mit einem Commodore PET angefangen und mich dann in
die C64 Technik vertieft) mit dem für Elektrotechnik verbinden.
So kam es zu meinem Forschungsinteressen, den rechnergestützten Entwurf von
Schaltungen.}

\item[Fachschaft]

Was fasziniert sie heutzutage an der Informatik?

\item[Prof. Hedrich]

\textit{Das weite Anwendungsfeld und besonders auch die großen
Gestaltungsmöglichkeiten, da Informatik ja inzwischen fast überall
zu finden ist. Speziell finde ich natürlich die technischen Entwicklungen
spannend z.B. den aktuellen Umstieg von Single-Prozessorsystemen auf
Multi-Cores bis hin zu 800 Prozessoren auf einer Grafikkarte für
jedermann.}

\item[Fachschaft]

Welche Tipps können sie aus ihrer eigenen Zeit Erstsemestern für das Studium geben?

\item[Prof. Hedrich]

\textit{Schauen Sie sich die Vorlesungen an. Es kann sowohl fachlich als auch
menschlich interessant sein. Ein guter Studienkommilitone hilft enorm,
sowohl bei der Auswahl der richtigen Veranstaltungen als auch während der
ganzen Lernerei. Und ein bisschen Entspannung bei einer Fete ist am Ende einer Woche
auch nicht verkehrt.}

\end{description}


%\includegraphics[width=\linewidth]{comics/FuturamaComics/Comicsfinal/Studienwahl5voll}

\begin{flushright} Grzegorz \end{flushright}
